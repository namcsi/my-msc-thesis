\subsection{HT and Equilibrium Logic}

\textbf{Don't forget to declare somewhere that sections 1-2 were part
  of your IM.}

The logic of Here-and-There forms the monotonic basis for defining
it's non-monotonic counterpart, equilibrium logic, which can in turn
be used to give a purely logical semantics for Answer Set Programming
(ASP) \cite{pearce06a}. In this subsection, we present the standard
definitions of these logics, and in the following subsection extend
them in the standard way into the temporal setting.

We define a signature $\A$ as a set, the elements of which $a\inA$ we
call atoms. A formula over a signature $\A$ is defined by the grammar:

\begin{align*}
    \varphi ::= &\; a \mid \bot \mid
                  \varphi_1 \wedge \varphi_2 \mid
                  \varphi_1 \vee \varphi_1 \mid
                  \varphi_1 \to \varphi_1
\end{align*}

We refer to the set of all possible formulas built from the grammar
above over signature $\A$ as the language $\langa$, and a
subset of $\langa$ as a theory.  We also define as shorthands
$\neg \varphi \defeq \varphi \to \bot$, $\top \defeq \bot \to \bot$
and
$\varphi \leftrightarrow \psi \defeq (\varphi \rightarrow \psi) \wedge
(\varphi \leftarrow \psi)$.

A specific syntactic subset of formulas of a particular form are of
greater interest to us due to their relevance to logic programming.
We define the set of all literals over $\A$ as
$\textit{Lit}(\A) = \{ a, \lnot a\ |\ a \in \A \}$.

\begin{definition}[Disjunctive rule]
    We call formulas of the following form disjunctive rules, or rules for short:
\begin{align*}
  b_{1} \wedge \cdots \wedge b_{m} \rightarrow h_1 \vee \cdots \vee h_{n}
\end{align*}
where $h_{i}, b_{j} \in \textit{Lit}(\A)$. We call the set of literals $\{ h_i | 1 \leq i \leq n, i\in \mathbb{N}\}$ the \emph{head} of the rule, and the set of literals $\{ b_j | 1 \leq j \leq m, j \in \mathbb{N}\}$ the \emph{body} of the rule.
\end{definition}

We call a theory $P$ containing only (disjunctive) rules a
(disjunctive) logic program.

An HT interpretation over $\A$ is an ordered pair of sets of atoms
$\handt$, where $H \subseteq T \subseteq \A$. Intuitively, the set of
atoms in $H$ and $T$ describe what is true in worlds ``here'' and
``there'', respectively. The satisfaction relation for formulas is
defined as follows:

\begin{definition}[HT satisfaction]
    An interpretation $\handt$ satisfies a formula $\varphi$, written $\handt \models \varphi$, according to the following definition:
    \begin{description}
        \item $\handt \not \models \bot$
        \item $\handt \models a$ iff $a\in H$
        \item $\handt \models \varphi_1 \wedge \varphi_2$ iff $\handt \models \varphi_1$ and  $\handt \models \varphi_2$
        \item $\handt \models \varphi_1 \vee \varphi_2$ iff $\handt \models \varphi_1$ or  $\handt \models \varphi_2$
        \item $\handt \models \varphi_1 \to \varphi_2$ iff $\langle w,T \rangle \not \models \varphi_1$ or  $\langle w,T \rangle \models \varphi_2\;$ for all 
        $w \in \{H,T\}$
    \end{description}
\end{definition}

We say that an interpretation $\handt$ is an HT model of theory
$\Gamma$ iff $\handt \models \varphi$ for all $\varphi \in \Gamma$. We
denote the set of all HT models of a theory $\Gamma$ as
$\text{HT}(\Gamma)$. A formula $\varphi$ is a HT tautology, written as
$\models \varphi$, if $\handt \models \varphi$ for any HT
interpretation. We call the logic induced by the set of all
tautologies the logic of Here-and-There (HT for short). Two formulas
$\varphi$ and $\psi$ are said to be HT equivalent, denoted as
$\varphi \equivht \psi$, iff $\models \varphi \leftrightarrow \psi$.

The characteristic property of persistence in HT can be proven via
structural induction.

\begin{proposition}[Persistence]
  For any HT interpretation $\handt$, if $\handt \models \varphi$,
  then $\tandt \models \varphi$.
\end{proposition}



We call an HT interpretation $\handt$ total if $H=T$. By imposing an
additional minimality condition on total interpretations, we arrive at
the concept of an equilibrium model, and of a stable model.

\begin{definition}[Equilibrium Model/Stable model \cite{lipeva01a} \cite{pearce06a}]
  A total HT interpretation $\tandt$ is said to be an equilibrium
  model of a theory $\Gamma$ iff $\tandt \models \Gamma$ and there is
  no HT interpretation $\handt$ such that $\handt \models \Gamma$ and
  $H \subset T$. A set of atoms $T$ is a stable model of a theory
  $\Gamma$ iff $\tandt$ is an equilibrium model of $\Gamma$. We denote
  the set of all equilibrium models of a theory $\Gamma$ as
  $\text{EL}(\Gamma)$.
\end{definition}

It is worth noting that this definition of a stable model above is
equivalent to the original reduct-based characterization
\cite{lipeva01a}, though we will not be making use of the reduct in
this thesis. 

We would now like to characterize the necessary and sufficient
conditions when two theories have the same answer sets in any context.

\begin{definition}[Strong Equivalence \cite{lipeva01a}]
  Two theories $\Gamma_1$ and $\Gamma_2$ are strongly equivalent iff for
  any theory $\Gamma$, $\Gamma_1 \cup \Gamma$ and
  $\Gamma_2 \cup \Gamma$ have the same stable models.
\end{definition}

Strong equivalence is a particularly significant property in a logic
programming context, as it allows one to replace parts of a program
with a strongly equivalent counterpart without affecting the stable
models of the program as a whole. Requiring the usual notion of
equivalence of theories in equilibrium logic, i.e. that $\Gamma_1$ and
$\Gamma_2$ possess the same equilibrium models, is not sufficient to
guarantee strong equivalence due to the non-monotonic nature of
equilibrium logic \cite{lipeva01a}.  However, equivalence in the
non-monotonic basis of HT is sufficient, and in fact also necessary.

\begin{proposition}[Strong Equivalence and HT equivalence \cite{lipeva01a}]
  Two theories $\Gamma_1$ and $\Gamma_2$ are strongly equivalent
  iff $\Gamma_1 \equivht \Gamma_2$
\end{proposition}

HT equivalence also satisfies a similar replacement property
to the one known in classical logic.

\begin{proposition}[Replacement property \cite{petowo09a} \cite{agcadipescscvi20a}]
  Let $\gamma[\varphi]$ denote a formula with some occurrence of
  subformula $\varphi$ and let $\gamma[\psi]$ be the formula resulting
  from replacing said occurrence of $\varphi$ with $\psi$. If
  $\varphi \equivht \psi$, then $\gamma[\varphi] \equivht \gamma[\psi]$.
\end{proposition}

HT can also be defined as a 3-valued logic, known commonly as Gödel's
3-valued logic \cite{capeva05a}. A 3-valued interpretation
$I: \A \rightarrow \textbf{3}$ assigns to every atom $a\in \A$ a value
from $\textbf{3}=\{ 0, 1, 2 \}$. We define a one-to-one correspondence
between 3-valued interpretations and HT interpretations via the
following mapping. Given an HT interpretation $\handt$, the
corresponding 3 valued interpretation $I_{\handt}$ is defined as
$I_{\handt}(a)=2$ iff $a \in H$, $I_{\handt}(a)=1$ iff
$a \in T \setminus H$ and $I_{\handt}(a)=0$ iff $a \not\in T$. Given a
3-valued interpretation $I$, the corresponding HT interpretation
$\handt_{I}$ is produced by the inverse of the mapping above. In the
following, we will occasionally use HT interpretations $\handt$ and
their corresponding 3-valued interpretations $I_{\handt}$
interchangeably when this does not lead to confusion.

A 3-valued interpretation $I$ can be extended to $\langa$
inductively using the following rules:
\begin{definition}[Extension of 3-valued interpretation]\label{def:3-valued-extension}
  \begin{align*}
    &I(\bot) = 0
    &&I(\top) = 2 \\
    &I(\varphi \wedge \psi) = min(I(\varphi),I(\psi)) 
    &&I(\varphi \vee \varphi) = max(I(\varphi),I(\psi)) \\
    &I(\varphi \rightarrow \psi) = 
      \begin{cases}
        2 & \text { if } I(\varphi) \leq I(\psi) \\
        I(\psi) & \text { otherwise }
      \end{cases}
    &&I(\neg \varphi) = \begin{cases}2 
      & \text { if } I(\varphi)=0 \\ 
      0 & \text { otherwise }\end{cases}
  \end{align*}
\end{definition}

The values assigned to formulas by a 3-valued interpretation can also
be characterized via HT models, as shown by the following proposition.

\begin{proposition}[\cite{capeva05a}]\label{prop:3-valued-ht}
For any 3-valued HT interpretation $I$ and formulas $\varphi$, $\psi$:
\begin{description}
  \item $I_{\handt}(\varphi) = 2$ iff $\handt \models \varphi$ 
  \item $I_{\handt}(\varphi) = 1$ iff $\tandt \models \varphi$ and $\handt \not\models \varphi$
  \item $I_{\handt}(\varphi) = 0$ iff $\tandt \not\models \varphi$
  \item  $I_{\handt}(\varphi) = I_{\handt}(\psi)$ iff $\handt \models \varphi \leftrightarrow \psi$
\end{description}
\end{proposition}

As a consequence of Proposition \ref{prop:3-valued-ht}, we can
equivalently define when a 3-valued interpretation satisfies a
formula, as follows: $I \models \varphi$ iff $I(\varphi)=2$.

As evident from the proposition, the HT formulation is more verbose,
and thus more cumbersome to use when all 3 values assigned to formulas
are relevant. 3-valued interpretations also allow us to check
satisfaction or (strong) equivalence of formulas via a simple
calculation of truth values according to the rules in Definition
\ref{def:3-valued-extension}.

On a more philosophical note, one way of conceptualizing the three
values is to view the atoms in $H$ as being \textit{certainly true},
$T \setminus H$ as being \textit{true by default} and $\A \setminus T$
as being \textit{false} \cite{capeva17a}. Total interpretations can
then be thought of as interpretations in which every atom that is true
is true with certainty. An equilibrium model of a theory $\Gamma$ is in
turn an interpretation according to which $\Gamma$ is certainly true,
any true atoms must be true with certainty, and none of the truth
values of any of these certainly true atoms may be weakened to be only
true by default, while also keeping $\Gamma$ certainly true.
