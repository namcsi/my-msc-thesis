\subsection{THT and TEL}

Linear time temporal logic (LTL) enhances propositional logic with a
linear sequence of worlds and temporal modal operators which allows us
to express temporal conditions over these sequence of worlds, such as
a formula holding in all future time points, or a formula eventually
holding in some future time point. To define a non-monotonic version
of LTL, we take similar approach as in propositional logic, defining
first a monotonic Here-and-There variant, among which we define models
which are in equilibrium via a totality and minimality condition
\cite{agcadipescscvi20a}.

A temporal formula over a set of atoms $\A$ is defined by the
following grammar:

\begin{align*}
    \varphi ::= &\; a \mid \bot \mid
                  \varphi_1 \otimes \varphi_2 \mid
                  \previous \varphi \mid \varphi_1 \since \varphi_2 \mid \varphi_1 \trigger \varphi_2 \mid
                  \Next \varphi \mid \varphi_1 \until \varphi_2 \mid \varphi_1 \release \varphi_2 \mid                  
\end{align*}
where $\otimes \in \{ \wedge, \vee, \to \}$ is a binary Boolean
connective. A a

