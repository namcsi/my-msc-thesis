\subsection{Temporal Here-and-There and Temporal Equilibrium
  Logic}\label{subsec:tht-tel}

Linear time temporal logic (LTL) enhances propositional logic with a
linear sequence of worlds and temporal modal operators which allows us
to express temporal conditions over these sequence of worlds, such as
a formula holding in all future time points, or a formula eventually
holding in some future time point. To define a non-monotonic version
of LTL, we take similar approach as in propositional logic, defining
first a monotonic Here-and-There variant, among which we define models
which are in equilibrium via a totality and minimality condition. The
definitions and propositions of this section follow those of
\citeauthor{agcadipescscvi20a}\cite{agcadipescscvi20a}, with some
modifications.

A temporal formula over a set of atoms $\A$ is defined by the
following grammar:

\begin{align*}
    \varphi ::=   \; a &\mid \bot \mid \varphi_1 \otimes \varphi_2 \mid\\
  \initially &\mid \previous \varphi \mid \wprevious \varphi \mid \eventuallyP \varphi \mid
  \alwaysP \varphi \mid \varphi_1 \since \varphi_2 \mid \varphi_1 \trigger \varphi_2 \mid\\
  \finally &\mid \Next \varphi \mid \wnext \varphi \mid \eventuallyF \varphi \mid
  \alwaysF \varphi \mid \varphi_1 \until \varphi_2 \mid \varphi_1 \release \varphi_2
\end{align*}
where $\otimes \in \{ \wedge, \vee, \to \}$ is a binary Boolean
connective. We denote set of all temporal formulas over $\A$ as the
language $\langat$, and a subset of
$\langat$ as a temporal theory. We sometimes drop the
temporal adjective from the definitions of this subsection when it is
clear from context that we are talking about the temporal setting. The
operators can be split into groups based on whether they act on past
or future time points, and can be distinguished visually; past
operators are filled, while future operators are outlined. The
temporal operators are read as follows:

\[
  \begin{array}{l l|l l}
    \initially & \text{\emph{initial}} & \finally & \text{\emph{final}}\\
    \previous & \text{\emph{previous}} & \Next & \text{\emph{next}}\\
    \wprevious & \text{\emph{weak previous}} & \wnext & \text{\emph{weak next}}\\
    \eventuallyP & \text{\emph{eventually before}} & \eventuallyF & \text{\emph{eventually after}}\\
    \alwaysP & \text{\emph{always before}} & \alwaysF & \text{\emph{always after}}\\
    \since & \text{\emph{since}} & \until & \text{\emph{until}}\\
    \trigger & \text{\emph{trigger}} & \release & \text{\emph{release}}
\end{array}
\]

To give an informal summary of the semantics of some of the operators,
the formulas $\previous \varphi$/$\Next \varphi$ say that $\varphi$ is
true at previous/next time point, respectively.  $\varphi \since \psi$
says that $\varphi$ has been true since some time point in the past
where $\psi$ was true, while $\varphi \until \psi$ says that $\varphi$
must be true until some time point in the future where $\psi$ is true.
The meaning of trigger and release are somewhat less
intuitive. $\varphi \trigger \psi$ mean that $\psi$ has been true
either for all time steps in the past (including the current), or
since some time point in the past when $\varphi$ and $\psi$ were both
true. $\varphi \release \psi$ in turn means that $\psi$ will be true
either for all time steps in the future (including the current), or
until some time point in the future when $\varphi$ and $\psi$ will
both be true.

A more precise discussion of the semantics of the temporal operators
requires us to first fix some terminology and notation. Let $\omega$
denote the first infinite ordinal. For any $a \in \mathbb{N}$ and
$b \in \mathbb{N} \cup {\omega}$, we introduce the notation
$\intervcc{a}{b} \defeq \{ i \mid i \in \mathbb{N}, a \leq i \leq b
\}$,
$\intervco{a}{b} \defeq \{ i \mid i \in \mathbb{N}, a \leq i < b \}$,
$\intervoc{a}{b} \defeq \{ i \mid i \in \mathbb{N}, a < i \leq b \}$
and $\intervoo{a}{b} \defeq \{ i \mid i \in \mathbb{N}, a < i < b
\}$. A trace $\bm{T}$ of length $\lambda$ over signature $\A$ is then
defined as a (possibly infinite) sequence of sets of atoms
$\bm{T} \in (2^{\A})^{\intervco{0}{\lambda}}$. We denote by $T_i$ the
$i$-th element of a trace (i.e. $\bm{T}(i)$), where
$\rangeco{i}{0}{\lambda}$. For two traces $\bm{H}$ and $\bm{T}$, we
write $\bm{H < T}$ iff $H_i \subseteq T_i$ for all
$\rangeco{i}{0}{\lambda}$, and there is some $\rangeco{j}{0}{\lambda}$
such that $H_j \subsetneq T_j$. We write $\bm{H \leq T}$ iff
$\bm{H < T}$ or $\bm{H} = \bm{T}$

If for each time point we consider two worlds, a "here" and a "there",
instead of a single world, we arrive at the notion of an HT
trace. Formally, an HT trace $\thandt$ of length $\lambda$ is an
ordered pair of traces, where for each $\rangeco{i}{0}{\lambda}$
$H_i \subseteq T_i$. An $HT$ trace is total iff $\bm{H} = \bm{T}$. We
are now ready to extend the logic of HT with the temporal operators
defined above, resulting in the logic of Temporal Here-And-There, or
THT for short.

\begin{definition}[THT satisfaction \cite{agcadipescscvi20a}]\label{def:tht-sat}
  An HT-trace $\thandt$ of length $\lambda$ over signature $\A$
  satisfies a temporal formula $\varphi \in \langat$ at
  time point $\kinlambda$, written as $\thandt,k \models \varphi$ if
  the following condition holds, where we use the shorthand $\bf{M} =$
  $\thandt$.
\begin{enumerate}
  \item $\bf{M}, k \not \models \bot$
  \item $\bf{M},k \models$ $a$ if $a \in H_{k}$ for any atom $a \in \mathcal{A}$
  \item $\bf{M},k \models \varphi \wedge \psi$ iff $\bf{M}, k \models \varphi$ and $\bf{M}, k \models \psi$
  \item $\bf{M},k \models \varphi \vee \psi$ iff $\bf{M}, k \models \varphi$ or $\bf{M}, k \models \psi$
  \item $\bf{M},k \models \varphi \rightarrow \psi$ iff 
    $\langle \bm{H}^{\prime},\bm{T} \rangle, k \not \models \varphi$ 
    or $\langle \bm{H}^{\prime},\bm{T} \rangle, k \models \psi$, 
    for all $\bm{H}^{\prime} \in\{\bm{H}, \bm{T}\}$
  \item $\bf{M},k \models \initially$ iff $k=0$
  \item $\bf{M},k \models \previous \varphi$ iff $k>0$ and $\bf{M}, k-1 \models \varphi$
  \item $\bf{M},k \models \wprevious \varphi$ iff $k=0$ or $\bf{M}, k-1 \models \varphi$
  \item $\bf{M},k \models \eventuallyP \varphi$ iff $j \models \varphi$ for some $\rangecc{j}{0}{k}$  
  \item $\bf{M},k \models \alwaysP \varphi$ iff $j \models \varphi$ for all $\rangecc{j}{0}{k}$  
  \item $\bf{M},k \models \varphi \since \psi$ iff for some $j \in[0 . . k]$, we have $\bf{M}, j \models \psi$ and $\bf{M},i \models \varphi$ for all $\rangeoc{i}{j}{k}$
  \item $\bf{M},k \models \varphi \trigger \psi$ iff for all $j \in[0 . . k]$, we have $\bf{M}, j \models \psi$ or $\bf{M}, i \models \varphi$ for some $\rangeoc{i}{j}{k}$
  \item $\bf{M},k \models \finally$ iff $k=\lambda - 1$
  \item $\bf{M},k \models \Next \varphi$ iff $k+1<\lambda$ and $\bf{M}, k+1 \models \varphi$
  \item $\bf{M},k \models \wnext \varphi$ iff $k=\lambda-1$ or $\bf{M},k+1 \models \varphi$
  \item $\bf{M},k \models \eventuallyF \varphi$ iff $j \models \varphi$ for some $\rangeco{j}{k}{\lambda}$  
  \item $\bf{M},k \models \alwaysF \varphi$ iff $j \models \varphi$ for all $\rangeco{j}{k}{\lambda}$  
  \item $\bf{M}, k \models \varphi \until \psi$ iff for some $j \in[k . . \lambda)$, we have $\bf{M}, j \models \psi$ and $\bf{M}, i \models \varphi$ for all $\rangeco{i}{k}{j}$
  \item $\bf{M}, k \models \varphi \release \psi$ iff for all $\rangeco{j}{k}{\lambda}$, we have $\bf{M}, j \models \psi$ or $\bf{M},
    i=\varphi$ for some $\rangeco{i}{k}{j}$
\end{enumerate}
\end{definition}

We say that an HT-trace $\thandt$ is a THT model of a temporal formula
$\varphi$ iff $\thandt,0 \models \varphi$. We write
$\thandt \models \varphi$ iff $\thandt,0 \models \varphi$, and for any
theory $\Gamma \subseteq \langat$ we write $\thandt \models \Gamma$
iff $\varphi \models \Gamma$ for all $\varphi \in \Gamma$. We denote
the set of all THT models of a theory $\Gamma$ of length $\lambda$ as
$\text{THT}(\Gamma,\lambda)$. A formula $\varphi$ is a THT tautology,
written as $\models \varphi$, if $\handt,k \models \varphi$ for any
HT-trace $\thandt$ and $\kinlambda$. We call the logic induced by the
set of all tautologies the logic of Temporal Here-and-There (THT for
short). Two formulas $\varphi$ and $\psi$ are said to be THT
equivalent, denoted as $\varphi \equivtht \psi$, iff
$\models \varphi \leftrightarrow \psi$. Note that while THT models are
only required to satisfy a formula at the initial time step,
tautologies (and thus THT equivalence) must be check not only for any
HT-trace, but also any time point.

Temporal equilibrium models and temporal stable models are defined
similarly as in the HT case.

\begin{definition}[Temporal Equilibrium Model/Temporal Stable Model \cite{agcadipescscvi20a}]
  A total HT-trace $\ttandt$ is said to be a temporal equilibrium
  model of a temporal theory $\Gamma$ iff $\ttandt \models \Gamma$,
  and there is no HT-trace $\thandt$ such that
  $\thandt \models \Gamma$ and $\bm{H} < \bm{T}$. A trace
  $\bm{T}$ is a temporal stable model of a temporal theory $\Gamma$
  iff $\ttandt$ is a temporal equilibrium model of $\Gamma$.

  We denote the set of all temporal equilibrium models of a theory
  $\Gamma$ of length $\lambda$ as $\text{TEL}(\Gamma,\lambda)$. We
  call the logic determined by the temporal equilibrium models of a
  theory \emph{Temporal Equilibrium Logic}.
\end{definition}

\begin{example}\label{exam:shoot-ground-theory}
  Let us consider as an example the following temporal theory, which
  models the dynamics of shooting a single-shot potato cannon:
\begin{equation*}
\begin{gathered}
\textit{shoot}\\
\Next\textit{not\_load}\\
\Next\Next\textit{shoot}\\
\alwaysF(\textit{shoot} \wedge \previous(\textit{not\_load} \since \textit{shoot})
\rightarrow \eventuallyF \textit{fail} \wedge \textit{forgetful}
\end{gathered}
\end{equation*}

The first three formulas state that at time point 0 the potato cannon
is shot, it is not loaded at time point 1, and it is shot again at
time point 2. The final formula states that if at some time point a
potato cannon is shot and it has been shot at some time point in the
past, with the potato cannon not being loaded in any intervening time
points, then it will eventually fail, and the person forgot to load it
in the intervening time has proven to be forgetful.

Setting $\lambda=4$, the temporal equilibrium models of this theory
are
\begin{align*}
\bm{T_1}&=(\{\textit{shoot}\}, \{\textit{not\_load}\},
\{\textit{shoot}, \textit{forgetful}\}, \{ \textit{fail} \})\\
\bm{T_2}&=(\{\textit{shoot}\}, \{\textit{not\_load}\},
\{\textit{shoot}, \textit{forgetful}, \textit{fail}\}, \{ \})
\end{align*}
\end{example}

Strong equivalence of two formulas in THT, as in HT, allows us to
replace one formula with another within a theory without affecting the
temporal stable models of the theory as a whole.

\begin{definition}[Temporal Strong Equivalence \cite{agcadipescscvi20a}]
  Two temporal theories $\Gamma_1$ and $\Gamma_2$ are strongly
  equivalent iff for any temporal theory $\Gamma$,
  $\Gamma_1 \cup \Gamma$ and $\Gamma_2 \cup \Gamma$ have the same
  temporal stable models.
\end{definition}

\begin{proposition}[Temporal Strong Equivalence and THT Equivalence \cite{agcadipescscvi20a}]
  Two temporal theories $\Gamma_1$ and $\Gamma_2$ are strongly equivalent
  iff $\Gamma_1 \equivtht \Gamma_2$
\end{proposition}

\begin{proposition}[Temporal Replacement Property \cite{agcadipescscvi20a}]
  Let $\gamma[\varphi]$ denote a temporal formula with some occurrence of a
  subformula $\varphi$ and let $\gamma[\psi]$ be the temporal formula resulting
  from replacing said occurrence of $\varphi$ with $\psi$. If
  $\varphi \equivtht \psi$, then $\gamma[\varphi] \equivtht \gamma[\psi]$.
\end{proposition}

While THT has a considerable number temporal operators, one can fix a
set of core operators, and express all other, so called \emph{derived
  operators} in terms of the core. One can for instance start from the
core operators
$\initially, \previous, \since, \trigger, \finally, \Next, \until,
\release$, as shown by the following proposition.

\begin{proposition}[Derived Operators\cite{agcadipescscvi20a}]\label{prop:derived-op}
\[
\begin{array}{lll|lll}
\wprevious \varphi & \equivtht & \previous \varphi \vee \initially &
\wnext \varphi & \equivtht & \Next \varphi \vee \finally\\
\eventuallyP \varphi & \equivtht & \top \since \varphi &
\eventuallyF \varphi & \equivtht & \top \until \varphi\\
\alwaysP \varphi & \equivtht & \bot \trigger \varphi &
\alwaysF \varphi & \equivtht & \bot \release \varphi\\
\end{array}
\]
\end{proposition}

For formulas $\gamma$ of the above form (containing a top-level
derived operator), we call it's THT equivalent rewriting as given
above, it's \textit{derivation}. We map such formulas to their
derivation via the function $der(\gamma)$, e.g. for
$\gamma = \wprevious \varphi$,
$der(\wprevious \varphi) = \previous \varphi \vee \initially$. We note
that the $\initially$ and $\finally$ operator could also be considered
as derived operators (but we do not do so), as
$\initially \equivtht \lnot \previous \top$ and
$\finally \equivtht \lnot \Next \top$.

THT, similarly to HT, can also be characterized via a 3-valued THT
interpretation $m$. In the case of THT though, the interpretation is
also parameterized by the time point. A 3-valued (temporal)
interpretation is a mapping
$m: \intervco{0}{\lambda} \times \A \rightarrow \textbf{3}$. Similarly
to the non-temporal case, there is a one-to-one correspondence; for
any HT trace $\thandt$, we define a mapping to 3-valued interpretation
$m_{\thandt}$ as $m_{\thandt}(k,a)=2$ iff $a \in H_k$,
$m_{\thandt}(k,a)=1$ iff $a \in T_k \setminus H_k$ and
$m_{\thandt}(k,a)=0$ iff $a \not\in T_k$. Given a 3-valued
interpretation $m$, the corresponding HT trace $\thandt_m$ given by the
inverse of this mapping.

A 3-valued interpretation can be extended to arbitrary temporal
formulas using the rules below \cite{agcadipescscvi20a}, where
$\operatorname{imp}(x,y)=2$ if $x \leq y$, otherwise
$\operatorname{imp}(x,y)=y$.
\begin{definition}[Extension of 3-valued Temporal Interpretation]\label{def:3-valued-extension-temporal}
\begin{align*}
  m(k, \perp) &= 0 \\
  m(k, \varphi \wedge \psi) &= \min (m(k, \varphi), m(k, \psi)) \\
  m(k, \varphi \vee \psi) &= \max (m(k, \varphi), m(k, \psi)) \\
  m(k, \varphi \rightarrow \psi) &= \operatorname{imp}(m(k, \varphi), m(k, \psi)) \\
  m(k, \previous \varphi) &= \begin{cases}
    0 & \text { if } k=0 \\
    m(k-1, \varphi) & \text { if } k>0
  \end{cases} \\
 m(k, \varphi \since \psi) &= \max \{\min (m(j, \psi), \min \{m(i, \varphi) \mid j<i \leq k\}) \mid 0 \leq j \leq k\} \\
 m(k, \varphi \trigger \psi) &= \min \{\max (m(j, \psi), \max \{m(i, \varphi) \mid j<i \leq k\}) \mid 0 \leq j \leq k\} \\
 m(k, \Next \varphi) &= \begin{cases}0 & \text { if } k+1=\lambda(\neq \omega) \\
m(k+1, \varphi) & \text { if } k+1<\lambda\end{cases} \\
 m(k, \varphi \until \psi) &= \max \{\min (m(j, \psi), \min \{m(i, \varphi) \mid k \leq i<j\}) \mid k \leq j<\lambda\} \\
 m(k, \varphi \release \psi) &= \min \{\max (m(j, \psi), \max \{m(i, \varphi) \mid k \leq i<j\}) \mid k \leq j<\lambda\}
\end{align*}
\end{definition}

The 3-valued THT interpretation satisfies similar properties as the
3-valued HT interpretation discussed in the previous section.

\begin{proposition}[\cite{capeva05a}]\label{prop:3-valued-temporal-properties}
  For any HT trace $\thandt$, $\kinlambda$ and temporal
  formulas $\varphi$, $\psi$, the following assertions hold:
\begin{description}
\item $m_{\thandt}(k,\varphi) = 2$ iff $\thandt,k \models \varphi$
\item $m_{\thandt}(k,\varphi) = 1$ iff $\ttandt,k \models \varphi$ and $\thandt,k \not\models \varphi$
  \item $m_{\thandt}(k,\varphi) = 0$ iff $\ttandt,k \not\models \varphi$
\item $m_{\thandt}(k,\varphi) = m_{\thandt}(k,\psi)$ iff $\thandt,k \models \varphi \leftrightarrow \psi$
\end{description}
\end{proposition}

We write $m \models \varphi$ iff $m(0,\varphi)=2$. The extension of
3-valued interpretations to the derived operators can be inferred from
Definition \ref{def:3-valued-extension-temporal} by noting that for
any derived operator $\gamma$ and $\kinlambda$,
$m(k,\gamma)=m(k,der(\gamma))$ due to strong equivalence. We refer the
reader to \cite{agcadipescscvi20a} for the concrete values.

THT equivalence can characterized via 3-valued interpretations, similarly as in HT.

\begin{proposition}\label{prop:char-strong-equiv-3val}
For any $\varphi$, $\psi$ the following assertions are equivalent:
  \begin{enumerate}[label={(\arabic*)}]
    \setlength\itemsep{0.15em}
    \item $\varphi \equivtht \psi$
    \item For all 3-valued interpretations $m$ and $\kinlambda$: $m(k,\varphi) = m(k,\psi)$
    \item For all 3-valued interpretations $m$ and $\kinlambda$: 
      
$m(k,\varphi) = 2$ iff $m(k,\psi) = 2$
  \end{enumerate}
\end{proposition}

Finally, we distinguish the syntactic fragment that we will restrict
our discussion to in the rest of the paper.

\begin{definition}[Temporal Rule, Temporal Program]
  We call a temporal formula $\varphi \in \langat$ a
  \emph{implication-free} iff $\varphi$ contains no implications (with
  the exception of negation). We denote the set of implication-free
  temporal formulas as $\langfat$.
  
  We call a temporal formula a temporal rule, iff it is of the form:
  $$
  \tempruleshort
  $$
  where $B=b_i \land \dots \land b_n$, $H=h_1 \lor \dots \lor h_m$,
  and $b_i,h_j \in \langfat$ for $\rangecc{i}{1}{n}$,
  $\rangecc{j}{1}{m}$.  

  We call a set of temporal rules a temporal logic program.
\end{definition}

\begin{example}\label{exam:shoot-ground-program}
  We can represent the theory in Example
  \ref{exam:shoot-ground-theory} equivalently as the following
  temporal logic program:
\begin{equation*}
\begin{gathered}
\alwaysF (\initially \rightarrow \textit{shoot})\\
\alwaysF (\initially \rightarrow \Next\textit{not\_load})\\
\alwaysF (\initially \rightarrow \Next\Next\textit{shoot})\\
\alwaysF(\textit{shoot} \wedge \previous(\textit{not\_load} \since \textit{shoot})
\rightarrow \eventuallyF \textit{fail} \wedge \textit{forgetful})
\end{gathered}
\end{equation*}
\end{example}
