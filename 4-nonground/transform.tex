\section{Non-Ground Metaprogramming}

In this section we present the
\emph{renopro}\footnote{\url{https://github.com/krr-up/renopro}}
system (\emph{re}ification for \emph{no}nground \emph{pro}grams), which
facilitates metaprogramming over non-ground ASP programs in the
language of gringo.

While meta-programming for ground programs are supported natively by
gringo, until this point no such facilities have existed for
non-ground programs. The renopro system supports the reification of
gringo programs by serializing the abstract syntax tree (AST) of,
which can then serve as input to a non-ground program. . The system
furthermore supports the inverse of this operation; that is, the
transformation of a set of facts describing a valid AST in renopro's
reification format back into a gringo program. We call this inverse
operation \emph{reflection}.

In the following subsections, we will discuss two operations of
reification and reflection in more detail, and finally, describe a
functionality of renopro that combines the two; namely
\emph{transformation}. This functionality first reifies an input
non-ground program. The reified facts are then given as input to a
\emph{transformation meta-encoding}, the stable models of which
describes a set of transformations to be applied to the
program. Finally, the transformed set of facts are reflected back into
a gringo program, completing the transformation. As an example
application of this transformation functionality, we will present a
meta-encoding which generates the set of externals required to
correctly ground a universal temporal logic program, as described at
the end of Section \ref{sec:non-ground}.

\subsection{Reification}

Due to the richness of the gringo language, the reification format of
renopro is much more complex than the ground reification format. We
will therefore not describe the format in it's entirety, and restrict
the discussion to the subset necessary to describe our example
use-case of external generation. We note that the reification format
has an almost one-to-one correspondence to the AST representation of
the clingo python
API\footnote{\url{https://potassco.org/clingo/python-api/current/clingo/ast.html}},
and as such, the api can serve as a guide to the format. Let us first
take the following program, and describe how it is reified by renopro.

\begin{lstlisting}[language=clingo]
c(a(X)) :- b(X), not d. 
b(1).
\end{lstlisting}

This program can be reified by the renopro system by calling:

\begin{lstlisting}[language=bash,numbers=none]
renopro reify prg.lp
\end{lstlisting}

assuming the program above is saved as prg.lp. The reified
representation we would see printed would contains the following set
of facts:

\begin{lstlisting}[language=clingo,label{lst:nong-reified}]
program(0,"base",constants(1),statements(2)).
  statements(2,0,rule(3)).
    rule(3,literal(4),body_literals(11)).
      literal(4,"pos",symbolic_atom(5)).
        symbolic_atom(5,function(6)).
          function(6,c,terms(7)).
            terms(7,0,function(8)).
              function(8,a,terms(9)).
                terms(9,0,variable(10)).
                  variable(10,"X").
      body_literals(11,0,body_literal(12)).
        body_literal(12,"pos",symbolic_atom(13)).
          symbolic_atom(13,function(14)).
            function(14,b,terms(15)).
              terms(15,0,variable(16)).
                variable(16,"X").
      body_literals(11,1,body_literal(17)).
        body_literal(17,"not",symbolic_atom(18)).
          symbolic_atom(18,function(19)).
            function(19,d,terms(20)).
  statements(2,1,rule(21)).
    rule(21,literal(22),body_literals(27)).
      literal(22,"pos",symbolic_atom(23)).
        symbolic_atom(23,function(24)).
          function(24,b,terms(25)).
            terms(25,0,number(26)).
              number(26,1).
\end{lstlisting}

The facts above are ordered an indented in a way that mimics the tree
structure of the AST it encodes. Each fact corresponds to a node in
the AST of the program. 

The first argument of each facts is an integer id of the fact. The id
of each fact is unique (with the exception of tuple facts described in
the next paragraph), and is not necessarily required to be an integer,
but integers are used by the reification implementation. We call the
id wrapped in a function term with the same name as the fact, the
fact's \emph{identifier}. The identifier of a fact can occur in other
facts, indicating a child-parent relationship between the two facts.
Take as example the fact on line 3, encoding a rule node in the
AST. The identifier \texttt{literal(4)} that serves as the second
argument of the fact, indicates that the literal fact with id 4 is the
head node of the rule (and is thus a child node of the rule node).

There are also facts that encode an element of a tuple of AST
nodes. Such tuples occur in the AST when there are a variable number
of nodes that may appear as children, such as the literals forming the
body of a rule, as rules may have any number literals in their
body. For such facts that encode a tuple element, the second argument
indicates the position of the element in the tuple, with the third and
final argument being the identifier of the element. On line 3 for
example, we see that the rule with id 3 has a tuple of body literals
with identifier \texttt{body\_literals(11)} constituting it's body. On
lines 11 and 17 we see that the tuple contains the body literals with
identifiers \texttt{body\_literal(12)} as it's first element, and
\texttt{body\_literal(17)} as it's second element.  A tuple may also be
empty; in this case there will be no facts encoding it's elements in
the reified representation. For example, the rule fact on line 22
which encodes the fact \texttt{b(1)} has the child tuple of body
literals with identifier \texttt{body\_literals(27)}, with no body
literal fact with id 27 occurring in the reification.

The root of the AST is the fact
\texttt{program(0,"base",constants(1),statements(2)).}. Rules in the
gringo language are considered to implicitly be under the
\texttt{\#program base.} subprogram\cite{karoscwa21a}. As such, this
facts encodes the presence of a (sub)program with name "base", with an
empty child tuple of program constants with identifier
\texttt{constants(1)}, and a child tuple of statements with identifier
\texttt{statements(2)}. This tuple of statements encode the sequence
of statements occurring within the subprogram, such as rules,
externals, an other meta-statements in the gringo language
\cite{PotasscoUserGuide19}. In this case, the tuple contains the two
rules, \texttt{rule(3)} and \texttt{rule(21)}, which we have discussed
previously. Note that a program may be comprised of multiple such
subprogram statements, so it would be perhaps more accurate to call
the encoded AST a forest instead of a tree.

The reification contains two types of facts encoding literals,
\texttt{literal} and \texttt{body\_literal}, with the latter only allowed
to occur in the the body of rules. For both, the second argument
indicates it's sign, with "pos" being positive, and "not" being
negative. The third argument identifies the child atom of the literal,
which will in our discussion always be a symbolic atom, but in general
the gringo language has other types of atoms, such as comparisons etc.

Symbolic atom facts always have a single function child identifier as
their second and final argument. These facts serve only to
differentiate function symbols that are an atoms from function symbols
that are function terms in the gringo language.

Function facts encode a function symbol node. The second argument of a
function fact it it's name, encoded as a constant. For example, on
line 6 we see that the name of the function symbol \texttt{c(a(X))} is
c. The third argument of a function fact is a tuple of terms, which
encode the function symbol's arguments. If this tuple is empty, the
function symbol has no arguments, and is thus a constant; this is the
case on line 20 for the constant d. This tuple of terms may in turn
include a further function symbol, indicating nested function symbols,
as is the case for the reification of \texttt{c(a(X))}. Other term
types we see in this reification are variables, as on lines 10 and 16,
and numbers, as on line 27. The second argument of a variable fact is
the variable's name, encoded as a string, while the second argument of
a number fact encodes the value of the number as itself.

Finally, we note that the reification format also contains a set of
facts over predicate \texttt{child/2}. These fact explicitly encode
the child-parent relationship between identifiers, which can be useful
if in some meta-encoding one would like to check reachability between
two nodes, for example. The set of child facts in the reification for
the previous example are the following.

\begin{lstlisting}[language=clingo]
child(program(0),constants(1)).
child(program(0),statements(2)).
  child(statements(2),rule(3)).
    child(rule(3),literal(4)).
      child(literal(4),symbolic_atom(5)).
        child(symbolic_atom(5),function(6)).
          child(function(6),terms(7)).
            child(terms(7),function(8)).
              child(function(8),terms(9)).
                child(terms(9),variable(10)).
    child(rule(3),body_literals(11)).
      child(body_literals(11),body_literal(12)).
        child(body_literal(12),symbolic_atom(13)).
          child(symbolic_atom(13),function(14)).
            child(function(14),terms(15)).
              child(terms(15),variable(16)).
      child(body_literals(11),body_literal(17)).
        child(body_literal(17),symbolic_atom(18)).
          child(symbolic_atom(18),function(19)).
            child(function(19),terms(20)).
  child(statements(2),rule(21)).
    child(rule(21),literal(22)).
      child(literal(22),symbolic_atom(23)).
        child(symbolic_atom(23),function(24)).
          child(function(24),terms(25)).
            child(terms(25),number(26)).
    child(rule(21),body_literals(27)).
\end{lstlisting}

\subsection{Reflection}

As mentioned before, reflection is the inverse operation of
reification, and as such, takes as input a set of reified facts
encoding a valid AST, and outputs it's representation as a gringo
program. For example, if we were to save the reified facts seen
previously as reified-prg.lp, and called the following:

\begin{lstlisting}[language=bash,numbers=none]
renopro reflect reified-prg.lp
\end{lstlisting}

we wound see the original program printed back to us as output.

Reflection is performed by starting from the facts over signature
\texttt{program/4}; as discussed before, these facts serve as the
root(s) of the AST. Then, based on the child identifiers found in
facts, the reachable AST nodes encoded by the set of facts are walked
through recursively to reconstruct the program in it's gringo
representation via the clingo API. It is required that every integer
id is unique per type of reified fact, as otherwise the child facts
cannot be be determined unambiguously.

Note that there is a certain amount of data validation that happens
during reflection, and helpful error messages are printed if the set
of input facts encode an invalid AST. 

As an example if there is a missing child fact where one is required,
we would would get a helpful error message indicating so. This would
be the case for example, if we were to delete the literal fact on line
4 of the reification, as a rule must have a head in the AST. However
deleting some child fact that is optional would result in no error, as
we would still have a valid AST. This would be the case if we remove
for example line 7. The term tuple would then be empty, lines 8-10
would be unreachable, and the reflection of the reified facts would
be:

\begin{lstlisting}[language=clingo]
c :- b(X), not d. 
b(1).
\end{lstlisting}

The facts themselves are also validated against a schema that
describes valid ASTs. For example, were we to replace the fact on line
16 with \texttt{variable(16,2)}, we would receive an error message
telling us that the second argument of a variable fact must be a
string. Similar validation happens for child identifiers that occur as
arguments of facts. If we replace the fact on rule 3 with
\texttt{rule(3,function(4),body\_literals(11)).}, we would receive a
helpful error message indication that \texttt{function(4)} is an
invalid type of identifier for the head of a rule.

\subsection{Transformation}

A transformation meta-encoding is an encoding that takes as input a
set of reified facts encoding a non-ground program. The occurrence of
a set of special predicates of the form \texttt{ast/1} in the stable
model of the encoding indicates some transformation to be applied to
the encoded AST. The types of transformation supported are the following:

\begin{enumerate}
\item \texttt{ast(add(Fact))} indicates that the fact \textt{Fact}
  should be added to the AST.
\item \texttt{ast(delete(Fact))} indicates that the fact \textt{Fact}
  should be deleted from the AST.
\item \texttt{ast(replace(Fact1,Fact2))} indicates that the fact
  \textt{Fact1} should be replaced by \texttt{Fact2} in the AST.
\end{enumerate}

The meaning of these predicates are enforced by the following
program, which is always automatically appended to the transformation
meta-encoding when the transformation functionality is used.

\begin{lstlisting}[language=clingo]
#include "wrap_ast.lp".
#include "replace_id.lp".

transformed(Fact) :- ast((fact(Fact);add(Fact))), 
                 not ast(delete(Fact)).

ast((delete(Fact1);add(Fact2))) :- ast(replace(Fact1,Fact2)).
\end{lstlisting}

The program \verb|wrap_ast.lp| contains schematic rules of the form
$$
\verb|ast(fact(Fact)) :- Fact.|
$$
where \verb|Fact| ranges over all types of facts in the reified
representation. This collects all facts into a single
"namespace". Line 4-5 then realizes the intended meaning of delete and
add. Any reified fact that was already present or was added will be
present in transformed set of reified facts, which are collected under
\verb|transformed/1|; that is, unless they are indicated to be
deleted. Line 7 shows that replacing a fact with another implies that
the first fact is added, and the second one is deleted. Replacement
has a further effect, as is defined by a set of schematic rules in
\verb|replace_id.lp| which we do not show here. In short, apart from
replacing one fact with another, in each reified fact, it also
replaces every identifier referencing the first fact with an
identifier referencing the second fact.

\subsubsection{External generation for meta-telingo}

As the final result of this thesis, we will now present a
transformation meta-encoding that performs the generation of external
statements required to ground a universal temporal program as defined
at the end of Section \ref{sec:non-ground}. The meta-encoding is
broken into two parts, ranging over Listings 3 and 4, which we will
now describe.

In lines 1-5 we define the predicate \verb|operator_arity/2|. The
first argument of this predicate is the name of a (temporal) operator in
it's gringo representation, and the second argument is the operator's
arity.

Next, on lines 7-12 we define when an argument of an operator is safe
or unsafe subformula via predicate \verb|arg/3|, in line with
Definition \ref{def:safe-subformula}. The second argument is the
0-indexed position of the argument, while the third argument indicates
safety via \verb|saf| and \verb|unsafe|.

With the next rule, we assign to each id function id occurring in the
input reified facts it's arity. This is done by counting up the number
of elements of the associated term tuple. With the rule spanning lines
17-19, we define when a function identifier may be considered an operator,
namely, when it's name and arity matches one defined by
\verb|operator_arity/2|.

Now, we define a predicate of key importance, \verb|operand/3|. The
predicate \verb|subformula(function(F),function(F'),Safety)| indicates
that \verb|function(F')| is the subformula of the temporal formula
\verb|function(F)|, with the safety given by \verb|Safety|, which is
determined based on the previously defined predicate
\verb|arg/3|. This predicate will later serve as the binary relation
based on which safe and positive subformulas are defined in
Definitions \ref{def:safe-subformula} and
\ref{def:pos-subformula}. Knowing this, and the fact that subformulas
of a negated formula can neither be safe nor positive, we do not
define \verb|subformula/3| when the operator is \verb|neg|.

On lines 25-28 we assign to symbolic atoms their sign, as given by
their parent literal fact. Then with the following rule we mark
operators that constitute positive symbolic atoms as being root
operators. We again ignore cases where the symbolic atom is negative,
as this will not lead to safe or positive subformulas.

Finishing the first part of the meta-encoding, lines 34-44 define the
predicate \verb|root2desc_subformula(Root,Operand,Safety)|. This
predicate collect operands reachable from a root temporal formula
through \verb|subformula/3|, while keeping track of the safety of the
subformula via \verb|Safety|. In the recursive rules, once we pass
through an unsafe argument, \verb|Safety| is set to unsafe, and
remains so for all subformulas.

Moving on to the second part of the encoding, we define
\texttt{safe\_atomic\_subformua/2}, which holds iff the first argument
is a safe atomic subformula of the second, exactly matching Definition
\ref{def:safe-subformula}. Lines 8-19 defines the predicate
\texttt{root\_tempf2body\_safe\_atomic\_subf/2}, which as the name
suggests, maps root temporal formulas to the safe atomic subformulas
that occur in the body of the given statement. On lines 21-25 we
distinguish root temporal operators which occur in the head vs the
body of a statement.

Having defined all the necessary helper predicates, the remainder of
the meta encoding derives the necessary \texttt{ast(add(Fact))} facts
required to add all the necessary external statement.

On lines 27-28 we add a new base subprogram, under which the external
statements will be gathered. Here we take advantage of the fact that
the ids of reified facts need not necessarily be integers, and may
also be constants of function symbols, and use the id
\texttt{externals} for the id of the new subprogram statement, as well
as it's tuple of statements. This reuse of the id is not problematic,
as ids need only to be unique per type of fact.

The rule spanning lines 27-36 takes care of adding the first type of
external statements, as described at the end of Section
\ref{sec:non-ground}. The rule make

\begin{center}
\begin{minipage}{\linewidth}
  \begin{lstlisting}[]
operator_arity((initial;final),0).
operator_arity(
  (prev;weak_prev;next;weak_next;neg;always_after;always_before;
   eventually_after;eventually_before),1).
operator_arity((until;since;release;trigger;and;or),2).

arg((weak_prev;weak_next;neg)),0,unsafe).
arg((until;since;release;trigger)),0,unsafe).
arg(or,(0;1),unsafe).
arg(Opname,Arg,safe)
  :- operator_arity(Opname,Arity), Arg=0..Arity-1, 
     not arg(Opname,Arg,unsafe).

func_arity(F,A) :- function(F,_,terms(TS)), 
                   A = #count{ P: terms(TS,P,_) }.

temporal_formula(function(F))
  :- function(F,Name,terms(T)), operator_arity(Name,Arity), 
     func_arity(F,Arity).

subformula(function(F),function(F'),Safety)
  :- temporal_formula(function(F)), function(F,Name,terms(T)), 
     Name!=neg, arg(Name,P,Safety), terms(T,P,function(F')).

atom_sign(A,Sign)
  :- literal(_,Sign,symbolic_atom(A)), symbolic_atom(A,_).
atom_sign(A,Sign)
  :- body_literal(_,Sign,symbolic_atom(A)), symbolic_atom(A,_).

root_temporal_formula(Func)
  :- temporal_formula(Func), symbolic_atom(A,Func), 
     atom_sign(A,"pos").

root2desc_subformula(Root,Subformula,Safety)
  :- root_temporal_formula(Root), subformula(Root,Subformula,Safety).
root2desc_subformula(Root,Subformula',Safety)
  :- root2desc_subformula(Root,Subformula,Safety), 
     subformula(Subformula,Subformula',safe).
root2desc_subformula(Root,Subformula',unsafe)
  :- root2desc_subformula(Root,Subformula,Safety), 
     subformula(Subformula,Subformula',unsafe).
root2desc_atomic_subformula(Root,Subformula,Safety)
  :- root2desc_subformula(Root,Subformula,Safety), 
     not temporal_formula(Subformula).

\end{lstlisting}
\captionof{lstlisting}{generate-external.lp - part 1}
\end{minipage}
\end{center}


\begin{center}
\begin{minipage}{\linewidth}
  \begin{lstlisting}[]
safe_atomic_subformula(Func,Func) 
  :- symbolic_atom(A,Func), not root_temporal_formula(Func), 
     atom_sign(A,"pos").
safe_atomic_subformula(Func,Func')
  :- symbolic_atom(A,Func), root_temporal_formula(Func),
     root2desc_atomic_subformula(Func,Func',safe).

stm2desc((statements(STM),Pos),Child) 
  :- statements(STM,Pos,Child).
stm2desc(Statement,Child') 
  :- stm2desc(Statement,Child), 
     not root_temporal_formula(Child), child(Child,Child').

root_tempf2body_safe_atomic_subf(Func,Func'')
  :- stm2desc(Statement,Func), root_temporal_formula(Func), 
     stm2desc(Statement,body_literal(BL)), 
     body_literal(BL,"pos",symbolic_atom(S)), 
     symbolic_atom(S,Func'), 
     safe_atomic_subformula(Func',Func'').

head_root_temp_form(Func)
  :-  literal(L,"pos",symbolic_atom(S)), symbolic_atom(S,Func),
      root_temporal_formula(Func).
body_root_temp_form(Func) 
  :- not head_root_temp_form(Func), root_temporal_formula(Func).

ast(add(program(externals,"base",constants(externals),
                statements(externals)))).

ast(add(statements(externals,pos(Func,Subformula),
                   external(Subformula));
        external(Subformula,symbolic_atom(Subformula),
                 body_literals(Func),false);
        symbolic_atom(Subformula,Subformula)))
  :- head_root_temp_form(Func), 
     root2desc_atomic_subformula(Func,Subformula,_).

ast(add(statements(externals,pos(Func),external(Func));
        external(Func,symbolic_atom(Func),
                 body_literals(Func),false);
	symbolic_atom(Func,Func)))
  :- body_root_temp_form(Func).

ast(add(body_literals(Func,pos(Func'),body_literal(Func'));
        body_literal(Func',"pos",symbolic_atom(Func'));
        symbolic_atom(Func',Func')))
  :- root_temporal_formula(Func), 
     root_tempf2body_safe_atomic_subf(Func,Func'), 
     body_root_temp_form(Func).
\end{lstlisting}
\captionof{lstlisting}{generate-external.lp - part 2}
\end{minipage}
\end{center}
