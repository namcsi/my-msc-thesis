\subsection{QTHT}

\subsubsection{Foundations}

Our introduction to QTHT will follow that of \cite{agcapevidi17a}. We
define a first-order signature as a tuple $(C,P)$, where $C$ is a set
of constants, and $P$ is a set of predicate symbols, each having an
associated arity. The terms of $(C,P)$ are then the set
$T = C \cup X$, where $X$ is a set of variables.

The atoms over signature $(C,P)$ are constructed from $T$, and have
the form $p(t_1,\dots,t_n)$, where $t_i \in T$ for
$\rangecc{i}{1}{n}$. We assume that the equality predicate is always
an element of $P$, and denote it with infix notation as $t_1=t_2$. A
first order temporal formula over $(C,P)$ is defined by the same
grammar as in the temporal formulas discussed in Subsection
\ref{subsec:tht-tel}, except that it uses atoms such as
$p(t_1, \dots, t_n)$ in place of propositional atoms, and has the
additional production rules:

$$
\varphi ::= \forall x \varphi(x) \mid \exists x \varphi(x)
$$

We denote the set of all such formulas over signature $(C,P)$ as the
language $\mathcal{L}_{(C,P)}$, which we sometimes abbreviate as
$\mathcal{L}$ when this does not lead to confusion. We say that a term
or formula is ground if it contains no variables. By $\A(C,P)$ we
denote the set of ground atoms over language $(C,P)$.

Let $D$ be a non-empty set, called a \emph{domain}. An an
interpretation $\sigma$ of $C$ and $D$ is defined as a function
$$
\sigma: C \cup D \rightarrow D
$$
such that $\sigma(d) = d$ for all $d \in D$. We call a mapping from
the set of variables $X$ to $D$ a variable assignment over domain
$D$. If $\varphi$ has free variables, let $\varphi^\mu$ be the formula
obtained by replacing every free variable $x$ by $\mu(x)$.

\begin{definition}[TQHT-Structure \cite{agcapevidi17a}]
  A temporal here-and-there $(D,P)$ structure with static domains, or
  TQHT-structure for short, is defined as a tuple $\qthandt$, where
  for any $\kinlambda$ $H_k,T_k \subseteq \A(D,P)$, and
  $H_k \subseteq T_k$.

  We call a TQHT structure total iff $\bm{H}=\bm{T}$
\end{definition}


The satisfaction relation for TQHT structures is almost completely the
same as that the of THT satisfaction relation defined in section 2, if
one substitutes $\bm{M}=\qthandt$. They differ in the satisfaction
relation for atoms, and additionally TQHT satisfaction is also defined
for for first order quantifiers \cite{agcapevidi17a}. These are
formalized below.

\begin{itemize}
  \item if $\varphi$ \in $\A(C \cup D,P)$, then:
    \begin{itemize}
    \item if $\varphi$ is $p(t_1,\dots,t_n)$, then $\bm{M},k \models \varphi$ iff $p(\sigma(t_1), \dots, \sigma(t_n))\in H_k$
    \item if $\varphi$ is $t = s$, then  $\bm{M},k \models t = s$ iff $\sigma(t) = \sigma(s)$
    \end{itemize}
  \item $\qthandt \models \forall x \phi(x)$ iff $\langle \dsigma, w,\bm{T} \rangle \models \phi(d)$ for all $d \in D$ and for all $w \in \{\bm{H}, \bm{T}\}$ 
  \item $\bm{M} \models \exists x \phi(x)$ iff $\bm{M} \models \varphi(d)$ for some $d \in D$
\end{itemize}

The logic induced by the tautologies as per the satisfaction relation
above is called quantified temporal Here-and-There logic with static
domain, or QTHT for short.

Persistence can be proven to hold in TQHT as well.

\begin{proposition}[QTHT Persistence \cite{agcapevidi17a}]
  For any formula $\varphi$, if $\qthandt \models \varphi$, then
  $\qttandt \models \varphi$.
\end{proposition}

First order temporal equilibrium models can be defined in the usual way.

\begin{definition}[First Order Temporal Equilibrium Model/Stable Model \cite{agcapevidi17a}]
  A total TQHT structure $\qttandt$ is a (first order) temporal
  equilibrium model of a theory
  $\Gamma \subset \mathcal{L}_{(C,P)}$ iff
  $\qttandt \models \Gamma$ and there is no $\bm{H} < \bm{T}$ such
  that $\qthandt \models \Gamma$.
\end{definition}

We define a first order variant of a temporal rule and temporal program.

\begin{definition}[Universal Temporal Rule/Temporal Program] We call
  formulas of the form $\forall x_1, \dots, \forall x_n \varphi$,
  where $\varphi$ is a temporal rule, a \emph{universal temporal rule}. We
  call a set of such formulas a \emph{universal temporal program}.
\end{definition}

We should note that in the logic programming context, we usually
assume that we have a so-called Herbrand structure \cite{peaval06a},
in which we have $\dsigma = \langle C, id \rangle$, where $id$ is the
identity function, i.e. that the constants of the input program
constitute the domain, and that each constant is mapped to itself. We
do not make this assumption where not necessary in the following, and
state the results in their general form.

\subsubsection{Safety}

The usual non-temporal definition of safety requires that each
variable occurring in a rule also occur in a positive literal of the
body of the rule \cite{gekakasc12a}. This safety definition is
extended to a syntactic fragment of temporal formulas in
\cite{agcapevidi17a} by allowing atoms in the scope of a singe $\Next$
to provide safety, and proved that this extended definition satisfied
two key properties that safety guarantees in the non-temporal setting;
namely, the extended definition ensures \emph{domain independence},
and ensures that there are no \emph{unnamed individuals} in the answer
sets of a program \cite{capeva09a} \cite{capeva09b}. Domain
independence states that the answer sets of the program do not change
if they are ground using a superset of the program's constants. This
is important, as otherwise, adding a new rule that one would expect to
have no affect on the answer sets of the program, can still change the
answer sets simply by the virtue of introducing a new constant symbol.

To formalize the second property, given an interpretation $\sigma$, we
call an element of the domain $d \in D$ an \emph{unnamed individual},
iff $d \not\in Im(\sigma\vert_{C})$, i.e. if no constant is mapped to
it by $\sigma$. Now, given a trace $\bm{T}$, we define the restriction
of the trace to the images of the constants in $C$ as
$\bm{T}\vert _C$; that is
$\bm{T}\vert_{Ci} \defeq T_i \cap At(\sigma(C),P)$. A answer set
$\bm{T}$ is then said to contain no unnamed individuals iff
$\bm{T} \vert_{C} = \bm{T}$. This property is desirable, as one would
usually not want arbitrary elements of the domain to appear in the
solutions to a theory. Note that in the logic programming setting this
property is guaranteed as we restrict the models to be Herbrand
structures.

In what follows, we will relax the safety condition even further than
in \cite{agcapevidi17a}, and prove that these desirable properties
still hold, by retracing the steps of the proof the aforementioned
paper. First, we define this extended notion of safety:

\begin{definition}[Immediate Subformula]
  For a formula $\varphi \in \mathcal{L}_{(C,P)}$, we say
  that $\psi$ is an immediate subformula of $\varphi$, iff
  $\varphi = \odot \psi$ or $\varphi = \gamma \otimes \psi$ or
  $\varphi = \psi \otimes \gamma$, where
  $\psi, \gamma \in \mathcal{L}_{(C,P)}$ and
  $\odot/\otimes$ is any unary/binary temporal operator or logical
  connective.
\end{definition}

\begin{definition}[Safety]
  Given an implication-free temporal formula $\varphi$, we say a
  subformula $\gamma \in subf(\varphi)$ is in an unsafe position in
  $\varphi$, if it is the immediate subformula of a formula
  $\psi \in subf(\varphi)$, where $\psi$ is of the following form:
  $\psi = \neg \gamma \mid \gamma \vee \xi \mid \xi \vee \gamma \mid
  \gamma \since \xi \mid \gamma \until \xi \mid \gamma \trigger \xi
  \mid \gamma \release \xi$. For any such $\psi$, we furthermore
  consider the subformulas in $subf(\psi) \cap subf(\varphi)$ to occur
  in an unsafe position in $\varphi$. Any subformula of $\varphi$ that
  is not in an unsafe position in $\varphi$ is said to be in a
  \textit{safe position} in $\varphi$. We denote the set of such
  subformulas as $\safe{\varphi}$.

  We call a temporal rule $\varphi = \alwaysF(B \rightarrow H)$
  \emph{safe}, iff each variable $x$ in $\varphi$ occurs in an atom
  that is in a safe position in $B$. We call a universal temporal rule
  $\forallx \psi$ safe iff $\psi$ is safe. We call a program safe, if
  all of it's rules are safe.
\end{definition}

As an example, under this expanded definition of safety, the following
rule would also be considered safe:

\begin{center}
    \begin{lstlisting}[numbers=none]
should_not_shoot(X)  :- since(unloaded(X),prev(shoot(X))).
    \end{lstlisting}
\end{center}

The rule is safe, as \verb|shoot(X)| is an atom which is in a safe
position w.r.t. of the body of the rule, and can thus provide safety
for the single implicitly universally quantified variable \verb|X|
occurring in the rule. Note that \verb|unloaded(X)| is in an unsafe
position of the body, and as such would not be able to provide safety
for the variable \verb|X|.

We will now embark on the path towards proving the two desirable
properties of the new safety definition. We first prove some
properties of implication-free temporal formulas.

\begin{lemma}\label{lemma:safe-atomic-subformula-satisfied}
  Let $\varphi$ be an implication-free ground temporal formula. If
  $\qthandt,j \models \varphi$, then for all atoms $p$ with
  $p \in \safe{\varphi}$, there exists a $\kinlambda$, such that
  $\qthandt,k \models p$.
\end{lemma}
\begin{proof}
  In the following, let $M=\qttandt$. The property can be shown to
  hold by structural induction on $\varphi$. Clearly, the property
  holds form atoms, $\top$ and $\bot$.

  To make the inductive step, let us first consider logical
  connectives. If $\varphi = \psi \wedge \xi$, and
  $M,j \models \varphi$, then $M,j \models \psi$ and
  $M,j \models \xi$. Then applying the induction hypothesis, for any
  atoms $p \in \safe{\psi}$ and $q \in \safe{\xi}$ we must have some
  $k,l \in \intervcc{0}{\lambda}$ such that $M,k \models p$ and
  $M,l \models q$. We can make the inductive step by noting that
  $\safe{\psi \wedge \xi} = \{ \psi \wedge \xi \} \cup \safe{\psi}
  \cup \safe{\xi}.$ In cases when $\varphi = \psi \vee \xi$ or
  $\varphi = \neg \psi$, the property holds trivially, as
  $\safe{\varphi} = \{ \varphi \}$, and $\varphi$ is not an atom.

  We will now consider past operators; the same argument can be
  applied to the dual future versions of these operators. If
  $M,j \models \varphi = \previous \psi$, then $M,j-1 \models \psi$,
  so we can make the inductive step by applying the induction
  hypothesis and noting that
  $\safe{\previous \psi} = \{ \previous \psi \} \cup \safe{\psi}$. For
  the cases where $\varphi = \psi \since \xi$ resp.
  $\varphi = \psi \trigger \xi$, by inspecting the satisfaction relation for
  respective the operators, one can conclude that if
  $M,j \models \varphi$, then it must be the case that
  $M,k \models \xi$ for some $\kinlambda$. Then, applying the
  induction hypothesis, and noting that
  $\safe{\varphi} = \safe{\xi} \cup \{ \psi \since \xi \}$
  resp. $\safe{\varphi} = \safe{\xi} \cup \{ \psi \trigger \xi \}$
  makes the inductive step, and concludes the proof.
  
\end{proof}

\begin{lemma}\label{lemma:t-not-h-atomic-subformula}
  Let $\varphi$ be an implication-free ground temporal formula. Then,
  if $\qttandt \models \varphi$ and $\qthandt \not\models \varphi$, it
  follow that there is a $\kinlambda$ and an atomic subformula
  $p \in subf(\varphi)$, such that $\qttandt \models p$ and
  $\qthandt \not\models p$.
\end{lemma}
\begin{proof}
  The property can easily be shown by structural induction on
  $\varphi$. The only step of note is that of $\varphi = \neg \psi$;
  in this case if $\qttandt \models \neg \psi$, then by
  persistence we must have $\qthandt \models \neg \psi$, so the
  property holds trivially.
\end{proof}



\begin{lemma}
  Let $\varphi=\alwaysF (B \rightarrow H)=\alwaysF \psi$ be a temporal
  rule, and $\mu$ a variable assignment in $\dsigma$. If $\varphi$ is
  safe, then $\qttandt,k \models \varphi^\mu$ implies
  $\qttcandt,k \models \varphi^\mu$
\end{lemma}
\begin{proof}
  In the following, let $M=\qttandt$, and $M_C=\qttcandt$.

  Suppose indirectly that $M,k \models \varphi^\mu$, but
  $M_C,k \not \models \varphi^\mu$. Then, there must be some
  $k\leq i$, such that $M,i \models \psi^\mu$ and
  $M_C,i \not \models \psi^\mu$, which must mean that
  $M_C,i \models B^\mu$ and $M_C,i \not \models H^\mu$. By
  persistence, we then have $M,i \models B^\mu$, which means we must
  also have $M,i \models H^\mu$, as we know that
  $M,k \models \alwaysF(B^\mu \rightarrow H^\mu)$. Now, we have found
  that $M,i \models H^\mu$ and $M_C,i \not \models H^\mu$. By applying
  Lemma \ref{lemma:t-not-h-atomic-subformula}, we derive that there is some atomic subformula
  $q$ occurring in $H$, and a $\rangecc{j}{0}{\lambda}$, such that
  $M,j \models q^\mu$ and $M_C,j \not \models q^\mu$. Therefore, we
  must have a variable $x$ in $q$, with $\mu(x) \not \in
  \sigma(C)$. As $x$ is safe, there must be an occurrence of $x$ in an
  atomic formula $p$ in a safe position of $B$. Furthermore, as
  $\mu(x) \not \in \sigma(C)$, we also have $M_C,l \not \models p^\mu$
  for any $\rangecc{l}{0}{\lambda}$. Having determined this fact, we
  can now apply the contraposition of Lemma \ref{lemma:safe-atomic-subformula-satisfied} to derive
  that $M_C,i \not \models B^\mu$, a contradiction that concludes the
  proof.
\end{proof}

From here, we can follow the proof of Proposition 5 and Theorem 2 from \cite{agcapevidi17a} more
or less verbatim to obtain the following results.

\begin{proposition}
  For any safe universal temporal rule $\varphi = \forallx \psi$
  \begin{equation*} \qttandt \models \varphi \text{ iff } \qttcandt \models \varphi. \end{equation*}
\end{proposition}

\begin{theorem}[No Unnamed Individuals]
  If $\varphi$ is a safe universal temporal rule, and $\qttandt$ is a temporal equilibrium
  model of $\varphi$, then $\bm{T}\vert_{C}=\bm{T}$.
\end{theorem}

\subsubsection{Grounding}

Let $D'\subseteq D$ be a finite subset of the domain. The
grounding over $D'$ of a sentence $\varphi$ is denoted as
$\ground{D'}{\varphi}$. Grounding is defined to assign each atom to
itself, expand universal/existential quantification
conjunctively/disjunctively over each possible replacement of the
quantified variable by a value $d \in D^\prime$, and to be operation
preserving w.r.t. all binary and unary logical and temporal
connectives. \cite{agcapevidi17a}. Formally
$\ground{D^{\prime}}{\varphi}$ is defined inductively as follows:

\begin{align*}
  \ground{D^{\prime}}{p} & \defeq p, \text { where } p \text { is an atom } \\
  \ground{D^{\prime}}{\forall x \varphi(x)} & \defeq \bigwedge_{d \in D^{\prime}} \ground{D^{\prime}}{\varphi(d)} \\
  \ground{D^{\prime}}{\exists x \varphi(x)} & \defeq \bigvee_{d \in D^{\prime}} \ground{D^{\prime}}{\varphi(d)} \\
  \ground{D^{\prime}}{\varphi_1 \otimes \varphi_2} & \defeq \ground{D^{\prime}}{\varphi_1} \otimes \ground{D^{\prime}}{\varphi_2}, \\
  \ground{D^{\prime}}{\odot \varphi} & \defeq \odot \ground{D^{\prime}}{\varphi} \\
\end{align*}

for any
$\otimes \in \{ \land, \lor, \rightarrow, \since, \trigger, \until,
\release \}$ and $\odot \in \{ \Next, \previous \}$. The grounding
over a set of constants $C' \subseteq C$ is defined in a similar way.

To show that our definition of safety guarantees domain independence
with respect to the formal definition of grounding above, we first
need to prove the following lemma.

\begin{lemma}\label{lemma:not-imsigma-domain-sat}
  Let $\varphi(x)$ be a safe temporal rule, and $\qthandt$ be a
  TQHT structure such that $\bm{T}\vert_{C}=\bm{T}$. Then, for any
  $d \in D \setminus \sigma(C)$, we have

  \begin{equation*}
    \qthandt \models \varphi(d)
  \end{equation*}
  
\end{lemma}

\begin{proof}
  Given a safe temporal rule
  $\varphi(x) = \alwaysF(B(x) \rightarrow H(x))$, by safety, there
  exists some atom $p(x)$ in a safe position of $B(x)$. Now let us
  take a TQHT structure $\qthandt$ such that $\bm{T}\vert_{C}=\bm{T}$
  and a domain element $d \in D \setminus \sigma(C)$. Since
  $\bm{T}\vert_{C}=\bm{T}$, we know that for any $\kinlambda$,
  $\qthandt, k \not \models p(d)$, and therefore by Lemma
  \ref{lemma:safe-atomic-subformula-satisfied} for any
  $\rangecc{j}{0}{\lambda}$ we have $\qthandt,j \not \models B(d)$,
  and as such
  $\qthandt \models \alwaysF(B(d) \rightarrow H(d)) = \varphi(d)$,
  concluding the proof.
\end{proof}

By building upon Lemma \ref{lemma:not-imsigma-domain-sat} and Lemma
\ref{lemma:safe-atomic-subformula-satisfied} in the same way as in
\cite{agcapevidi17a}, one can obtain the following results, the final
one showing that our definition of safety guarantees domain independence.

\begin{proposition}\label{prop:sat-iff-sat-groundc}
  Let $\varphi = \forallx \psi$ be a safe universal temporal rule and \\
  $\mathcal{M} = \qthandt$ a QTHT structure such that
  $\bm{T}\vert_{C}=\bm{T}$. Then, the following holds:

  \begin{equation*}
    \mathcal{M} \models \varphi \text{ iff } \mathcal{M} \models \ground{C}{\varphi}
  \end{equation*}
\end{proposition}

\begin{proposition}
  Let $\varphi$ be a safe universal temporal rule and
  $\mathcal{M} = \qthandt$ a QTHT structure. Then, $\mathcal{M}$ is a
  first order temporal equilibrium model of $\varphi$ iff
  $\mathcal{M}$ is a first order temporal equilibrium model of
  $\ground{C}{\varphi}$.
\end{proposition}

\begin{theorem}[Domain Independence]\label{theorem:domain-independence}
  Let $\varphi$ be a safe universal temporal rule. Suppose we expand
  the language $\mathcal{L}$ by considering an expanded set of
  constants $C \subseteq C'$. A total QTHT model $\qttandt$ is a
  temporal equilibrium model of $\ground{C}{\varphi}$ iff it is a
  temporal equilibrium model of $\ground{C'}{\varphi}$.
\end{theorem}

In practice, obtaining a ground instantiation of a universal temporal
program in the symbolic input language of our meta-telingo system
using gringo is, as demonstrated at the end of the previous section,
not as straightforward as one might hope. This is due to the fact that
the simplifications performed by the grounding procedure of gringo are
not equivalent under the semantics of QTHT. We will therefore come up
with an alternative grounding procedure. Our method will work by
constructing a much simpler positive normal logic program based on the
original program, and retrieving the ground instances of our original
program from the grounding of this simpler program.

% Given a temporal trace $\bm{T}$, we define the set
% $\text{Facts}(T) \subseteq \A(D,P)$ as
% $\text{Facts}(T) \defeq \{ p \in T_k \mid \kinlambda \}$. Now, given a
% universal temporal logic program $\Gamma$, we define the set of
% derivable facts as
% $\text{Derivable}(\Gamma) \defeq \bigcup_{\qttandt} \in TSM(\Gamma)
% \text{Facts}(T)$. In plain language, the derivable facts of a theory
% is the set of atoms for which there is some temporal stable model in
% which the atom holds at some time point.

% It is easy to see that, if we can somehow obtain
% $\text{Derivable}(\Gamma)$, or a superset, then we use it ground
% $\Gamma$ by substituting

Formally, given a universal temporal rule
$r=\forall x_1 \dots x_n \alwaysF(B \rightarrow H)$, let
$p_1,\dots, p_n$ be the atomic subformulas occurring in a safe
position in $B$, and $h_1, \dots h_m$ the atomic subformulas occurring
in H. The transformation of $r$, denoted as $\delta(r)$ is defined as:

\begin{equation*} \delta(r) \defeq \{ \forall x_1, \dots \forall x_n
\alwaysF(\bigwedge_{j=1}^{n}p_j \rightarrow h_i) \mid
\rangecc{i}{1}{m})\}
\end{equation*}
and $\delta (\Gamma^{\prime}) \defeq \cup_{r \in Gamma}
\delta(r)$. Then, the following result holds:

\begin{proposition}\label{prop:simplified-facts}
  Let $\qttandt$ be a temporal equilibrium model of a safe universal
  temporal program $\Gamma$, and
  $\langle \dsigma, \bm{J},\bm{J} \rangle$ the unique temporal
  equilibrium model of $\delta(\Gamma)$. Then,
  $\bm{T} \subseteq \bm{J}$.
\end{proposition}
\begin{proof}
  Let us abbreviate $\qttandt$,
  $\langle \dsigma, \bm{J},\bm{J} \rangle$ and $\qttjandt$ as \ttandt,
  \tjandj and \ttjandt, respectively. We will show that
  $\ttjandt \models \Gamma$, from which it will follow that
  $\bm{T} \cap \bm{J} = \bm{T}$ from the fact that $\ttandt$ is in
  equilibrium.

  Since $\Gamma$ is safe, by Proposition
  \ref{prop:sat-iff-sat-groundc} it also suffices to show that
  $\langle T \cap J, T \rangle \models \ground{C}{\Gamma}$. Let us
  take a grounding of an arbitrary rule from $\Gamma$,
  $r = \tempruleshort$. To prove that $\ttjandt \models r$, we first
  note that, since we already know that $\ttandt \models r$, it
  suffices to show that, for any time point $\kinlambda$, if
  $\ttjandt_k \models B$, then $\ttjandt_k \models H$. Let us assume
  that $\ttjandt_k \models B$. Then, by persistence
  $\ttandt_k \models B$, which means we must also have
  $\ttandt_k \models H$, as $\ttandt$ is an equilibrium model of
  $\tempruleshort$.

  Now, seeing as we have $\ttjandt_k \models B$, by Lemma
  \ref{lemma:safe-atomic-subformula-satisfied} for all safe atomic
  subformulas $p \in subf(B)$, there must be some
  $\rangeco{j}{0}{\lambda}$ such that $\ttjandt_j \models p$, which
  implies that $\tjandj_j \models p$. Due to the form of the rules in
  $\delta(\Gamma)$, this also implies that $\tjandj_i \models h$ for
  any atomic subformula $h \in subf(H)$ and time point
  $\rangeco{i}{0}{\lambda}$.

  Having established this, let us suppose indirectly that
  $\ttjandt_k \not\models H$. As $\ttandt_k \models H$, by Lemma
  \ref{lemma:t-not-h-atomic-subformula} we must have some atomic
  subformula $h \in subf(H)$ and some $\rangeco{j}{0}{\lambda}$ such
  that $\ttandt_j \models h$ and $\ttjandt_j \not\models h$. But this
  cannot be the case, as we have shown that $\tjandj_i \models h$, for
  any time point $\rangeco{i}{0}{\lambda}$, so in particular
  $\tjandj_j \models h$ a contradiction which concludes the proof.
\end{proof}

Proposition \ref{prop:simplified-facts} can be used to obtain a ground
instantiation of $\Gamma$ given $\delta(\Gamma)$ in the following
way. 

First we obtain the unique temporal equilibrium model $\bm{J}$ of
$\delta(\Gamma)$, and collect all ground atoms occurring at any time
point in the trace $\bm{J}$. Note that due to the simple structure of
$\delta(\Gamma)$, the set of ground atoms are identical at any time
point in the trace $\bm{J}$, as we are essentially just solving the same
positive logic program at each time point. As positive logic programs
can be fully evaluated during grounding by gringo, one does not even
require a temporal answer set solver for this step; one can simply
pass the program $\delta^{\prime}(\Gamma)$ consisting of rules

\begin{equation*} \delta^{\prime}(r) =\{ \forall x_1, \dots \forall x_n
  \bigwedge_{j=1}^{n}p_j \rightarrow h_i) \mid \rangecc{i}{1}{m})\}
\end{equation*}

to gringo to obtain the set of ground atoms.

Once this set of ground atoms have been obtained, they can then be
utilized to ground the program by substituting for atomic formulas
with variables, while taking care to perform a join operation when a
variable occurs in multiple atomic formulas, like in e.g.:
\verb|:- next(a(Y,X)), b(X,Z).|. This grounding will be correct, as due to
Proposition \ref{prop:simplified-facts}, any atom that may occur in
any temporal stable model of the original program will occur in
$\bm{J}$.

Instead of performing these substitutions and join operations
ourselves, we can make gringo perform these tasks for us during the
grounding of the input program, written in the input language of
meta-telingo. We can achieve this by cleverly enhancing our original
program $\Gamma$ with external statements similar to the form of those
of $\delta^\prime(\Gamma)$, which will ensure that all the ground
atoms of $J$ of the unique stable model of $\delta^\prime(\Gamma)$ are
substituted into the program during grounding. We will not make a
formal proof of the correctness of this method, as it involves the
grounding algorithm of gringo which is out of scope for this thesis,
but we will give an informal description of why the approach works.

To roughly summarize how external statements work in gringo, any
\verb|a| marked as external via a \verb|#external a.| statement will
mark \verb|a| to not be removed from bodies of statements and
conditions, even if they do not appear in the head of any rule
\cite{PotasscoUserGuide19}. External statements may also have bodies,
in which case the external statement is ground like a rule, and the
obtained ground heads are marked as external, while the ground body is
discarded. As an example, the program
\begin{lstlisting}[language=clingo]
#external b(1).
#external a(X) : b(X).
\end{lstlisting}
would be ground to the two external statements \verb|#external b(1).|
and \\\verb|#external a(1)|.

Now to describe the external generation. First, we will generate one
external statement for each rule in $r' \in \delta^\prime(\Gamma)$,
with the head and bodies of the external statement being the same as
$r'$. These first set of external statements will, in effect cause the
grounder to derive unique stable model $J$ of $\delta^\prime(\Gamma)$,
but in the form of ground external statements, thus marking each atom
in $J$ to not be simplified during grounding.

Second, for each rule $r \in \Gamma$ we will generate one external
statement for each non-atomic formula $\gamma$ in the body of $r$, the
head of which will be $\gamma$, and the body of which will be the
conjunction of all safe atomic subformulas occurring in the body of
$r$. This second set of rules will ensure that the protected atoms in
$J$ are substituted in their correct place in atomic subformulas.

As an example, take the following program:



We carry out the generation of these external statements using a novel
non-ground meta-programming system for AST which allows us to specify
transformations over non-ground. This system is described, as well as
the meta-encoding that performs this transformation, in the following
subsection.
