\subsection{Quantified Temporal Here-and-There}

Our introduction to Quantified Temporal Here-and-There (QTHT) will
follow that of \cite{agcapevidi17a}. We define a (function-free)
first-order signature as a tuple $(C,P)$, where $C$ is a set of
constants, and $P$ is a set of predicate symbols, each having an
associated arity. The terms of $(C,P)$ are then the set
$T = C \cup X$, where $X$ is a set of variables.

The atoms over signature $(C,P)$ are constructed from $T$ and $P$, and
have the form $p(t_1,\dots,t_n)$, where $t_i \in T$ for
$\rangecc{i}{1}{n}$. We assume that the equality predicate is always
an element of $P$, and denote it with infix notation as $t_1=t_2$. A
first-order temporal formula over $(C,P)$ is defined by the same
grammar as in the temporal formulas discussed in Subsection
\ref{subsec:tht-tel}, except that it uses atoms such as
$p(t_1, \dots, t_n)$ in place of propositional atoms, and has the
additional production rules:

$$
\varphi ::= \forall x \varphi(x) \mid \exists x \varphi(x)
$$

We denote the set of all such formulas over signature $(C,P)$ as the
language $\langcpt$. We say that a term or formula is ground if it
contains no variables. By $\A(C,P)$ we denote the set of ground atoms
over language $(C,P)$.

Let $D$ be a non-empty set, called a \emph{domain}. An an
interpretation $\sigma$ of $C$ and $D$ is defined as a function
$$
\sigma: C \cup D \rightarrow D
$$
such that $\sigma(d) = d$ for all $d \in D$. We call a mapping from
the set of variables $X$ to $D$ a \textit{variable assignment} over
domain $D$. If $\varphi$ has free variables, let $\varphi^\mu$ be the
formula obtained by replacing every free variable $x$ by $\mu(x)$.

\begin{definition}[TQHT-Structure \cite{agcapevidi17a}]
  A temporal here-and-there $(D,P)$ structure with static domains, or
  TQHT-structure for short, is defined as a tuple $\qthandt$, where
  for any $\kinlambda$ $H_k,T_k \subseteq \A(D,P)$, and
  $H_k \subseteq T_k$.

  We call a TQHT structure total iff $\bm{H}=\bm{T}$
\end{definition}


The satisfaction relation for TQHT structures is almost completely the
same as that the of THT satisfaction relation defined in section 2, if
one substitutes $\bm{M}=\qthandt$ considers formulas from \langcdpt
instead of \langat. They differ in the satisfaction relation for
atoms, and additionally TQHT satisfaction is also defined for first
order quantifiers, as formalized below.

\begin{itemize}
  \item if $\varphi$ \in $\A(C \cup D,P)$, then:
    \begin{itemize}
    \item if $\varphi$ is $p(t_1,\dots,t_n)$, then $\bm{M},k \models \varphi$ iff $p(\sigma(t_1), \dots, \sigma(t_n))\in H_k$
    \item if $\varphi$ is $t = s$, then  $\bm{M},k \models t = s$ iff $\sigma(t) = \sigma(s)$
    \end{itemize}
  \item $\qthandt,k \models \forall x \phi(x)$ iff $\langle \dsigma, w,\bm{T} \rangle,k \models \phi(d)$ for all $d \in D$ and for all $w \in \{\bm{H}, \bm{T}\}$ 
  \item $\bm{M},k \models \exists x \phi(x)$ iff $\bm{M},k \models \varphi(d)$ for some $d \in D$
  \end{itemize}
  We write $\qthandt \models$
Note that though our input language is $\langcpt$, the satisfaction
relation defined above for a TQHT structure $\qthandt$ is defined for
formulas in $\langcdpt$, as to check the satisfaction of universal and
existential quantifiers we must substitute values from the domain in
place of the quantified variable.

The logic induced by the tautologies as per the satisfaction relation
above is called quantified temporal Here-and-There logic with static
domain, or QTHT for short.

Persistence can be proven to hold in QTHT.

\begin{proposition}[QTHT Persistence \cite{agcapevidi17a}]
  For any formula $\varphi \in \langcpt$, if $\qthandt \models \varphi$, then
  $\qttandt \models \varphi$.
\end{proposition}

First-order temporal equilibrium models and stable models can be defined in the usual way.

\begin{definition}[First-Order Temporal Equilibrium Model/Stable Model \cite{agcapevidi17a}]
  A total TQHT structure $\qttandt$ is a (first-order) temporal
  equilibrium model of a theory $\Gamma \subset \langcpt$ iff
  $\qttandt \models \Gamma$ and there is no $\bm{H} < \bm{T}$ such
  that $\qthandt \models \Gamma$. A trace
  $\lange \dsigma, \bm{T} \rangle$ is a (first-order) temporal stable
  model of a temporal theory $\Gamma$ iff $\qttandt$ is a temporal
  equilibrium model of $\Gamma$.
\end{definition}

We define a first-order variant of a temporal rule and temporal program.

\begin{definition}[First-Order Temporal Rule/Universal Temporal
  Rule/\quad Universal Temporal Program]
  We call a formula $\varphi \in \langcpt$ \emph{implication and
    quantifier-free} if it contains no quantifiers and no implications
  (apart from negation). We denote the set of such formulas as
  $\langfcpt$. We call a formula of the form $\varphi=\tempruleshort$,
  where $B, H \in \langfcpt$, a \emph{first-order temporal rule}. We
  call formulas of the form $\forallx \varphi$, where $\varphi$ is a
  first-order temporal rule with no variables aside from
  $x_1, x_2, \dots, x_n$, a \emph{universal temporal rule}. We call a
  set of such formulas a \emph{universal temporal program}.
\end{definition}

We should note that in the logic programming context, we usually
restrict the set of QTHT structures under consideration to so-called
Herbrand structures \cite{peaval06a}. Given a universal (temporal)
logic program $\Gamma$, in a Herbrand structure over $\Gamma$ the
constants of the input program constitute the domain, and each
constant is mapped to itself. Formally:

\begin{definition}[Herbrand structure/Herbrand Equilibrium Model/Herbrand Stable Model]
  Let $\Gamma$ be a universal temporal program. A \emph{Herbrand
    structure} (over $\Gamma$) is defined as a QTHT structure
  $\herbrandhandt$, where $C$ is the set of constants occurring in
  $\Gamma$, and $id$ is the identity function. We sometimes abbreviate
  $\herbrandhandt$ as $\thandt$ when this does not lead to confusion.

  We say that $\ttandt$ is a \emph{Herbrand equilibrium model} of
  $\varphi$ iff $\ttandt \models \varphi$, and there is no $\thandt$
  with $\bm{H} < \bm{T}$ such that $\thandt \models \varphi$. A trace
  $\bm{T}$, $T_i \subseteq \A(C,P)$ is a \emph{Herbrand stable model}
  of $\varphi$ iff $\ttandt$ is a Herbrand equilibrium model of
  $\varphi$.

\end{definition}

In the following subsection we do not restrict our discussion to
Herbrand structures, making our results applicable in settings where
this restriction is not desirable, e.g.: in \cite{henive08a}.

\subsection{Safety}

The usual non-temporal definition of safety requires that each
variable occurring in a rule also occur in a positive literal of the
body of the rule \cite{gekakasc12a}. This safety definition was
extended in \cite{agcapevidi17a} to a syntactic fragment of temporal
formulas, allowing atoms in the scope of a singe $\Next$ to provide
safety. The authors proved that this extended definition satisfied two
key properties that safety guarantees in the non-temporal setting;
namely, the extended definition ensures \emph{domain independence},
and ensures that there are no \emph{unnamed individuals} in the answer
sets of a program \cite{capeva09a}\cite{capeva09b}. Domain
independence states that the answer sets of the program do not change
if they are ground using a superset of the program's constants. This
is important, as otherwise, adding a new rule that one would expect to
have no affect on the answer sets of the program, can still change the
answer sets simply by the virtue of introducing a new constant symbol,
as seen in Example \ref{exam:shoot-nonground-unsafe}.

To formalize the second property, given an interpretation $\sigma$, we
call an element of the domain $d \in D$ an \emph{unnamed individual},
iff $d \not\in Im(\sigma\vert_{C})$, i.e. if no constant is mapped to
it by $\sigma$. Now, given a trace $\bm{T}$, we define the restriction
of the trace to the images of the constants in $C$ as
$\bm{T}\vert _C$; that is
$\bm{T}\vert_{Ci} \defeq T_i \cap At(\sigma(C),P)$. A answer set
$\bm{T}$ is then said to contain no unnamed individuals iff
$\bm{T} \vert_{C} = \bm{T}$. This property is desirable, as one would
usually not want arbitrary elements of the domain to appear in the
solutions to a theory. Note that in the typical logic programming
setting this property is guaranteed as we restrict the models to be
Herbrand structures.

In what follows, we will relax the safety condition even further than
in \cite{agcapevidi17a}, and prove that these desirable properties
still hold, by retracing the steps of the proofs in the aforementioned
paper. First, we define this extended notion of safety. We denote by
$atoms(\varphi)$ the set of atomic subformulas occurring in $\varphi$.

\begin{definition}[Safe Subformula]\label{def:safe-subformula}
  For a formula $\varphi \in \langfcpt$, we define a binary relation
  $\mathrel{S}_{\varphi}$ over $subf(\varphi)$ as follows:
  \[
  \begin{aligned}
    \mathrel{S}_{\varphi} \defeq \{ &(p,p) \mid p \in atoms(\varphi) \}\\
    \cup \{ &(\gamma, \psi) \mid
              \begin{aligned}[t]
              \gamma, \psi, \xi \in subf(\varphi), \gamma \in \{ &\xi \land \psi, \psi \land \xi, \\
    &\previous \psi, \eventuallyP \psi, \alwaysP \psi, \xi \since \psi, \xi \trigger \psi, \\
    &\Next \psi, \eventuallyF \psi, \alwaysF \psi, \xi \until \psi, \xi \release \psi \} \} \end{aligned}
  \end{aligned}
\]
Let $\mathrel{S}^{+}_{\varphi}$ be the transitive closure of
$\mathrel{S}_{\varphi}$. We say that a subformula
$\gamma \in subf(\varphi)$ is \emph{safe subformula} of $\varphi$ iff
$\varphi \mathrel{S}^{+}_{\varphi} \gamma$. We denote the set of all
such subformulas as $\safe{\varphi}$. We say that an atomic subformula
$p \in atoms(\varphi)$ is a \emph{safe atomic subformula} of $\varphi$ iff
$p \in \safe{\varphi}$
\end{definition}

\begin{definition}[Safe Temporal Rule, Safe Temporal Program]
  We call a variable $x_i$ occurring in a temporal rule
  $\varphi = \alwaysF(B \rightarrow H)$ safe, iff it has an occurrence
  in a safe atomic subformula of $B$. We call a temporal rule safe iff
  all of it's variables are safe. A universal temporal rule
  $\forallx \psi$ is safe iff $\psi$ is safe. A universal temporal
  program is safe, if all of it's rules are safe.
\end{definition}
\begin{example}
  Under this definition of safety, the following universal temporal
  rule from Example \ref{exam:shoot-nonground-unsafe} would be
  considered unsafe.
\begin{equation*}
  \forall x \forall y \alwaysF( \textit{shoot}(x) \land \previous(\textit{not\_load}(y,x) \since \textit{shoot}(x)) 
  \rightarrow \eventuallyF \textit{fail}(x) \land \textit{forgetful}(y)) 
\end{equation*}
This is due to the fact that the atomic subformula
$\textit{not\_load}(y,x)$ is not a safe subformula of the body of the
rule, and as such, $y$ is unsafe as this is the only atomic subformula
of the body in which $y$ occurs. The safety of $x$ is established by
the atom $\textit{shoot}(x)$ in the body. Therefore, if we modify the
rule to the following, it will now be considered safe.
\begin{equation*}
  \forall x \forall y \alwaysF( \textit{shoot}(x) \land \textit{loader}(y) \land \previous(\textit{not\_load}(y,x) \since \textit{shoot}(x)) 
  \rightarrow \eventuallyF \textit{fail}(x) \land \textit{forgetful}(y)) 
\end{equation*}
\end{example}

We will now embark on the path towards proving the two desirable
properties of the new safety definition. As a first step, we will show
some useful properties of implication and quantifier-free ground
temporal formulas. The following two lemmas establish a relationship
between satisfaction of such formulas and their atomic subformulas.

\begin{lemma}\label{lemma:safe-atomic-subformula-satisfied}
  Take a ground formula $\varphi$ from $\langfcdpt$. If for some
  $\rangeco{j}{0}{\lambda}$\\ $\qthandt,j \models \varphi$, then for
  all safe atomic subformulas
  $p \in \safe{\varphi} \cap atoms(\varphi)$, there exists a
  $\kinlambda$, such that $\qthandt,k \models p$.
\end{lemma}
\begin{proof}
  In the following, let $M=\qthandt$. The property can be shown to
  hold by structural induction on $\varphi$. Clearly, the property
  holds for atoms, $\top$ and $\bot$.

  To make the inductive step, let us first consider logical
  connectives. If $\varphi = \psi \wedge \xi$, and
  $M,j \models \varphi$, then $M,j \models \psi$ and
  $M,j \models \xi$. Then applying the induction hypothesis, for any
  atoms $p \in \safe{\psi} \cap atoms(\psi)$ and
  $q \in \safe{\xi} \cap atoms(\xi)$ we must have some
  $k,l \in \intervcc{0}{\lambda}$ such that $M,k \models p$ and
  $M,l \models q$. We can make the inductive step by noting that
  $\safe{\psi \wedge \xi} = \safe{\psi} \cup \safe{\xi}$ and
  $atoms(\psi \wedge \xi) = atoms(\psi) \cup atoms(\xi)$. In cases
  when $\varphi = \psi \vee \xi$ or $\varphi = \neg \psi$, the
  property holds trivially, as $\safe{\varphi} = \emptyset$.

  We will now consider past operators; the same argument can be
  applied to the dual future versions of these operators. If
  $\varphi = \initially$ or $\varphi = \wprevious \psi$, then
  $\safe{\varphi} = \emptyset$. If
  $M,j \models \varphi = \previous \psi$, then $M,j-1 \models \psi$,
  so we can make the inductive step by applying the induction
  hypothesis and noting that
  $\safe{\previous \psi} = \safe{\psi}$. For the cases where
  $\varphi = \xi \since \psi$ resp.  $\varphi = \xi \trigger \psi$, by
  inspecting the satisfaction relation for respective the operators,
  one can conclude that if $M,j \models \varphi$, then it must be the
  case that $M,k \models \psi$ for some $\kinlambda$. Then, applying
  the induction hypothesis, and noting that
  $\safe{\varphi} = \safe{\psi}$ makes the inductive step, and
  concludes the proof.
  
\end{proof}

\begin{definition}[Positive Subformula]\label{def:pos-subformula}
  Let $\otimes$ be any binary temporal/propositional connective, and
  $\oplus$ be any unary temporal/propositional connective, excluding
  negation.  For a formula $\varphi \in \langfcpt$, we define a binary
  relation $\mathrel{P}_{\varphi}$ over $subf(\varphi)$ as follows:
  \[
    \begin{aligned}
      \mathrel{P}_{\varphi} \defeq \{ &(p,p) \mid p \in atoms(\varphi) \}\\
      \cup \{ &(\gamma, \psi) \mid
                  \gamma, \psi, \xi \in subf(\varphi), \gamma \in \{ \oplus \psi, \xi \otimes \psi, \psi \otimes \xi \}
    \end{aligned}
  \]
  Let $\mathrel{P}^{+}_{\varphi}$ be the transitive closure of
  $\mathrel{P}_{\varphi}$. We say that a subformula
  $\gamma \in subf(\varphi)$ is \emph{positive subformula} of
  $\varphi$ iff $\varphi \mathrel{P}^{+}_{\varphi} \gamma$. We denote
  the set of all such subformulas as $pos(\varphi)$.
\end{definition}

\begin{lemma}\label{lemma:t-not-h-atomic-subformula}
  Take a ground formula $\varphi$ from $\langfcdpt$. Then, if for some
  $\rangeco{j}{0}{\lambda}$\\ $\qttandt,j \models \varphi$ and
  $\qthandt,j \not\models \varphi$, it follow that there is a
  $\kinlambda$ and an atomic subformula
  $p \in atoms(\varphi) \cap pos(\varphi)$, such that
  $\qttandt,k \models p$ and $\qthandt,k \not\models p$.
\end{lemma}
\begin{proof}
  The property can easily be shown by structural induction on
  $\varphi$. The only step of note is that of $\varphi = \neg \psi$;
  in this case if $\qttandt_j \not\models \psi$, then by
  persistence we must have $\qthandt_j \not\models \psi$, so the
  property holds vacuously.
\end{proof}

\subsubsection{No Unnamed Individuals}

\begin{lemma}\label{lemma:var-assign-c-restrict}
  Let $\varphi=\alwaysF (B \rightarrow H)=\alwaysF \psi$ be a
  first-order temporal rule, and $\mu$ a variable assignment in
  $\dsigma$. If $\varphi$ is safe, then for all $\kinlambda$:

$$
\qttandt \models \varphi^\mu \text{ implies } \qttcandt \models \varphi^\mu
$$
\end{lemma}
\begin{proof}
  In the following, let $M=\qttandt$, and $M_C=\qttcandt$.

  Suppose indirectly that $M,0 \models \varphi^\mu$, but
  $M_C,0 \not \models \varphi^\mu$. Then, there must be some
  $\rangeco{i}{0}{\lambda}$, such that $M,i \models \psi^\mu$ and
  $M_C,i \not \models \psi^\mu$, which must mean that
  $M_C,i \models B^\mu$ and $M_C,i \not \models H^\mu$. By
  persistence, we then have $M,i \models B^\mu$, which means we must
  also have $M,i \models H^\mu$, as we know that
  $M,0 \models \alwaysF(B^\mu \rightarrow H^\mu)$. Now, we have found
  that $M,i \models H^\mu$ and $M_C,i \not \models H^\mu$. By applying
  Lemma \ref{lemma:t-not-h-atomic-subformula}, we derive that there is
  some atomic subformula $q \in atoms(H) \cap pos(H)$, and a
  $\rangeco{j}{0}{\lambda}$, such that $M,j \models q^\mu$ and
  $M_C,j \not \models q^\mu$. Therefore, we must have a variable $x$
  in $q$, with $\mu(x) \not \in \sigma(C)$. As $x$ is safe, there must
  be an occurrence of $x$ in a safe atomic subformula $p$ of
  $B$. Furthermore, as $\mu(x) \not \in \sigma(C)$, we also have
  $M_C,l \not \models p^\mu$ for any $\rangeco{l}{0}{\lambda}$. Having
  determined this fact, we can now apply the contraposition of Lemma
  \ref{lemma:safe-atomic-subformula-satisfied} to derive that
  $M_C,i \not \models B^\mu$, a contradiction that concludes the
  proof.
\end{proof}

By making use of Lemmas \ref{lemma:safe-atomic-subformula-satisfied},
\ref{lemma:t-not-h-atomic-subformula} and
\ref{lemma:var-assign-c-restrict}, we can now follow the proofs of
Proposition 5 and Theorem 2 from \cite{agcapevidi17a} more or less
verbatim to obtain the following results.

\begin{proposition}
  For any safe universal temporal rule $\varphi = \forallx \psi$
  \begin{equation*} \qttandt \models \varphi \text{ iff } \qttcandt \models \varphi. \end{equation*}
\end{proposition}

\begin{theorem}[No Unnamed Individuals]
  If $\varphi$ is a safe universal temporal rule, and $\qttandt$ is a
  temporal equilibrium model of $\varphi$, then
  $\bm{T}\vert_{C}=\bm{T}$ and $T_i \subseteq \A(\sigma(C),P)$ for all
  $\rangeco{i}{0}{\lambda}$.
\end{theorem}

\subsubsection{Domain Independence}

The grounding of a universal temporal rule is defined as the set of
ground temporal rules obtained by substituting all possible
combinations of values in place of the universally quantified
variables in the rule. Formally:

\begin{definition}[Grounding]
  Let $V \subseteq D \cup C$ be a finite set of constants and/or
  domain elements. The grounding over $V$ of a universal temporal rule
  $\varphi = \forallx \psi(x_1,x_2,\dots,x_n)$, denoted as
  $\ground{V}{\varphi}$, is defined as:
  \begin{align*}
    \ground{V}{\varphi} \defeq \{ &\psi(v_1,v_2,\dots,v_n) \mid v_i \in V, \rangecc{i}{0}{n}\}
  \end{align*}
  Given a safe universal program $\Gamma$, we define
  $\ground{V}{\Gamma} \defeq \cup_{\varphi \in
    \Gamma}\ground{V}{\varphi}$
\end{definition}

To show that our definition of safety guarantees domain independence
with respect to the formal definition of grounding above, we first
need to prove the following lemma.

\begin{lemma}\label{lemma:not-imsigma-domain-sat}
  Let $\varphi(x)$ be a safe temporal rule, and $\qthandt$ be a
  TQHT structure such that $\bm{T}\vert_{C}=\bm{T}$. Then, for any
  $d \in D \setminus \sigma(C)$, we have

  \begin{equation*}
    \qthandt \models \varphi(d)
  \end{equation*}
  
\end{lemma}

\begin{proof}
  Given a safe temporal rule
  $\varphi(x) = \alwaysF(B(x) \rightarrow H(x))$, by safety, there
  exists some atom $p(x)$ in a safe subformula of $B(x)$. Now let us
  take a TQHT structure $\qthandt$ such that $\bm{T}\vert_{C}=\bm{T}$
  and a domain element $d \in D \setminus \sigma(C)$. Since
  $\bm{T}\vert_{C}=\bm{T}$, we know that for any $\kinlambda$,
  $\qthandt, k \not \models p(d)$, and therefore by Lemma
  \ref{lemma:safe-atomic-subformula-satisfied} for any
  $\rangeco{j}{0}{\lambda}$ we have $\qthandt,j \not \models B(d)$,
  and as such
  $\qthandt \models \alwaysF(B(d) \rightarrow H(d)) = \varphi(d)$,
  concluding the proof.
\end{proof}

By building upon Lemma \ref{lemma:not-imsigma-domain-sat} and Lemma
\ref{lemma:safe-atomic-subformula-satisfied} in the same way as in
\cite{agcapevidi17a}, one can obtain the following results, the final
one showing that our definition of safety guarantees domain independence.

\begin{proposition}\label{prop:sat-iff-sat-groundc}
  Let $\varphi = \forallx \psi$ be a safe universal temporal rule and \\
  $\mathcal{M} = \qthandt$ a QTHT structure such that
  $\bm{T}\vert_{C}=\bm{T}$. Then, the following holds:

  \begin{equation*}
    \mathcal{M} \models \varphi \text{ iff } \mathcal{M} \models \ground{C}{\varphi}
  \end{equation*}
\end{proposition}

\begin{proposition}\label{prop:equil-iff-equil-groundc}
  Let $\varphi$ be a safe universal temporal rule and
  $\mathcal{M} = \qthandt$ a QTHT structure. Then, $\mathcal{M}$ is a
  first-order temporal equilibrium model of $\varphi$ iff
  $\mathcal{M}$ is a first-order temporal equilibrium model of
  $\ground{C}{\varphi}$.
\end{proposition}

\begin{theorem}[Domain Independence]\label{theorem:domain-independence}
  Let $\varphi$ be a safe universal temporal rule. Suppose we expand
  the language $\mathcal{L}$ by considering an expanded set of
  constants $C \subseteq C'$. A total QTHT model $\qttandt$ is a
  temporal equilibrium model of $\ground{C}{\varphi}$ iff it is a
  temporal equilibrium model of $\ground{C'}{\varphi}$.
\end{theorem}

\subsection{Grounding}
Having shown that the desirable properties of domain independence and
no unnamed individuals are guaranteed by our safety definition when
considering arbitrary QTHT structures $\qthandt$, we will now consider
only Herbrand structures in the remainder of this section.

Obtaining a ground instantiation of a universal temporal program in
it's gringo representation cannot simply be done by passing the
program to gringo, as seen at the end of Section
\ref{sec:ground}. This is due to, among other things, the fact that
the simplifications performed by the grounding procedure of gringo are
not equivalent under the semantics of QTHT. We will therefore come up
with an alternative grounding procedure. Our method will work by
constructing a much simpler positive (temporal) logic program based on
the original program, and grounding our original program by using
ground atoms found in the unique Herbrand stable model of this simpler
program. This will also allows us to restrict the number of ground
rules compared to the naive approach of simply substituting any
combination of constants in place of variables.

Formally, given a safe universal temporal rule
$r=\forallx \alwaysF(B \rightarrow H)$. The
transformation of $r$, denoted as $\delta(r)$ is defined as the set of
rules:

\begin{equation*} \delta(r) \defeq \{ \forall x_1, \dots \forall x_n
  \alwaysF(\bigwedge_{\substack{b \in atoms(B)\\ \cap \safe{B}}}b
  \rightarrow h) \mid h \in atoms(H) \cap pos(H)\}
\end{equation*}
For a safe universal program we define
$\delta (\Gamma) \defeq \cup_{r \in \Gamma} \delta(r)$. Note that
since $\Gamma$ is assumed to be safe, $\delta(\Gamma)$ is also
guaranteed to be safe. Furthermore, since finding a temporal
equilibrium model of the program $\delta(\Gamma)$ boils down to just
solving the same positive logic program at each time point,
$\delta(\Gamma)$ has a unique Herbrand stable model.

\begin{proposition}\label{prop:simplified-facts}
  Given a safe universal temporal program $\Gamma$, let $\bm{T}$ be a
  Herbrand stable model of $\Gamma$, and $\bm{J}$ the unique Herbrand
  stable model of $\delta(\Gamma)$. Then, $\bm{T} \subseteq \bm{J}$.
\end{proposition}
\begin{proof} We will show that $\ttjandt \models \Gamma$, from which
  it will follow that $\bm{T} \cap \bm{J} = \bm{T}$ from the fact that
  $\ttandt$ is in equilibrium.

  Since $\Gamma$ is safe, by Proposition
  \ref{prop:sat-iff-sat-groundc} it suffices to show that
  $\ttjandt \models \ground{C}{\Gamma}$. Let $r \in \ground{C}{\Gamma}$, $r =
  \tempruleshort$. To prove that $\ttjandt \models r$, we first note
  that, since we already know that $\ttandt \models r$, it suffices to
  show that, for any time point $\kinlambda$, if
  $\ttjandt_k \models B$, then $\ttjandt_k \models H$. Let us assume
  that $\ttjandt_k \models B$. Then, by persistence
  $\ttandt_k \models B$, which means we must also have
  $\ttandt_k \models H$.

  Now, seeing as we have $\ttjandt_k \models B$, by Lemma
  \ref{lemma:safe-atomic-subformula-satisfied} for all safe atomic
  subformulas $p \in \safe{B} \cap atoms(B)$, there must be some
  $\rangeco{j}{0}{\lambda}$ such that $\ttjandt_j \models p$, which
  implies that $\tjandj_j \models p$ since
  $\bm{T} \cap \bm{J} \subseteq \bm{J}$. As the rules in
  $\delta(\Gamma)$ are simply positive rules enclosed by a $\alwaysF$
  operator, this also implies that $\tjandj_i \models h$ for any
  $h \in atoms(H) \cap pos(H)$ and time point $\rangeco{i}{0}{\lambda}$.

  Having established this, let us suppose indirectly that
  $\ttjandt_k \not\models H$. As $\ttandt_k \models H$, by Lemma
  \ref{lemma:t-not-h-atomic-subformula} we must have some
  $h \in atoms(H) \cap pos(H)$ and some $\rangeco{j}{0}{\lambda}$ such that
  $\ttandt_j \models h$ and $\ttjandt_j \not\models h$. But this
  cannot be the case, as we have shown that $\tjandj_i \models h$, for
  any time point $\rangeco{i}{0}{\lambda}$, so in particular
  $\tjandj_j \models h$, a contradiction which concludes the proof.
\end{proof}

To obtain the set of atoms in $\bm{J}$, note that it suffices to
obtain the unique Herbrand stable model $J$ of the program
$\delta^{\prime}(\Gamma)$ consisting of rules

\begin{equation*} \delta^{\prime}(r) \defeq \{ \forall x_1, \dots \forall x_n
  \bigwedge_{\substack{b \in atoms(B)\\ \cap \safe{B}}}b
  \rightarrow h \mid h \in atoms(H) \cap pos(H)\}
\end{equation*}

as we have $J = J_k$ for any $\kinlambda$. Positive logic programs can
be fully evaluated during grounding by gringo, so one can simply pass
the program $\delta^{\prime}(\Gamma)$ to gringo to obtain $J$ at
grounding time.

\begin{example}\label{exam:shoot-nonground-safe}
  Let the safe universal temporal program $\Gamma$ be the following:
  \begin{lstlisting}[language=clingo,numbers=none]
loader(john).
shoot(potato_gun) :- initial.
next(shoot(potato_gun)) :- initial.
and(eventually_after(fail(X)),forgetful(Y)) 
 :- shoot(X), loader(Y), prev(since(not_load(Y,X),shoot(X))).
  \end{lstlisting}
  For $\lambda=2$ this program has the unique first-order stable
  temporal stable model
  \begin{align*}
  \bm{T} = (&\{ loader(john), shoot(potato\_gun) \},\\
            &\{ loader(john), shoot(potato\_gun), forgetful(john), fail(potato\_gun) \})
  \end{align*}
  The transformed set of (non-temporal) rules
  $\delta^{\prime}(\Gamma)$ would then be:
  \begin{lstlisting}[language=clingo,numbers=none]
loader(john).
shoot(potato_gun).
fail(X) :- shoot(X), loader(Y), shoot(X).
forgetful(Y) :- shoot(X), loader(Y), shoot(X).
  \end{lstlisting}
  Which has the unique Herbrand stable model
  $$
  T = \{ loader(john), shoot(potato\_gun), fail(potato\_gun), forgetful(john) \}
  $$
\end{example}

To make the notion of using a set of atoms to ground a program
precise, we introduce the following definition.

\begin{definition}
  Let $A \subseteq \A(C,P)$ be a set of ground atoms. Given a safe
  universal temporal rule
  $\varphi = \forallx \psi(x_1,x_2,\dots,x_n) = \forallx \alwaysF
  (B(x_1,x_2,\dots,x_n) \rightarrow H(x_1,x_2,\dots,x_n))$, we define
  it's grounding over $A$ as:
  \begin{align*}
    \ground{A}{\varphi}& \defeq \{ \psi(c_1,c_2,\dots,c_n) \mid c_i \in C, \rangecc{i}{0}{n}\\
    &\text{ for all } p \in atoms(B(c_1,c_2,\dots,c_n)) \cap \safe{B(c_1,c_2,\dots,c_n)}:p \in A \}
  \end{align*}
  Given a safe universal program $\Gamma$, we define
  $\ground{A}{\Gamma} \defeq \cup_{\varphi \in
    \Gamma}\ground{A}{\varphi}$
\end{definition}

The grounding of a safe universal temporal program $\Gamma$ over a set
of atoms $\ground{A}{\Gamma}$ results in a subset of the rules
generated via the naive grounding $\ground{C}{\Gamma}$. Nonetheless,
when we use the unique Herbrand stable model of
$\delta^{\prime}(\Gamma)$ as set of atoms we ground over, they share
the same set of equilibrium models, as shown by the following
proposition.

\begin{proposition}
  Let $\Gamma$ be a safe universal temporal program, $\bm{T}$ a trace
  with $T_i \subseteq \A(C,P)$ and $J$ the unique Herbrand stable
  model of $\delta^{\prime}(\Gamma)$. Then, the following three
  assertions are equivalent:
  \begin{enumerate}[label={(\arabic*)}]
    \setlength\itemsep{0.15em}
    \item $\bm{T}$ is a Herbrand stable model of $\Gamma$
    \item $\bm{T}$ is a Herbrand stable model of $\ground{C}{\Gamma}$.
    \item $\bm{T}$ is a Herbrand stable model of $\ground{J}{\Gamma}$.
  \end{enumerate}
\end{proposition}
\begin{proof}
  By Proposition \ref{prop:equil-iff-equil-groundc}, we already know that
  (1) and (2) are equivalent; we will now show that (2) and (3) are equivalent.

  $(2)\Rightarrow (3)$: Suppose that $\ttandt$ is a Herbrand equilibrium
  model of $\ground{C}{\Gamma}$; it then follows that
  $\ttandt \models \ground{J}{\Gamma}$, as
  $\ground{J}{\Gamma} \subseteq \ground{C}{\Gamma}$. Suppose that
  $\thandt \models \ground{J}{\Gamma}$; our goal is to show that then
  necessarily $\bm{H} = \bm{T}$. To show this, it suffices to prove
  that $\thandt \models \ground{C}{\Gamma}$, as if this is the case,
  we must have $\bm{H} = \bm{T}$ due to $\ttandt$ being in equilibrium
  w.r.t $\ground{C}{\Gamma}$.

  Now, take some rule from $r = \tempruleshort$ from
  $\ground{C}{\Gamma} \setminus \ground{J}{\Gamma}$.  We will show
  that for all $\kinlambda$ $\ttandt,k \not\models B$, which, by
  persistence, will mean that
  $\thandt \models \alwaysF( B \rightarrow H)=r$. To this end, note
  that by Proposition \ref{prop:simplified-facts}, we have
  $T_i \subseteq J$ for any $\rangeco{i}{0}{\lambda}$. Notice that as
  $r \in \ground{C}{\Gamma} \setminus \ground{J}{\Gamma}$, there must
  be some atom $p \in atoms(B) \cap safe(B)$ such that $p \not \in J$,
  and therefore $p \not \in T_i$ for all $\rangeco{i}{0}{\lambda}$. We can
  apply the contraposition of Lemma
  \ref{lemma:safe-atomic-subformula-satisfied} to conclude that
  $\ttandt,k \not \models B$ for all $\kinlambda$, concluding the
  proof of this direction.


  $(3)\Rightarrow (2)$: Suppose that $\ttandt$ is a Herbrand
  equilibrium model of $\ground{J}{\Gamma}$. By a similar argument as
  before, we can show that $\ttandt \models r$ for any
  $r \in \ground{C}{\Gamma} \setminus \ground{J}{\Gamma}$, so we have
  $\ttandt \models \ground{C}{\Gamma}$. Suppose
  $\thandt \models \ground{C}{\Gamma}$; then,
  $\thandt \models \ground{J}{\Gamma}$ as
  $\ground{J}{\Gamma} \subseteq \ground{C}{\Gamma}$, so we must have
  $\bm{H} = \bm{T}$ as $\ttandt$ is in equilibrium
  w.r.t. $\ground{J}{\Gamma}$, concluding the proof.
\end{proof}

Our method to obtain the ground instantiation $\ground{J}{\Gamma}$ of
our input program is to to cleverly enhance our original program
$\Gamma$ with external statements similar to the form of those of
$\delta^\prime(\Gamma)$.

As a refresher on how external statements work in gringo, any \verb|a|
marked as external via a \verb|#external a.| statement will mark
\verb|a| to not be removed from bodies of statements and conditions,
even if they do not appear in the head of any rule
\cite{PotasscoUserGuide19}. External statements may also have bodies,
in which case the external statement is ground like a rule, and the
obtained ground heads are marked as external, while the ground body is
discarded. As an example, the program
\begin{lstlisting}[language=clingo]
c(X) :- a(X), b(X).
#external b(1).
#external a(X) : b(X).
\end{lstlisting}
would be ground to the two external statements \verb|#external b(1).|
and \\\verb|#external a(1)|. and the ground rule
\verb|c(1) :- a(1), b(1).|.

Now to describe the external statement we enhance our program
with. The external statements are broken down into two sets, which we
call \emph{head externals} and \emph{body externals}.

In the set of head externals, we have one external statement for each
rule in $r' \in \delta^\prime(\Gamma)$, with the head and bodies of
the external statement being the same as $r'$ (unless $r'$ is a fact,
in which case the external in unnecessary). In the body externals we
have for each rule $r \in \Gamma$ one external statement for each
non-atomic formula $\gamma$ in the body of $r$, the head of which will
be $\gamma$, and the body of which will be the body of (any)
$r' \in \delta^\prime(\Gamma)$.


\begin{example}
  As an example, take the temporal program in Example
  \ref{exam:shoot-nonground-safe}. The externals generated for that program would be:
\begin{lstlisting}[numbers=none]
#external fail(X) : shoot(X); loader(Y); shoot(X).
#external forgetful(Y) : shoot(X); loader(Y); shoot(X).

#external initial.
#external prev(since(not_load(Y,X),shoot(X))) : shoot(X); loader(Y); shoot(X).
  \end{lstlisting}
  The first set of rules are the head externals, while the second set
  are the body externals.
\end{example}

The head externals will, in effect cause the grounder to derive unique
stable model $J$ of $\delta^\prime(\Gamma)$, but in the form of ground
external statements (or facts). This will allow gringo to generate the
ground program $\ground{J}{\Gamma}$ by using $J$ as the
\emph{instantiation domain} \cite[p. 82]{gekakasc12a}, with one
caveat. Gringo will not execute this grounding as we would desire for
rules $r$ who's body contain a non-atomic formula $\gamma$, as it will
not (necessarily) see any rules with $\gamma$ in it's head, and will
therefore discard the rule.

To protect such rules from simplification, we have the set of body
externals. These externals rectify the previous flaw in our approach
by protecting $\gamma$ from simplification, but only for such
substitutions of variables as allowed by the definition of
$\ground{J}{\Gamma}$. This completes the description of our grounding
method.

We carry out the generation of these external statements automatically
using a novel non-ground meta-programming system for ASP which allows
us to specify transformations over non-ground programs. This system is
described, as well as the meta-encoding that performs this
transformation, in the following subsection.
