Before we present our translation, we show that when we restrict
ourselves to finite traces, the $\release/\trigger$ operator can be
defined in terms of the $\until/\since$ operator, respectively
(J. Romero, personal communication 2023).  This fact is relevant to
our discussion, as it allows us to simplify the translation to be
presented shortly by considering the $\release/\trigger$ to also be
derived operators.

\begin{proposition}[Definability of $\release/\trigger$ via $\until/\since$]

If $\lambda \neq \omega$ then:
\begin{align*}
\varphi \release \psi &\equivtht \psi \until (\psi \wedge (\varphi \vee \finally)) \\
\varphi \trigger \psi &\equivtht \psi \since (\psi \wedge (\varphi \vee \initially))
\end{align*}
\end{proposition}
\begin{proof}
  We will prove only the case of $\release$; $\since$ can be handled
  the same way. By Proposition \ref{prop:char-strong-equiv-3val}, it
  suffices to show that for any HT trace $\thandt$ and $\kinlambda$,
  $\thandt,k \models \varphi \release \psi$ iff
  $\thandt,k \models \psi \until (\psi \wedge (\varphi \vee
  \finally))$.

  For the left to right direction assume that
  $\thandt,k \models \varphi \release \psi$. Then, per the definition
  of the operator, either $\thandt,k \models \psi$ for all
  $\rangeco{j}{k}{\lambda}$, or there exists some
  $\rangeco{j}{k}{\lambda}$ such that $\handt,j \models \varphi$ and
  $\handt,i \models \psi$ for all $\rangecc{i}{k}{j}$. In both cases,
  $\thandt,k \models \psi \until (\psi \wedge (\varphi \vee
  \finally))$, with the time point who's existence is required for the
  satisfaction of the until being $\lambda-1$ in the first case, and
  $j$ in the second. For the right to left direction, a similar
  argument can be made.
\end{proof}

Note that in the infinite case, the first equivalence does not hold as
in the infinite setting we have $\models \neg \finally$. In fact, it
was show by \cite{babodife20a} that $\release$ is not definable in
terms of a $\release$-free formula in the infinite case; though the
argument made in \cite[p. 20]{agcadipescscvi20a} to extend the result
to the finite case does not hold, as shown by the proposition
above. We extend the derivation function to release and trigger,
defining
$der(\varphi \release \psi) \defeq \psi \until (\psi \wedge (\varphi
\vee \finally)) \text{ and } der(\varphi \trigger \psi) \defeq \psi \since (\psi
\wedge (\varphi \vee \initially))$.

\subsection{Translating Temporal Rules to Propositional Rules}

We will now present the translation from temporal logic programs to
logic programs. Once this translation is established, the stable
models of the output logic program can then be obtained by an answer
set solver, in our case, \verb|clasp|, which will correspond to the
temporal stable models of the input temporal logic program.

The formalization of the translation is inspired by
\cite[p. 9]{capeva05a}, where the authors define a translation from
arbitrary propositional theories to rules; the proof of our upcoming
theorem will follow a similar logical structure as that of Theorem
2. in the aforementioned paper.

We start by introducing some notation.  Given a temporal formula
$\gamma$, let $subf(\gamma)$ denote all subformulas occurring in
$\gamma$. We consider $\gamma$ to be a subformula of $\gamma$ and so
$\gamma \in subf(\gamma)$. We extend $subf$ to temporal logic programs
in a non-standard way, defining it as the following:
\begin{align*}
subf(\Gamma) \defeq (&\{ subf(b_i) \mid \temprulelong \in \Gamma, \rangecc{i}{1}{n} \}\\
  \cup &\{ subf(h_j) \mid \temprulelong \in \Gamma,\rangecc{j}{1}{m} \})\setminus \A
\end{align*}

The set $subf(\Gamma)$ collects, for each rule in $\Gamma$, every
non-atomic subformula occurring in the implication-free temporal
formulas who's conjunction/disjunction constitute the head/body of the
rule, respectively.

We define the set of temporal formulas over $\A$ with a top-level
derived operator as:
$$
\langdat \defeq \{ \gamma \mid \varphi \in
\langat, \psi \in \langat, 
\gamma \in \{ \wnext\varphi, \wprevious\varphi, \eventuallyF\varphi,
              \eventuallyP\varphi, \alwaysF\varphi, \alwaysP\varphi, \varphi
              \release \psi, \varphi \trigger \psi \} \}
$$
Then, for a temporal logic program $\Gamma$ over signature \A, we
define a new signature $\A_{\Gamma,\lambda}$, over which $\Gamma$'s
translation will be constructed, as follows:
\begin{align*}
  \A_{\Gamma,\lambda} = &\{ L_{\varphi}^k \mid \varphi \in subf(\Gamma), \kinlambda \} \\
                      \cup &\{ L_{der(\gamma)}^k \mid \gamma \in subf(\Gamma) \cap \langdat, \kinlambda \} \\
                      \cup &\{ L^k_{\varphi \since \psi,j} \mid \varphi \since \psi \in subf(\Gamma),  \kinlambda, \rangecc{j}{0}{k} \} \\
                      \cup &\{ L^k_{\varphi \until \psi,j} \mid \varphi \until \psi \in subf(\Gamma),  \kinlambda, \rangecc{j}{k}{\lambda} \}
\end{align*}
This signature contains one atom $L_\varphi^k$ for each time point
$\kinlambda$ and formula $\varphi$ in $subf(\Gamma)$, as well as some
additional auxiliary atoms. The translation of $\Gamma$ will be
constructed in such a way that $L^k_\varphi$ is satisfied by a stable
model of the translation iff $\varphi$ is satisfied by the
corresponding temporal stable model or $\Gamma$ at time point $k$.

To this end we construct for any time point $\kinlambda$ and any
$\gamma \in subf(\Gamma)$ a formula $df^k(\gamma)$ over
$\A_{\Gamma,\lambda}$, called the \emph{definition} of $\gamma$ at
time point $k$. The formulas $df^k(\gamma)$ are shown in the middle
column of Figure \ref{fig:gamma-df-pi}. These formulas are,
in essence, a translation of the satisfaction relations for THT as
described in the meta-language of this text in Definition
\ref{def:tht-sat}, into a symbolic representation using propositional
connectives, while also making use of the strongly equivalent
formulation of derived operators to simplify the definition.

\begin{figure}
\begin{center}
$\begin{array}{|l|l|l|}
\hline 

\gamma & df(\gamma) & \pi^k(\gamma)\\

\hline 

\varphi \wedge \psi 
& L_{\varphi \wedge \psi}^k \leftrightarrow L^k_{\varphi} \wedge L^k_\psi 
& \begin{array}{l}
L_{\varphi \wedge \psi}^k \rightarrow L_{\varphi}^k \\
L_{\varphi \wedge \psi}^k \rightarrow L_\psi^k \\
L_{\varphi}^k \wedge L_\psi^k \rightarrow L_{\varphi \wedge \psi}^k
\end{array} \\

\hline

\varphi \vee \psi 
& L_{\varphi \vee \psi}^k \leftrightarrow L_{\varphi}^k \vee L_\psi^k 
& \begin{array}{l}
L_{\varphi}^k \rightarrow L_{\varphi \vee \psi}^k \\
L_\psi^k \rightarrow L_{\varphi \vee \psi}^k \\
L_{\varphi \vee \psi}^k \rightarrow L_{\varphi}^k \vee L_\psi^k
\end{array} \\

\hline 

\neg \varphi 
& L_{\neg \varphi}^k \leftrightarrow \neg L_{\varphi}^k 
& \begin{array}{l}
\neg L_{\varphi}^k \rightarrow L_{\neg \varphi}^k \\
L_{\neg \varphi}^k \rightarrow \neg L_{\varphi}^k
\end{array} \\

\hline

L^k_{\bot}
& L^k_{\bot} \leftrightarrow \bot
& L^k_{\bot} \rightarrow \bot\\

\hline

L^k_{\top}
& L^k_{\top} \leftrightarrow \top
& \top \rightarrow L^k_{\top}\\

\hline

L^k_{\finally}
& L^k_{\finally} \leftrightarrow \begin{cases} \top &\text{ when } k = \lambda - 1\\ \bot &\text{ when } k < \lambda - 1 \end{cases}
&
    \begin{array}{l}
      \top \rightarrow L^{\lambda - 1}_{\finally}\\
      L^k_{\finally} \rightarrow \bot \text{ when } k < \lambda - 1
    \end{array}\\

\hline

L^k_{\initially}
& L^k_{\initially} \leftrightarrow \begin{cases} \top &\text{ when } k = 0\\ \bot &\text{ when } k > 0 \end{cases}
&
    \begin{array}{l}
      \top \rightarrow L^0_{\initially}\\
      L^k_{\initially} \rightarrow \bot \text{ when } k > 0
    \end{array}\\

\hline

\Next \varphi

& L^k_{\Next \varphi} \leftrightarrow \begin{cases} L^{k+1}_{\varphi} &\text{when } k < \lambda - 1\\ \bot &\text{ when } k = \lambda - 1 \end{cases}
& 
  \begin{array}{l}
    L^k_{\Next \varphi} \rightarrow \begin{cases} L^{k+1}_{\varphi} &\text{when } k < \lambda - 1\\ \bot &\text{ when } k = \lambda - 1 \end{cases}\\
    L^{k+1}_{\varphi} \rightarrow L^k_{\Next \varphi} \qquad \text{when } k < \lambda -1
  \end{array} \\

\hline

\previous \varphi
& L^k_{\previous \varphi} \leftrightarrow \begin{cases} L^{k-1}_{\varphi} &\text{when } 0 < k\\ \bot &\text{when } k = 0 \end{cases}
& \begin{array}{l}
L^k_{\previous \varphi} \rightarrow \begin{cases} L^{k-1}_{\varphi} &\text{when } 0 < k\\ \bot &\text{when } k = 0 \end{cases}\\
L_{\varphi}^{k-1} \rightarrow L_{\previous \varphi}^k \qquad \text{when } 0 < k
\end{array} \\


\hline
\varphi \until \psi
& \begin{aligned} 
  &$(L^k_{\varphi \until \psi} \leftrightarrow \bigvee_{j=k}^{\lambda-1} L^k_{\varphi \until \psi,j}) \land$\\
  &$\bigwedge_{j=k}^{\lambda-1} (L^k_{\varphi \until \psi,j} \leftrightarrow L_\psi^j \wedge \bigwedge_{i=k}^{j-1}L_{\varphi}^i)$ 
  \end{aligned}
&
  \begin{array}{l}
    L^k_{\varphi \until \psi} \rightarrow \bigvee_{j=k}^{\lambda-1} L^k_{\varphi \until \psi,j}\\
    L^k_{\varphi \until \psi,j} \rightarrow  L^k_{\varphi \until \psi} \hfill (\rangeco{j}{k}{\lambda})\\
    L^k_{\varphi \until \psi,j} \rightarrow L^j_{\psi} \hfill (\rangeco{j}{k}{\lambda}) \\
    L^k_{\varphi \until \psi,j} \rightarrow L^i_{\varphi} \hfill (\rangeco{j}{k}{\lambda}, \rangeco{i}{k}{j})\\
    L_\psi^j \wedge \bigwedge_{i=k}^{j-1}L_{\varphi}^i \rightarrow L^k_{\varphi \until \psi,j} \hfill (\rangeco{j}{k}{\lambda})
    
  \end{array} \\

\hline

\varphi \since \psi
& \begin{aligned} 
  &$(L^k_{\varphi \since \psi} \leftrightarrow \bigvee_{j=0}^{k} L^k_{\varphi \since \psi,j}) \land$\\
  &$\bigwedge_{j=0}^{k} (L^k_{\varphi \since \psi,j} \leftrightarrow L_\psi^j \wedge \bigwedge_{i=j+1}^{k}L_{\varphi}^i)$ 
  \end{aligned}
&
  \begin{array}{l}
    L^k_{\varphi \since \psi} \rightarrow \bigvee_{j=0}^{k} L^k_{\varphi \since \psi,j}\\
    L^k_{\varphi \since \psi,j} \rightarrow  L^k_{\varphi \since \psi} \hfill (\rangecc{j}{0}{k})\\
    L^k_{\varphi \since \psi,j} \rightarrow L^j_{\psi} \hfill (\rangecc{j}{0}{k}) \\
    L^k_{\varphi \since \psi,j} \rightarrow L^i_{\varphi} \hfill (\rangecc{j}{0}{k}, \rangeoc{i}{j}{k}) \\
    L_\psi^j \wedge \bigwedge_{i=j+1}^{k}L_{\varphi}^i \rightarrow L^k_{\varphi \since \psi,j} \hfill (\rangecc{j}{0}{k})
    
  \end{array} \\

\hline

\gamma \in \langdat
& L^k_{\gamma} \leftrightarrow L^k_{der(\gamma)}
& \begin{array}{l}
L_{\gamma}^k \rightarrow L_{der(\gamma)}^k \\
L_{der(\gamma)}^k \rightarrow L_{\gamma}^k
\end{array} \\

\hline

\end{array}$
\end{center}
\caption{Table defining $df^k(\gamma)$ and
  $\pi^k(\gamma)$ for any temporal formula $\gamma \in
  subf(\Gamma)$ and $\kinlambda$.\label{fig:gamma-df-pi}}
\end{figure}

We can now define the translation of a temporal logic program $\Gamma$
as:
\begin{align*}
  \chi(\Gamma)  = &\{ \bigwedge_{k=0}^{\lambda-1} L_{b_1}^k \land \dots \land L_{b_n}
                    \rightarrow L_{h_1}^k \lor \dots \lor L_{h_m} \mid \tempruleshort \in \Gamma \} \\
                  & \cup \{ df^k(\gamma) \mid \gamma \in subf(\Gamma), \kinlambda \}
\end{align*}
% The first element of the union adds literals $L_\varphi^0$ who's
% intended meaning is that $\varphi$ must hold at time point $0$ for any
% $\varphi \in \Gamma$. The second element of the union realizes this
% intention by recursively adding the definitions of all subformulas of
% $\Gamma$ for all time points $\kinlambda$.

We would like to use the stable models of this translated theory to
find the temporal stable models of the input theory. It stands to
reason then, that some correspondence between the two must be
proven. The following definition, and the subsequent theorem establish
such a connection. Let us also recall from Proposition
\ref{prop:3-valued-ht-properties} that
$I_m \models \varphi \leftrightarrow \psi$ iff
$I_m(\varphi) = I_m(\psi)$; this will be made use of heavily in the
following proofs.
\begin{definition}[Mapping of Interpretations]
  For any 3-valued interpretation $I$ over $\A_{\Gamma,\lambda}$ with
  $I \in \text{HT}(\chi(\Gamma))$, we define the mapping from $I$ to a
  3-valued temporal interpretation as $\mathcal{M}(I) = m_I$, where
  $m_I(k,a) \defeq I(L_{a}^k)$ for any $\kinlambda$ and $a \in \A$.
\end{definition}

\begin{lemma}\label{lemma:bij-helper}
  For any $I \in \text{HT}(\chi(\Gamma))$, the interpretation $I$ is
  fully determined by it's values on atoms
  $\{ L^k_{a} \mid \kinlambda, a \in A \}$, as the following assertions hold for any $\kinlambda$:
  \begin{enumerate}[label={(\arabic*)}]
    \setlength\itemsep{0.15em}
    \item $I(L^k_{der(\gamma)}) = I(L^k_{\gamma})$ for all $\gamma \in subf(\Gamma) \cap \langdat$
    \item $I(L^k_{\varphi \until \psi,j}) = \min (I(L^j_{\psi}), \min
      \{I(L^i_{\varphi}) \mid k \leq i < j\})$ for all $\rangeco{j}{k}{\lambda}$
    \item
      $I(L^k_{\varphi \since \psi,j}) = \min (I(L^j_{\psi}), \min
      \{I(L^i_{\varphi}) \mid j<i \leq k\})$ for all $\rangecc{j}{0}{k}$
    \item $I(L^k_{\gamma}) = m_I(k,\gamma)$ for all $ \gamma \in subf(\Gamma)$
  \end{enumerate}
\end{lemma}

\begin{proof} We will prove each assertion in order.

  (1): Since $I \models \chi(\Gamma)$, we have
  $I \models \{ df^k(\gamma) \mid \gamma \in subf(\Gamma), \kinlambda
  \}$. In particular, for any $\gamma \in subf(\Gamma) \cap \langdat$,
  $I \models L^k_{\gamma} \leftrightarrow L^k_{der(\gamma)}$, which
  means $I(L^k_{der(\gamma)}) = I(L^k_{\gamma}))$.

  (2): As $I \models df^k(\varphi \until \psi)$, we have
  $I \models \bigwedge_{j=k}^{\lambda-1} (L^k_{\varphi \until \psi,j}
  \leftrightarrow L_\psi^j \wedge
  \bigwedge_{i=k}^{j-1}L_{\varphi}^i)$, which implies that for any
  $\rangeco{j}{k}{\lambda}$
  $$
  I(L^k_{\varphi \until \psi,j}) = I(L_\psi^j \wedge \bigwedge_{i=k}^{j-1}L_{\varphi}^i) = \min (I(L^j_{\psi}), \min \{I(L^i_{\varphi}) \mid k \leq i < j\})
  $$

  (3): Can be show to hold the same way as (2).

  (4): We shall prove the assertion by structural induction on
  $\gamma \in subf(\Gamma)$. The statement holds when $\gamma \in \A$
  per the definition of the mapping $\mathcal{M}$.

  We will show the inductive step when $\gamma$ is of the form
  $\varphi \lor \psi/\varphi \land \psi/\previous \varphi/\varphi
  \since \psi$ or when $\gamma \in \langdat$; a similar argument can
  be made the remaining cases.

  For a binary propositional connective
  $\otimes \in \{ \vee, \wedge\}$ let $f_{\otimes}$, $f_{\otimes}^k$
  denote it's associated 3 valued evaluation functions in the
  non-temporal and temporal setting, respectively. Then, we have
  \begin{align*}
    I(L^k_{\varphi \otimes \psi}) &\stackrel{*}{=} I(L^k_{\varphi} \otimes L^k_{\psi}) 
    = f_{\otimes}(I(L^k_\varphi),I(L^k_\psi)) \stackrel{**}{=} f_{\otimes}(m_I(k,\varphi),m_I(k,\psi))  \\
    &= f_{\otimes}^k(m_I(k,\varphi),m_I(k,\psi)) = m_I(k,\varphi \otimes \psi)\\
    I(L^k_{\previous \varphi}) &\stackrel{*}{=} \left. 
      \begin{cases}
        I(L^{k-1}_\varphi) \stackrel{**}{=} m_I(k-1,\varphi) \text{ if } k > 0\\
        I(\bot) = 0 \text{ if } k = 0 
      \end{cases} = m_I(k,\previous \varphi)\\
    I(L^k_{\varphi \since \psi}) &\stackrel{*}= I(\bigvee_{j=0}^{k} L^k_{\varphi \since \psi,j}) 
    = \max \{ I(L^k_{\varphi \since \psi,j}) \mid 0 \leq j \leq k \}\\
                                  &\stackrel{(3)}{=} \max \{\min (I(L^j_{\psi}), \min \{I(L^i_{\varphi}) \mid j<i \leq k\}) \mid 0 \leq j \leq k\} \\
                                  &\stackrel{**}{=} \max \{\min (m_I(j, \psi), \min \{m_I(i, \varphi) \mid j<i \leq k\}) \mid 0 \leq j \leq k\} \\
                                  &= m_I(k,\varphi \since \psi)\\
  I(L^k_{\gamma}) &\stackrel{*}{=} I(L^k_{der(\gamma)}) \stackrel{**}{=} m_I(k,der(\gamma)) = m_I(k,\gamma) \text{ if } \gamma \in \langdat
  \end{align*}
  where $\stackrel{*}=$ holds due to the fact that
  $I \models \{ df^k(\gamma) \mid \gamma \in subf(\Gamma), \kinlambda
  \}$, $\stackrel{**}=$ holds due to the induction hypothesis, and
  $\stackrel{(3)}{=}$ holds due to point (3) of this Lemma.
\end{proof}

\begin{theorem}[Bijection Between HT and THT
  Models]\label{theorem:translation}
  $\mathcal{M}$ is a bijection between $\text{HT}(\chi(\Gamma))$ and
  $\text{THT}(\Gamma,\lambda)$
\end{theorem}

\begin{proof} We break the proof down into the following steps:
  \begin{enumerate}[label={(\arabic*)}]
    \setlength\itemsep{0.15em}
  \item $\mathcal{M}$ is injective.
  \item The codomain of $\mathcal{M}$ is $\text{THT}(\Gamma,\lambda)$.
  \item $\mathcal{M}$ is surjective.
  \end{enumerate}
    

  (1): Let $I$, $I^{\prime} \in \text{HT}(\chi(\Gamma))$. If the two
  interpretation differ on some atom of the form $L^k_a$, $a \in \A$,
  then of course $I \neq I^{\prime}$, and also
  $m_I \neq m_{I^{\prime}}$. On the other hand, if they match on all
  the atoms of the aforementioned form, then they will also match on
  all atoms in the signature $\A_{\Gamma,\lambda}$ due to Lemma
  \ref{lemma:bij-helper}, and as such $I = I^{\prime}$.

  (2): Let $I \in \text{HT}(\chi(\Gamma))$; we must then show that
  $m_I(0,\Gamma) = 2$. Since $I \models \chi(\Gamma)$, we have for any
  $\tempruleshort \in \Gamma$ that
  $I(\bigwedge_{k=0}^{\lambda-1} L_{b_1}^k \land \dots \land L_{b_n}
  \rightarrow L_{h_1}^k \lor \dots \lor L_{h_m})=2$. We then also have the following equation
  \begin{align*}
  m_I(0,\tempruleshort) &= max\{ m_I(k, \lrulelong ) \mid \kinlambda \}\\
  &\begin{aligned} = max\{ impl(&min\{ m_I(k,b_i) \mid \rangecc{i}{1}{n} \},\\
                               &max\{ m_I(k,h_j) \mid \rangecc{j}{1}{m} \}) \mid \kinlambda \}\end{aligned}\\
  &\begin{aligned} \stackrel{*}{=} max\{ impl(&min\{ I(L^k_{b_i}) \mid \rangecc{i}{1}{n} \},\\
                                &max\{ I(L^k_{h_j}) \mid \rangecc{j}{1}{m} \}) \mid \kinlambda \}\end{aligned}\\
  &= I(\bigwedge_{k=0}^{\lambda-1} L_{b_1}^k \land \dots \land L_{b_n} \rightarrow L_{h_1}^k \lor \dots \lor L_{h_m}) = 2
  \end{align*}
  where $\stackrel{*}=$ holds due to Lemma \ref{lemma:bij-helper}.

  (3): To show that $\mathcal{M}$ is surjective, we must find for any
  $m \in \text{THT}(\Gamma,\lamgda)$ some $I \in \text{HT}(\Gamma)$
  such that $\mathcal{M}(I) = m$. Let us define a 3-valued
  interpretation $I_m$ over $\A_{\Gamma,\lambda}$ as follows. We
  define $I_m(L_{a}^k) \defeq m(k,a)$ for all $a \in \A$,
  $I_m(L^k_{\gamma}) \defeq m(k,\gamma)$ for all
  $\kinlambda, \gamma \in subf(\Gamma)$, and for any other
  $L^k_\varphi \in \A_{\Gamma,\lambda}$, we define
  $I_m(L^k_{\varphi})$ as determined by equations in points (1)-(3) of
  Lemma \ref{lemma:bij-helper}. Clearly we have
  $\mathcal{M}(I_m) = m$, so it suffices to show that
  $I_m \models \chi(\Gamma)$. We break this into two steps, showing
  that $I_m$ satisfies the two sets of formulas who's union
  constitutes $\chi(\Gamma)$.

  For the first set, given that $m \models \Gamma$, for any
  $\tempruleshort \in \Gamma$ we have $m(0,\tempruleshort)=2$, and we
  can construct a similar set of equations as in step (2) of the proof
  to show that
  $I_m(\bigwedge_{k=0}^{\lambda-1} L_{b_1}^k \land \dots \land L_{b_n}
  \rightarrow L_{h_1}^k \lor \dots \lor L_{h_m}) = 2$:

  For the second set we must show that $df^k(\gamma)$ is satisfied by
  $I_m$ for any $\gamma \in subf(\Gamma), \kinlambda$. The following
  equations show that this is the case when $\gamma$ is of the form
  $\varphi \lor \psi/\varphi \land \psi/\previous \varphi/\varphi
  \since \psi$ or when $\gamma \in \langdat$. A similar argument can
  be made in the remaining cases.
  \begin{align*}
  I_m(L^k_{\varphi \otimes \psi}) &= m(k,\varphi \otimes \psi) = f_{\otimes}^k(m(k,\varphi),m(k,\psi))\\
  &= f_{\otimes}(I_m(L^k_\varphi),I_m(L^k_\psi)) = I_m(L^k_{\varphi} \otimes L^k_{\psi}) \text{ if } \otimes \in \{ \land, \lor \}\\
  I_m(L^k_{\previous \varphi}) &= m(k,\previous \varphi) = \begin{cases}
    m(k-1,\varphi)=I_m(L^{k-1}_\varphi) &\text{ if } k > 0\\
    0=I_m(\bot) &\text{ if } k = 0
  \end{cases}\\
    I_m(L_{\varphi \since \psi,j}^k) &= \min (I(L^j_{\psi}), \min \{I(L^i_{\varphi}) \mid j<i \leq k\})\\
                                  &= I_m(L_\psi^j \wedge \bigwedge_{i=j+1}^{k}L_{\varphi}^i) \text{ for all } 0 \leq j \leq k\\
    I_m(L^k_{\varphi \since \psi}) 
                                  &= m(k,\varphi \since \psi)\\ 
                                  &= \max \{\min (m(j, \psi), \min \{m(i, \varphi) \mid j<i \leq k\}) \mid 0 \leq j \leq k\}\\
                                  &= \max \{\min (I_m(L^j_{\psi}), \min \{I_m(L^i_{\varphi}) \mid j<i \leq k\}) \mid 0 \leq j \leq k\} \\
                                  &= \max \{ I_m(L_{\varphi \since \psi,j}^k) \mid 0 \leq j \leq k \} \\
                                  &= I_m(\bigvee_{j=0}^{k} L_{\varphi \since \psi,j}^k )\\
    I_m(L^k_{\gamma}) &= m(k,\gamma) \stackrel{*}{=} m(k,der(\gamma)) = I_m(L^k_{der(\gamma)}) \text{ if } \gamma \in \langdat
  \end{align*}
  where $\stackrel{*}{=}$ holds as $\gamma \equivtht der(\gamma)$ for
  any $\gamma \in \langdat$.
\end{proof}

\begin{corollary}[Bijection of Equilibrium Models]\label{cor:bijection-of-sm}
  $\mathcal{M}\vert_{\text{EL}(\chi(\Gamma))}$ is a bijection
  between $\text{EL}(\chi(\Gamma))$ and $\text{TEL}(\Gamma,\lambda)$.
\end{corollary}

\begin{proof}
  Given Theorem \ref{theorem:translation}, it is enough to show that
  $\mathcal{M}^{-1}\vert_{\text{TEL}(\Gamma)}$ maps surjectively onto
  $\text{EL}(\chi(\Gamma))$.

  First, suppose that $m \in \text{TEL}(\Gamma,\lambda)$, and let
  $I_m = \mathcal{M}^-1(m)$. From Theorem \ref{theorem:translation} we
  know that $I_m \models \chi(\Gamma)$. Suppose indirectly that there
  is some $J \in \text{HT}(\chi(\Gamma))$, such that
  $J^{-1}(0) = I_m^{-1}(0)$ and $J^{-1}(2) \subsetneq I_m^{-1}(2)$. By
  Lemma \ref{lemma:bij-helper}, for such a $J$, there must be some
  atom of the form $L_{a}^k \in \A_{\Gamma,\lambda}$ for which
  $J(L_{a}^k)=1$, $I_m(L_a^k)=2$. But then, we have
  $m_J^{-1}(0) = m^{-1}(0)$ as well as $m_J(k,a) = 1$, $m(k,a) = 2$, a
  contradiction as $m$ is in equilibrium.

  On the other hand, for any any $I \in \text{EL}(\chi(\Gamma))$ we
  can similarly show that we must have $m_I \in \text{TEL}(\Gamma)$.
  As $\mathcal{M}^{-1}(m_I) = I$, $\mathcal{M}^{-1}$ does indeed hit
  $I$, and is therefore surjective.
\end{proof}

Corollary \ref{cor:bijection-of-sm} thus establishes a method of
finding temporal stable models of a temporal logic program $\Gamma$ by
translating it to the propositional theory $\chi(\Gamma)$, and finding
it's stable models. Since we have $m_I(k,a)=I(L_a^k)$ for any
$a\in \A$, for any stable model $T$ of $\chi(\Gamma)$, we be can
construct the corresponding temporal stable model $\bm{T}$ of
$\Gamma$, where $\bm{T} = (T_i)_{i=0}^{\lambda - 1}$ and
$T_i = \{ a \mid L^i_{a} \in T, a \in \A \}$.

However, commonly used answer set solvers do not accept arbitrary
propositional theories as input, but instead restrict the input to
logic programs. Fortunately, for any $\gamma \in subf(\Gamma)$ we can
apply a set of strongly equivalent transformations on $df^k(\gamma)$
described \cite{capeva05a} to transform $df^k(\gamma)$ to a set of
rules $\pi^k(\gamma)$ such that $df^k(\gamma) \equivht
\pi^k(\gamma)$. These sets of rules are depicted in the rightmost
column of Figure \ref{fig:gamma-df-pi}.

For a temporal logic program $\Gamma$ we define it's translation to a logic
program $\tau(\Gamma)$ over signature $\A_{\Gamma,\lambda}$ as:
\begin{align*}
  \tau(\Gamma)  = &\{ \bigwedge_{k=0}^{\lambda-1} L_{b_1}^k \land \dots \land L_{b_n}
                    \rightarrow L_{h_1}^k \lor \dots \lor L_{h_m} \mid \tempruleshort \in \Gamma \} \\
                  & \cup \{ \cup \pi^k(\gamma) \mid \gamma \in subf(\Gamma), \kinlambda \}
\end{align*}

As $df^k(\gamma) \equivht \pi^k(\gamma)$, we have
$\tau(\Gamma) \equivht \chi(\Gamma)$, so in particular they have the
same stable models. Therefore, we can construct the set of temporal
stable models of $\Gamma$ by using the stable models $T$ of
$\tau(\Gamma)$ as before.
