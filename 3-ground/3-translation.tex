\subsection{Translating Temporal Theories}

Before we present our translation, we show that when we restrict
ourselves to finite traces, the $\release/\trigger$ operator can be
defined in terms of the $\until/\since$ operator, respectively
(J. Romero, personal communication 2023).  This fact is relevant to
our discussion, as it allows us to simplify the translation to be
presented shortly by using the replacement property, thereby
simplifying the metaprogramming implementation of the
$\release/\trigger$ operators by reducing it to the $\until/\since$
operator.

\begin{proposition}[Definability of $\release/\trigger$ via $\until/\since$]

If $\lambda \neq \infty$ then:
\begin{align*}
\varphi \release \psi &\equivtht \psi \until (\psi \wedge (\varphi \vee \finally)) \\
\varphi \trigger \psi &\equivtht \psi \since (\psi \wedge (\varphi \vee \initially))
\end{align*}
\end{proposition}
\begin{proof}
  We will show the first equivalence, as the second can be proven
  similarly.  To prove strong equivalence in THT, we need to show that
  for any $\kinlambda$ and any HT trace $\thandt$,
  $\thandt,k \models \varphi \release \psi$ iff
  $\thandt,k \models \psi \until (\psi \wedge (\varphi \vee
  \finally)).$

  For the left to right direction assume that
  $\thandt,k \models \varphi \release \psi$. Then, per the definition
  of the operator, either $\thandt,k \models \psi$ for all
  $\rangeco{j}{k}{\lambda}$, or there exists some
  $\rangeco{j}{k}{\lambda}$ such that $\handt,j \models \psi$ and
  $\handt,i \models \varphi$ for any $i \in \rangeco{i}{k}{j}$. In
  both cases,
  $\handt \models \until (\psi \wedge (\varphi \vee \finally))$, with
  the time point required by the until being $\lambda-1$ in the first
  case, and $j$ in the second. For the right to left direction, a
  similar argument can be made.
\end{proof}

Note that in the infinite case, the first equivalence does not hold as
in the infinite setting we have $\models \neg \finally$. In fact, it
was show by \cite{babodife20a} that $\release$ is not definable in
terms of a $\release$-free formula in the infinite case; though the
argument made in \cite[p. 20]{agcadipescscvi20a} to extend the result
to the finite case does not hold, as shown by the proposition above.

We now present the translation from temporal theories to propositional
rules. The formalization of this translation is inspired by
\cite[p. 9]{capeva05a}, where the authors define a translation from
arbitrary propositional theories to rules; the proof of our upcoming
theorem will follow a similar logical structure as that of Theorem
2. in the aforementioned paper. As a first step, we will translate any
temporal theory to propositional theory; following that, we shall
translate such a propositional theory to a logic program, the stable
models of which can then be obtained by an answer set solver, in our
case, \verb|clasp|, which will correspond to the temporal stable
models of the input temporal theory.

Given a temporal theory $\Gamma$, let $subf(\Gamma)$ denote all
subformulas occurring in $\Gamma$. For a temporal theory $\Gamma$ over
signature \A, we define a new signature $\A_{\Gamma,\lambda}$ as:
$$
\A_{\Gamma,\lambda} = \{ L_{\varphi}^k \mid \kinlambda, \varphi \in subf(\Gamma)
\setminus \{ \bot, \top \} \}
$$

For any time point $\kinlambda$ and any non-atomic formula $\gamma \in \mathcal{L}_\A^T$ the \emph{definition} of $\gamma$ at time point $k$, written as $df^k(\gamma) \in \mathcal{L}_{\A_{\Gamma,\lambda}}$, corresponds to:

\begin{align*}
df^k(\gamma) \defeq \begin{cases}
  L^k_{\gamma} \leftrightarrow L_{\varphi}^k \otimes L_{\psi}^k 
  &\text{ if } \gamma = \varphi \otimes \psi \text{ and } \otimes \in \{ \vee, \wedge, \rightarrow \}\\[2ex]
  L^k_\gamma \leftrightarrow \begin{cases} 
    L^{k+1}_{\varphi} &\text{ when } k < \lambda - 1\\
    \bot &\text{ when } k = \lambda - 1
    \end{cases}
  &\text{ if } \gamma = \Next \varphi \\[2ex]
  L^k_{\gamma} \leftrightarrow \begin{cases} 
    L^{k-1}_{\varphi} &\text{ when } 0 < k\\
    \bot &\text{ when } k = 0
    \end{cases}
  &\text{ if } \gamma = \previous \varphi \\[2ex]
  L^k_{\gamma} \leftrightarrow \bigvee_{j=k}^{\lambda-1}(L_\psi^j \wedge \bigwedge_{i=k}^{j-1}L_{\varphi}^i)
  &\text{ if } \gamma = \varphi \until \psi \\[2ex]
  L^k_{\gamma} \leftrightarrow \bigvee_{j=0}^{k}(L_\psi^j \wedge \bigwedge_{i=j+1}^{k}L_{\varphi}^i)
  &\text{ if } \gamma = \varphi \since \psi \\[2ex]
  L^k_{\gamma} \leftrightarrow L^k_{def(\gamma)}
  &\text{ if } \gamma \in \{ \wnext\varphi, \wprevious\varphi, \eventuallyF\varphi, \eventuallyP\varphi, \alwaysF\varphi, \alwaysP\varphi \}
\end{cases}
\end{align*}

Note that this implicitly also defines $df^k$ for all of our derived
operators. The definitions above, in essence, are a translation of the
satisfaction relations for THT, as described in the meta-language of
this text, into a concrete representation using propositional
connectives.

We can now define the translation
of $\Gamma$ as:
$$
\chi(\Gamma) = \{ L_\varphi^0 \mid \varphi \in \Gamma \} 
\cup \{ df^k(\gamma) \mid \gamma \in subf(\Gamma), \kinlambda \}
$$

The first element of the union adds literals $L_\varphi^0$ who's
intended meaning is that $\varphi$ must hold at time point $0$ for any
$\varphi \in \Gamma$. The second element of the union realizes this
intention by recursively adding the definitions of all subformulas of
$\Gamma$ for all time points $\kinlambda$.

We would like to use the stable models of this translated theory to
find the temporal stable models of the input theory. It stands to
reason then, that some correspondence between the two must be
proven. The following definition, and the subsequent results establish
such a connection.
\begin{definition}[Mapping of Interpretations]
For any temporal theory $\Gamma$ and 3-valued interpretation
$m \in \text{THT}(\Gamma)$, we define the mapping
$\mathcal{I}(m) = I_m$, where $I_m(L^k_\varphi) \defeq
m(k,\varphi))$. Furthermore, for any 3-valued interpretation
$I \in \text{HT}(\chi(\Gamma))$, we define the mapping
$\mathcal{M}(I) = m_I$, where
$m_I(k,\varphi) \defeq I(L_{\varphi}^k)$.
\end{definition}

\begin{theorem}[Bijection of Models]\label{theorem:translation}


  $\mathcal{I}$ is a bijection between
  $\text{THT}(\Gamma,\lambda)$ and $\text{HT}(\chi(\Gamma))$, and
  $\mathcal{M}$ is the inverse function of $\mathcal{I}$.
\end{theorem}

\begin{proof}
  For any $m$ we have
  $m_{I_{m}}(k,\varphi) = I_m(L^k_\varphi) = m(k,\varphi)$. Similarly,
  we get for any $I$ that
  $I_{m_I}(L^k_\varphi) = m_I(k,\varphi) = I(L_{\varphi}^k)$.  To
  prove the theorem then, it remains to show that
  $\mathcal{I}: \text{THT}(\Gamma,\lambda) \rightarrow
  \text{HT}(\chi(\Gamma))$ and
  $\mathcal{M}: \text{HT}(\chi(\Gamma)) \rightarrow
  \text{THT}(\Gamma,\lambda)$.

  For the first step, we will show that
  $\mathcal{I}: \text{THT}(\Gamma,\lambda) \rightarrow
  \text{HT}(\chi(\Gamma))$. Let $m$ be a 3-valued interpretation over
  $\A$ such that $m \models \Gamma$. Then, we must show that
  $I_m \models \chi(\Gamma)$. Given that $I_m \models L^0_\gamma$ for
  any $\gamma \in \Gamma$, as $m \models \Gamma$, we now only have to
  show that
  $I_m \models \{ df^k(\gamma) \mid \gamma \in subf(\Gamma),
  \kinlambda \}$.

  Recall from Proposition \ref{prop:3-valued-ht} that
  $I_m \models \varphi \leftrightarrow \psi$ iff $I_m(\varphi) =
  I_m(\psi)$. For a binary propositional connective
  $\otimes \in \{ \vee, \wedge, \rightarrow \}$ let $f_{\otimes}$,
  $f_{\otimes}^k$ denote it's associated 3 valued evaluation functions
  in the non-temporal and temporal setting, respectively. Then, we
  have
  $I_m \models L^k_{\varphi \otimes \psi} \leftrightarrow L^k_\varphi
  \otimes L^k_\psi$, as:
  $$
  I_m(L^k_{\varphi \otimes \psi}) \defeq m(k,\varphi \otimes \psi) 
  = f_{\otimes}^k(m(k,\varphi),m(k,\psi)) = f_{\otimes}(I_m(L^k_\varphi),I_m(L^k_\psi)) = I_m(L^k_{\varphi} \otimes L^k_{\psi})
  $$
  For $\gamma=\previous \varphi$, we have
  $$
  I_m(L^k_{\previous \varphi}) \defeq m(k,\previous \varphi) = \begin{cases}
    m(k-1,\varphi)=I_m(L^{k-1}_\varphi)=I_m(L^{k-1}_\varphi) &\text{ if } k > 0\\
    0=I_m(\bot) &\text{ if } k = 0
    \end{cases}
  $$
  and for $\gamma=\varphi \since \psi$ we have
  \begin{align*}
  I_m(L^k_{\varphi \since \psi}) \defeq m(k,\varphi \since \psi) &= \max \{\min (m(j, \psi), \min \{m(i, \varphi) \mid j<i \leq k\}) \mid 0 \leq j \leq k\} \\
    &= I_m(\bigvee_{j=0}^{k}(L_\psi^j \wedge \bigwedge_{i=j+1}^{k}L_{\varphi}^i))
  \end{align*}
  A similar argument can also be made for the definitions next and
  until, which concludes the proof of this direction.

  For the second step, we will show that
  $\mathcal{M}: \text{HT}(\chi(\Gamma)) \rightarrow
  \text{THT}(\Gamma,\lambda)$, that is, for any $I$ such that
  $I \models \chi(\Gamma)$, we have $m_I \models \Gamma$. We will show
  by structural induction that if $\gamma \in subf(\Gamma)$, then
  $m_I(k,\gamma) = I(L^k_\gamma)$.  Given this result, we will be able
  to conclude that $m_I \models \Gamma$, as by our initial assumption
  $I(L^0_\gamma)=2$ for any $\gamma \in \Gamma$, and thus
  $m_I(0,\gamma)=I(L^0_\gamma)=2$.

  To prove the structural induction, note that for any atomic formula
  $\gamma \in \A \cup \{ \bot, \top \}$,
  $m_I(k,\gamma) = I(L^k_\gamma)$ does indeed hold per definition of
  $m_I$. Then, the following equalities prove the inductional step
  (again omitting future operators for brevity):
  \begin{align*}
  I(L^k_{\varphi \otimes \psi}) &\stackrel{*}{=} I(L^k_{\varphi} \otimes L^k_{\psi})
  =  f_{\otimes}(I(L^k_\varphi),I(L^k_\psi))\\
  &\stackrel{**}{=} f_{\otimes}^k(m(k,\varphi),m(k,\psi)) =  m_I(k,\varphi \otimes \psi)\\
  I(L^k_{\previous \varphi}) &\stackrel{*}{=} \left. \begin{cases}
    I(L^{k-1}_\varphi)  \stackrel{**}{=} m(k-1,\varphi) \text{ if } k > 0\\
    0 \text{ if } k = 0
    \end{cases} \right\} = m_I(k,\previous \varphi)\\
  I(L^k_{\varphi \since \psi}) 
    & \stackrel{*}{=}  I(\bigvee_{j=0}^{k}(L_\psi^j \wedge \bigwedge_{i=j+1}^{k}L_{\varphi}^i)) \\
    & \stackrel{**}{=} \max \{\min (m(j, \psi), \min \{m(i, \varphi) \mid j<i \leq k\}) \mid 0 \leq j \leq k\} \\
    & =  m_I(k,\varphi \since \psi)
  \end{align*}

  where in equalities $\stackrel{*}{=}$ we use th fact that $I \models
  \{ df^k(\gamma) \mid \gamma \in subf(\Gamma), \kinlambda
  \}$ and in equalities
  $\stackrel{**}{=}$ we use the induction hypothesis.
\end{proof}

\begin{corollary}[Bijection of Equilibrium Models]\label{cor:bijection-of-sm}
  $\mathcal{I}\vert_{\text{TEL}(\Gamma,\lambda)}$ is a bijection
  between $\text{TEL}(\Gamma,\lambda)$ and $\text{EL}(\chi(\Gamma))$
  and $\mathcal{M}\vert_{\text{EL}(\chi(\Gamma))}$ is the inverse
  function of $\mathcal{I}\vert_{\text{TEL}(\Gamma,\lambda)}$.
\end{corollary}

\begin{proof}
  Given Theorem \ref{theorem:translation}, it is enough to show that
  the respective function restrictions do indeed map to the codomains
  as stated in this result.

  For one direction, suppose that $m \in
  \text{TEL}(\Gamma,\lambda)$. From Theorem \ref{theorem:translation}
  we know that $I_m \models \chi(\Gamma)$. Suppose indirectly that
  $I_m$ is not in equilibrium, that is, there is another 3-valued
  interpretation $J \in \text{HT}(\chi(\Gamma))$, such that
  $J^{-1}(0) = I_m^{-1}(0)$ and $J^{-1}(2) \subset I_m^{-1}(2)$. For
  such a $J$, there must be some atom of the form
  $L_{a}^k \in \A_{\Gamma,\lambda}, a \in \A$, for which
  $J(L_{a}^k)=1$, $I(L_a^k)=2$, as if there were no such atom, the
  bijection established in Theorem \ref{theorem:translation} would not
  in fact be a bijection. But, if this is the case, then
  $m_J(k,a) = 1$ and $m(k,a) = 2$, and $m_J \models \Gamma$, so $m$ is
  not in equilibrium, a contradiction.

  A similar argument can be made for the other direction.
\end{proof}

Corollary \ref{cor:bijection-of-sm} thus establishes a method of
finding temporal stable models of a temporal theory $\Gamma$ by
translating it to the propositional theory $\chi(\Gamma)$, and finding
it's stable models; as we have $m_I(k,a)=I(L_a^k)$ we need simply to
extract the values of $I(L_a^k), a\in \A$ from the stable models of
$\chi(\Gamma)$ to construct the temporal stable models of
$\Gamma$. However, commonly used answer set solvers do not accept
arbitrary theories as input, but instead restrict the input to logic
programs. Fortunately, for any $\gamma$ we can apply a set of strongly
equivalent transformations on $df(\gamma)$ described in Figure
\ref{fig:gamma-df-pi} to reduce the definition of the operator to a
set of rules, denoted as $\pi(\gamma)$. The

The
\begin{figure}
\begin{center}
$\begin{array}{|l|l|l|}
\hline 

\gamma & df(\gamma) & \pi(\gamma)\\

\hline 

\varphi \wedge \psi 
& \mathbf{L}_{\varphi \wedge \psi}^k \leftrightarrow \mathbf{L}^k_{\varphi} \wedge \mathbf{L}^k_\psi 
& \begin{array}{l}
\mathbf{L}_{\varphi \wedge \psi}^k \rightarrow \mathbf{L}_{\varphi}^k \\
\mathbf{L}_{\varphi \wedge \psi}^k \rightarrow \mathbf{L}_\psi^k \\
\mathbf{L}_{\varphi}^k \wedge \mathbf{L}_\psi^k \rightarrow \mathbf{L}_{\varphi \wedge \psi}^k
\end{array} \\

\hline

\varphi \vee \psi 
& \mathbf{L}_{\varphi \vee \psi}^k \leftrightarrow \mathbf{L}_{\varphi}^k \vee \mathbf{L}_\psi^k 
& \begin{array}{l}
\mathbf{L}_{\varphi}^k \rightarrow \mathbf{L}_{\varphi \vee \psi}^k \\
\mathbf{L}_\psi^k \rightarrow \mathbf{L}_{\varphi \vee \psi}^k \\
\mathbf{L}_{\varphi \vee \psi}^k \rightarrow \mathbf{L}_{\varphi}^k \vee \mathbf{L}_\psi^k
\end{array} \\

\hline 

\neg \varphi 
& \mathbf{L}_{\neg \varphi}^k \leftrightarrow \neg \mathbf{L}_{\varphi}^k 
& \begin{array}{l}
\neg \mathbf{L}_{\varphi}^k \rightarrow \mathbf{L}_{\neg \varphi}^k \\
\mathbf{L}_{\neg \varphi}^k \rightarrow \neg \mathbf{L}_{\varphi}^k
\end{array} \\

\hline

\varphi \rightarrow \varphi 
& \mathbf{L}_{\varphi \rightarrow \psi}^k \leftrightarrow(\mathbf{L}_{\varphi}^k \rightarrow \mathbf{L}_\psi^k) 
& \begin{array}{l}
\mathbf{L}_{\varphi \rightarrow \psi}^k \wedge \mathbf{L}_{\varphi}^k \rightarrow \mathbf{L}_\psi \\
\neg \mathbf{L}_{\varphi}^k \rightarrow \mathbf{L}_{\varphi \rightarrow \psi}^k \\
\mathbf{L}_\psi^k \rightarrow \mathbf{L}_{\varphi \rightarrow \psi}^k \\
\mathbf{L}_{\varphi}^k \vee \neg \mathbf{L}_\psi^k \vee \mathbf{L}_{\varphi \rightarrow \psi}^k
\end{array} \\

\hline

\previous \varphi
& L^k_{\previous \varphi} \leftrightarrow \begin{cases} L^{k-1}_{\varphi} &\text{when } 0 < k\\ \bot &\text{when } k = 0 \end{cases}
& \begin{array}{l}
L^k_{\previous \varphi} \rightarrow \begin{cases} L^{k-1}_{\varphi} &\text{when } 0 < k\\ \bot &\text{when } k = 0 \end{cases}\\
L_{\varphi}^{k-1} \rightarrow L_{\previous \varphi}^k \qquad \text{when } 0 < k
\end{array} \\

\hline

\Next \varphi
& L^k_{\Next \varphi} \leftrightarrow \begin{cases} L^{k+1}_{\varphi} &\text{when } k < \lambda - 1\\ \bot &\text{ when } k = \lambda - 1 \end{cases}
& 
  \begin{array}{l}
    L^k_{\Next \varphi} \rightarrow \begin{cases} L^{k+1}_{\varphi} &\text{when } k < \lambda - 1\\ \bot &\text{ when } k = \lambda - 1 \end{cases}\\
    L^{k+1}_{\varphi} \rightarrow L^k_{\Next \varphi} \qquad \text{when } k < \lambda -1
  \end{array} \\

\hline
\varphi \until \psi
& L^k_{\varphi \until \psi} \leftrightarrow \bigvee_{j=k}^{\lambda-1}(L_\psi^j \wedge \bigwedge_{i=k}^{j-1}L_{\varphi}^i)
&
  \begin{array}{l}
    L^k_{\varphi \until \psi} \rightarrow \bigvee_{j=k}^{\lambda-1} L^k_{\varphi \until \psi,j}\\
    L^k_{\varphi \until \psi,j} \rightarrow  L^k_{\varphi \until \psi} \text{ for all } \rangeco{j}{k}{\lambda}\\
    L^k_{\varphi \until \psi,j} \rightarrow L^j_{\psi} \text{ for all } \rangeco{j}{k}{\lambda} \\
    L^k_{\varphi \until \psi,j} \rightarrow L^i_{\varphi} \text{ for all } \rangeco{j}{k}{\lambda}, \rangeco{i}{k}{j}\\
    L_\psi^j \wedge \bigwedge_{i=k}^{j-1}L_{\varphi}^i \rightarrow L^k_{\varphi \until \psi,j} \text{ for all } \rangeco{j}{k}{\lambda}
    
  \end{array} \\

\hline

\varphi \since \psi
& L^k_{\varphi \since \psi} \leftrightarrow \bigvee_{j=0}^{k}(L_\psi^j \wedge \bigwedge_{i=j+1}^{k}L_{\varphi}^i)
&
  \begin{array}{l}
    L^k_{\varphi \since \psi} \rightarrow \bigvee_{j=0}^{k} L^k_{\varphi \since \psi,j}\\
    L^k_{\varphi \since \psi,j} \rightarrow  L^k_{\varphi \since \psi} \text{ for all } \rangecc{j}{0}{k}\\
    L^k_{\varphi \since \psi,j} \rightarrow L^j_{\psi} \text{ for all } \rangecc{j}{0}{k} \\
    L^k_{\varphi \since \psi,j} \rightarrow L^i_{\varphi} \text{ for all } \rangecc{j}{0}{k}, \rangeoc{i}{j}{k}\\
    L_\psi^j \wedge \bigwedge_{i=j+1}^{k}L_{\varphi}^i \rightarrow L^k_{\varphi \since \psi,j} \text{ for all } \rangecc{j}{0}{k}
    
  \end{array} \\

\hline

\end{array}$
\end{center}
\caption{Table defining transformation $\pi(\gamma)$ for any temporal formula $\gamma$\label{fig:gamma-df-pi}, omitting derived operators.}
\end{figure}


\textbf{I'm not sure if this actually reduces the number of rules -
  it's more a question of how we can write the translation the most
  succinctly in the meta-telingo implementation}

It is also worthwhile in one case to apply the
non-vocabulary preserving transformation described in
\cite{capeva05a}, as follows. To to avoid the blowup in the number of
rules generated when unfolding the definitions of $\until$ and
$\since$ via the vocabulary-preserving transformation, we modify the
definition of the operators to the following:
\begin{align*}
df^k(\gamma) \defeq \begin{cases}
  (L^k_{\gamma} \leftrightarrow \bigvee_{j=k}^{\lambda-1}L_{\gamma,j}^k) 
  \wedge (L_{\gamma,j}^k \leftrightarrow L_\psi^j \wedge \bigwedge_{i=k+1}^{j}L_{\varphi}^i)
  &\text{ if } \gamma = \varphi \until \psi \\[2ex]
  (L^k_{\gamma} \leftrightarrow  \bigvee_{j=0}^{k} L_{\gamma,j}^k) \wedge 
  (L_{\gamma,j}^k \leftrightarrow L_\psi^j \wedge \bigwedge_{i=j+1}^{k}L_{\varphi}^i)
  &\text{ if } \gamma = \varphi \since \psi \\[2ex]
\end{cases}
\end{align*}

The auxiliary atoms of the form $L_{\varphi \until \psi,j}^k$ that are introduced represent

\textbf{Insert example formula and it's translation here.}
