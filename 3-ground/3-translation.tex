\subsection{Translating Temporal Theories}

Before we present our translation, we show that when we restrict
ourselves to finite traces, the $\release/\trigger$ operator can be
defined in terms of the $\until/\since$ operator, respectively
(J. Romero, personal communication 2023).  This fact is relevant to
our discussion, as it allows us to simplify the translation to be
presented shortly by considering the $\release/\trigger$ to also be
derived operators.

\begin{proposition}[Definability of $\release/\trigger$ via $\until/\since$]

If $\lambda \neq \omega$ then:
\begin{align*}
\varphi \release \psi &\equivtht \psi \until (\psi \wedge (\varphi \vee \finally)) \\
\varphi \trigger \psi &\equivtht \psi \since (\psi \wedge (\varphi \vee \initially))
\end{align*}
\end{proposition}
\begin{proof}
  We will show the first equivalence, as the second can be proven
  similarly.  To prove strong equivalence in THT, we need to show that
  for any $\kinlambda$ and any HT trace $\thandt$,
  $\thandt,k \models \varphi \release \psi$ iff
  $\thandt,k \models \psi \until (\psi \wedge (\varphi \vee
  \finally)).$

  For the left to right direction assume that
  $\thandt,k \models \varphi \release \psi$. Then, per the definition
  of the operator, either $\thandt,k \models \psi$ for all
  $\rangeco{j}{k}{\lambda}$, or there exists some
  $\rangeco{j}{k}{\lambda}$ such that $\handt,j \models \varphi$ and
  $\handt,i \models \psi$ for all $\rangecc{i}{k}{j}$. In both cases,
  $\handt \models \psi \until (\psi \wedge (\varphi \vee \finally))$,
  with the time point that satisfies the until being $\lambda-1$ in
  the first case, and $j$ in the second. For the right to left
  direction, a similar argument can be made.
\end{proof}

Note that in the infinite case, the first equivalence does not hold as
in the infinite setting we have $\models \neg \finally$. In fact, it
was show by \cite{babodife20a} that $\release$ is not definable in
terms of a $\release$-free formula in the infinite case; though the
argument made in \cite[p. 20]{agcadipescscvi20a} to extend the result
to the finite case does not hold, as shown by the proposition
above. We extend the derivation function to release and trigger,
defining
$der(\varphi \release \psi) \defeq \psi \until (\psi \wedge (\varphi
\vee \finally)) \text{ and } der(\varphi \trigger \psi) \defeq \psi \since (\psi
\wedge (\varphi \vee \initially))$.

We now present the translation from temporal logic programs to logic
programs. The formalization of this translation is inspired by
\cite[p. 9]{capeva05a}, where the authors define a translation from
arbitrary propositional theories to rules; the proof of our upcoming
theorem will follow a similar logical structure as that of Theorem
2. in the aforementioned paper. Once the translation from temporal
logic programs to logic programs is established, the stable models of
the output logic program can then be obtained by an answer set solver,
in our case, \verb|clasp|, which will correspond to the temporal
stable models of the input temporal logic program.

We start by introducing some notation. 
Given a temporal formula $\gamma$, let $subf(\gamma)$ denote all
subformulas occurring in $\gamma$. We extend $subf$ over temporal
logic programs , defining it as the following:
\begin{align*}
subf(\Gamma) \defeq &\{ subf(b_i) \mid \temprulelong \in \Gamma, \rangecc{i}{1}{n}, n,m \in \mathbb{N} \}\\
  \cup &\{ subf(h_j) \mid \temprulelong \in \Gamma,\rangecc{j}{1}{m}, n,m \in \mathbb{N} \}
\end{align*}
We define the set of temporal formulas over $\A$ with a top-level
derived operator as:
$$
\mathcal{D}_{\A}^{T} \defeq \{ \gamma \mid \varphi \in
\mathcal{L}_{\A}^{T}, \psi \in \mathcal{L}_{\A}^{T}, 
\gamma \in \{ \wnext\varphi, \wprevious\varphi, \eventuallyF\varphi,
              \eventuallyP\varphi, \alwaysF\varphi, \alwaysP\varphi, \varphi
              \release \psi, \varphi \trigger \psi \} \}
$$
We define a variant of $subf$, $subf^*$ as:
$$
subf^*(\Gamma) \defeq subf(\Gamma) \cup \{ der(\gamma) \mid \gamma \in subf(\Gamma) \cap \mathcal{D}_{\A}^{T} \}\\
$$
To sum up in words, the function $subf^*$ collects each subformula
occurring in implication-free temporal formulas constituting the head
and body of $\Gamma$, and additionally, the derivation of each
subformula that has a top-level derived operator. Note that
$subf^*(\Gamma) \subset \mathcal{F}_{\A}^{T}$.

Then, for a temporal theory $\Gamma$ over signature \A, we define a
new signature $\A_{\Gamma,\lambda}$ as:
$$
\A_{\Gamma,\lambda} = \{ L_{\varphi}^k \mid \kinlambda, \varphi \in subf^*(\Gamma) \}
$$
This signature contains one atom $L_\varphi^k$ for each time point
$\kinlambda$ and formula $\gamma$ in $subf^*(\Gamma)$. For any time
point $\kinlambda$ and any non-atomic formula
$\gamma \in \mathcal{F}_\A^T$ the \emph{definition} of $\gamma$ at
time point $k$, written as $df^k(\gamma)$, is the formula over the new
signature $\A_{\Gamma,\lambda}$:

\begin{align*}
df^k(\gamma) \defeq \begin{cases}
  L^k_{\gamma} \leftrightarrow L_{\varphi}^k \otimes L_{\psi}^k 
  &\text{ if } \gamma = \varphi \otimes \psi \text{ and } \otimes \in \{ \vee, \wedge\}\\[2ex]
  L^k_{\gamma} \leftrightarrow \gamma
  &\text{ if } \gamma \in \{ \bot, \top \}\\[2ex]
  L^k_\gamma \leftrightarrow \begin{cases} 
    \top &\text{ when } k = \lambda - 1\\
    \bot &\text{ when } k < \lambda - 1
  \end{cases}
  &\text{ if } \gamma = \finally \\[2ex]
  L^k_\gamma \leftrightarrow \begin{cases} 
    \top &\text{ when } k = 0\\
    \bot &\text{ when } k > 0
  \end{cases}
  &\text{ if } \gamma = \initially \\[2ex]
  L^k_\gamma \leftrightarrow \begin{cases} 
    L^{k+1}_{\varphi} &\text{ when } k < \lambda - 1\\
    \bot &\text{ when } k = \lambda - 1
    \end{cases}
  &\text{ if } \gamma = \Next \varphi \\[2ex]
  L^k_{\gamma} \leftrightarrow \begin{cases} 
    L^{k-1}_{\varphi} &\text{ when } 0 < k\\
    \bot &\text{ when } k = 0
    \end{cases}
  &\text{ if } \gamma = \previous \varphi \\[2ex]
  L^k_{\gamma} \leftrightarrow \bigvee_{j=k}^{\lambda-1}(L_\psi^j \wedge \bigwedge_{i=k}^{j-1}L_{\varphi}^i)
  &\text{ if } \gamma = \varphi \until \psi \\[2ex]
  L^k_{\gamma} \leftrightarrow \bigvee_{j=0}^{k}(L_\psi^j \wedge \bigwedge_{i=j+1}^{k}L_{\varphi}^i)
  &\text{ if } \gamma = \varphi \since \psi \\[2ex]
  L^k_{\gamma} \leftrightarrow L^k_{der(\gamma)}
  &\text{ if } \gamma \in \mathcal{D}^T_{\A}
\end{cases}
\end{align*}

The values assigned by $df^k(\gamma)$ above are, in essence, a
translation of the satisfaction relations for THT, as described in the
meta-language of this text, into a concrete representation using
propositional connectives, while also making use of the strongly
equivalent formulation of derived operators to simplify the
definition.

We can now define the translation of a temporal logic program $\Gamma$
as:
\begin{align*}
  \chi(\Gamma)  = &\{ \bigwedge_{k=0}^{\lambda-1} L_{b_1}^k \land \dots \land L_{b_n}
                    \rightarrow L_{h_1}^k \lor \dots \lor L_{h_m} \mid \tempruleshort \in \Gamma \} \\
                  & \cup \{ df^k(\gamma) \mid \gamma \in subf^*(\Gamma), \kinlambda \}
\end{align*}
% The first element of the union adds literals $L_\varphi^0$ who's
% intended meaning is that $\varphi$ must hold at time point $0$ for any
% $\varphi \in \Gamma$. The second element of the union realizes this
% intention by recursively adding the definitions of all subformulas of
% $\Gamma$ for all time points $\kinlambda$.

We would like to use the stable models of this translated theory to
find the temporal stable models of the input theory. It stands to
reason then, that some correspondence between the two must be
proven. The following definition, and the subsequent results establish
such a connection.
\begin{definition}[Mapping of Interpretations]
  Given a temporal logic program $\Gamma$ over $\A$, we define the
  following mappings.  

  For any 3-valued temporal interpretation $m$
  over $\A$ with $m \in \text{THT}(\Gamma,\lambda)$, we define the
  mapping from $m$ to a 3 valued interpretation over
  $\A_{\Gamma,\lambda}$ as $\mathcal{I}(m) = I_m$, where
  $I_m(L^k_\varphi) \defeq m(k,\varphi))$ for any
  $L^k_\varphi \in \A_{\Gamma,\lambda}$.

  Furthermore, for any 3-valued interpretation $I$ over
  $\A_{\Gamma,\lambda}$ with $I \in \text{HT}(\chi(\Gamma))$, we
  define the mapping from $I$ to a 3-valued function
  $m_I: \intervco{0}{\lambda} \times \mathcal{L}_{\A}^{T} \rightarrow
  \textbf{3}$ as $\mathcal{M}(I) = m_I$, where
  $m_I(k,\varphi) \defeq I(L_{\varphi}^k)$ for any $\kinlambda$ and
  $\varphi \in \mathcal{L}_{\A}^{T}$.
\end{definition}

\begin{theorem}[Bijection of Models]\label{theorem:translation}

  For any $I \in \text{HT}(\chi(\Gamma))$, $\mathcal{M}(I)$ is a
  3-valued temporal interpretation over $\A$. Additionally,
  $\mathcal{I}$ is a bijection between $\text{THT}(\Gamma,\lambda)$
  and $\text{HT}(\chi(\Gamma))$, and $\mathcal{M}$ is the inverse
  function of $\mathcal{I}$.
\end{theorem}

\begin{proof}
  Let us first recall from Proposition \ref{prop:3-valued-ht} that
  $I_m \models \varphi \leftrightarrow \psi$ iff
  $I_m(\varphi) = I_m(\psi)$. We begin by showing that $m_I$ is indeed
  a 3-valued interpretation if $I \in \text{HT}(\chi(\Gamma))$. To
  show this, we must verify that the values assigned by $m_I$ satisfy
  the rules defining the extension of 3-valued interpretations to
  formulas, as laid out in Definition
  \ref{def:3-valued-extension-temporal}.
  

  For a binary propositional connective
  $\otimes \in \{ \vee, \wedge\}$ let $f_{\otimes}$, $f_{\otimes}^k$
  denote it's associated 3 valued evaluation functions in the
  non-temporal and temporal setting, respectively. Then, we have
  \begin{align*}
    m_I(k,\varphi \otimes \psi) 
    &= I(L^k_{\varphi \otimes \psi})
    \stackrel{*}{=} I(L^k_{\varphi} \otimes L^k_{\psi})\\
    &= f_{\otimes}(I_m(L^k_\varphi),I_m(L^k_\psi)) 
    = f_{\otimes}^k(m(k,\varphi),m(k,\psi))\\
    m_I(k,\previous \varphi) 
    &= I(L^k_{\previous \varphi}) \stackrel{*}{=} \left. 
      \begin{cases}
        I(L^{k-1}_\varphi) = m(k-1,\varphi) \text{ if } k > 0\\
        0 \text{ if } k = 0 
      \end{cases}\\
    m_I(k,\varphi \since \psi)
    &= I(L^k_{\varphi \since \psi}) \stackrel{*}= I_m(\bigvee_{j=0}^{k}(L_\psi^j \wedge \bigwedge_{i=j+1}^{k}L_{\varphi}^i))\\
    &= \max \{\min (I(L^j_{\psi}), \min \{I(L^i_{\varphi}) \mid j<i \leq k\}) \mid 0 \leq j \leq k\} \\
    &= \max \{\min (m(j, \psi), \min \{m(i, \varphi) \mid j<i \leq k\}) \mid 0 \leq j \leq k\}
  \end{align*}
  where $\stackrel{*}=$ holds due to the fact that $I$ satisfies the
  definitions of the formulas above. A similar argument can be made
  for remaining operators/formulas.

  To prove the bijection, we first note that for any $m$ we have
  $m_{I_{m}}(k,\varphi) = I_m(L^k_\varphi) = m(k,\varphi)$. Similarly, we
  get for any $I$ that
  $I_{m_I}(L^k_\varphi) = m_I(k,\varphi) = I(L_{\varphi}^k)$.  To
  prove the theorem then, it remains to show that
  $\mathcal{I}: \text{THT}(\Gamma,\lambda) \rightarrow
  \text{HT}(\chi(\Gamma))$ and
  $\mathcal{M}: \text{HT}(\chi(\Gamma)) \rightarrow
  \text{THT}(\Gamma,\lambda)$.

  We will first prove that
  $\mathcal{I}: \text{THT}(\Gamma,\lambda) \rightarrow
  \text{HT}(\chi(\Gamma))$. Let $m$ be a 3-valued interpretation over
  $\A$ such that $m \models \Gamma$. Then, we must show that
  $I_m \models \chi(\Gamma)$. We will do so by showing that $I_m$
  satisfies the two sets of formulas who's union constitutes
  $\chi(\Gamma)$.

  For the first set, given that $m \models \Gamma$, for any
  $\tempruleshort \in \Gamma$ we have $m(0,\tempruleshort)=2$ as well
  as the following equalities:
  \begin{align*}
  m(0,\tempruleshort)& = max\{ m(k, \lrulelong ) \mid \kinlambda \}\\
  &\begin{aligned} = max\{ impl(&min\{ m(k,b_i) \mid \rangecc{i}{1}{n} \},\\
                               &max\{ m(k,h_j) \mid \rangecc{j}{1}{n} \}) \mid \kinlambda \}\end{aligned}\\
  &\begin{aligned} = max\{ impl(&min\{ I_m(L^k_{b_i}) \mid \rangecc{i}{1}{n} \},\\
                                &max\{ I_m(L^k_{h_j}) \mid \rangecc{j}{1}{n} \}) \mid \kinlambda \}\end{aligned}\\
  &= I_m(\bigwedge_{k=0}^{\lambda-1} L_{b_1}^k \land \dots \land L_{b_n} \rightarrow L_{h_1}^k \lor \dots \lor L_{h_m})
  \end{align*}
  showing us that $I_m$ satisfies the first set of formulas.

  For the second set we must show that $df^k(\gamma)$ is satisfied by
  $I_m$ for any $\gamma \in subf^*(\Gamma), \kinlambda$. To start off,
  for $\gamma=\varphi \otimes \psi, \otimes \in \{ \land, \lor \}$, we
  have
  $I_m \models L^k_{\varphi \otimes \psi} \leftrightarrow L^k_\varphi
  \otimes L^k_\psi$, as:
  \begin{align*}
  I_m(L^k_{\varphi \otimes \psi}) = m(k,\varphi \otimes \psi) &= f_{\otimes}^k(m(k,\varphi),m(k,\psi))\\
  &= f_{\otimes}(I_m(L^k_\varphi),I_m(L^k_\psi)) = I_m(L^k_{\varphi} \otimes L^k_{\psi})
  \end{align*}
  if $\gamma=\previous \varphi$:
  $$
  I_m(L^k_{\previous \varphi}) = m(k,\previous \varphi) = \begin{cases}
    m(k-1,\varphi)=I_m(L^{k-1}_\varphi)=I_m(L^{k-1}_\varphi) &\text{ if } k > 0\\
    0=I_m(\bot) &\text{ if } k = 0
    \end{cases}
  $$
  if $\gamma=\varphi \since \psi$:
  \begin{align*}
  I_m(L^k_{\varphi \since \psi}) 
    &= m(k,\varphi \since \psi)\\ 
    &= \max \{\min (m(j, \psi), \min \{m(i, \varphi) \mid j<i \leq k\}) \mid 0 \leq j \leq k\}\\
    &= \max \{\min (I(L^j_{\psi}), \min \{I(L^i_{\varphi}) \mid j<i \leq k\}) \mid 0 \leq j \leq k\} \\
    &= I_m(\bigvee_{j=0}^{k}(L_\psi^j \wedge \bigwedge_{i=j+1}^{k}L_{\varphi}^i))
  \end{align*}
  if $\gamma \in \mathcal{D}_{\A}^{T}$
  $$
  I_m(L^k_{\gamma}) = m(k,\gamma) \stackrel{*}{=} m(k,der(\gamma)) = I_m(L^k_{der(\gamma)})
  $$
  where $\stackrel{*}{=}$ can be seen by noting that
  $\gamma \equivtht der(\gamma)$, and applying Proposition
  \ref{prop:3-valued-temporal-properties}. The definitions of the
  remaining formulas can also be easily checked.

  We will now proceed to prove that
  $\mathcal{M}: \text{HT}(\chi(\Gamma)) \rightarrow
  \text{THT}(\Gamma,\lambda)$, that is, for any $I$ such that
  $I \models \chi(\Gamma)$, we have $m_I \models \Gamma$. Since
  $I \models \chi(\Gamma)$, we have for any
  $\tempruleshort \in \Gamma$ that
  $I_m(\bigwedge_{k=0}^{\lambda-1} L_{b_1}^k \land \dots \land L_{b_n}
  \rightarrow L_{h_1}^k \lor \dots \lor L_{h_m})=2$. We can also
  easily show, (as we did before for $I$ and $m_I$) that
  $m(0,\tempruleshort)=I_m(\bigwedge_{k=0}^{\lambda-1} L_{b_1}^k \land
  \dots \land L_{b_n} \rightarrow L_{h_1}^k \lor \dots \lor L_{h_m}$,
  which concludes the proof.
\end{proof}

\begin{corollary}[Bijection of Equilibrium Models]\label{cor:bijection-of-sm}
  $\mathcal{I}\vert_{\text{TEL}(\Gamma,\lambda)}$ is a bijection
  between $\text{TEL}(\Gamma,\lambda)$ and $\text{EL}(\chi(\Gamma))$
  and $\mathcal{M}\vert_{\text{EL}(\chi(\Gamma))}$ is the inverse
  function of $\mathcal{I}\vert_{\text{TEL}(\Gamma,\lambda)}$.
\end{corollary}

\begin{proof}
  Given Theorem \ref{theorem:translation}, it is enough to show that
  the respective function restrictions do indeed map to the codomains
  as stated in this result.

  For one direction, suppose that $m \in
  \text{TEL}(\Gamma,\lambda)$. From Theorem \ref{theorem:translation}
  we know that $I_m \models \chi(\Gamma)$. Suppose indirectly that
  $I_m$ is not in equilibrium, that is, there is another 3-valued
  interpretation $J \in \text{HT}(\chi(\Gamma))$, such that
  $J^{-1}(0) = I_m^{-1}(0)$ and $J^{-1}(2) \subset I_m^{-1}(2)$. For
  such a $J$, there must be some atom of the form
  $L_{a}^k \in \A_{\Gamma,\lambda}$ for which $J(L_{a}^k)=1$,
  $I(L_a^k)=2$, as if there were no such atom, the bijection
  established in Theorem \ref{theorem:translation} would not in fact
  be a bijection. But, if this is the case, then $m_J(k,a) = 1$ and
  $m(k,a) = 2$, and $m_J \models \Gamma$, so $m$ is not in
  equilibrium, a contradiction.

  A similar argument can be made for the other direction.
\end{proof}

Corollary \ref{cor:bijection-of-sm} thus establishes a method of
finding temporal stable models of a temporal logic program $\Gamma$ by
translating it to the propositional theory $\chi(\Gamma)$, and finding
it's stable models. Since we have $m_I(k,a)=I(L_a^k)$ we need simply
to extract the values of $I(L_a^k), a\in \A$ from a stable model of
$\chi(\Gamma)$ to obtain the corresponding temporal stable model of
$\Gamma$. However, commonly used answer set solvers do not accept
arbitrary theories as input, but instead restrict the input to logic
programs. Fortunately, for any $\gamma \in subf^*(\Gamma)$ we can
apply a set of transformations on $df^k(\gamma)$ described in Figure
\ref{fig:gamma-df-pi} to reduce the definition of $\gamma$ to a set of
rules, denoted as $\pi^k(\gamma)$. Figure \ref{fig:gamma-df-pi} omits
the value of $\pi^k(\gamma)$ when $\gamma \in \mathcal{D}_{\A}^{T}$,
in which case it is simply defined as
$\pi^k(\gamma) \defeq \{ L^k_{\gamma} \rightarrow L^k_{der(\gamma)},
L^k_{der(\gamma)} \rightarrow L^k_{\gamma} \}$. Note that the
transformation introduces new auxiliary atoms of the form
$L^k_{\varphi \until \psi,j}$ and $L^k_{\varphi \since \psi,j}$, so we
introduce the revised signature
\begin{align*}
\A_{\Gamma,\lambda}^* = \A_{\Gamma,\lambda} 
&\cup \{ L^k_{\varphi \until \psi,j} \mid L^k_{\varphi \until \psi} \in \A_{\Gamma,\lambda}, \rangeco{j}{0}{\lambda} \}\\
&\cup \{ L^k_{\varphi \since \psi,j} \mid L^k_{\varphi \since \psi} \in \A_{\Gamma,\lambda}, \rangeco{j}{0}{\lambda} \}
\end{align*}
For a temporal logic program $\Gamma$ we define it's translation to a logic
program over signature $\A_{\Gamma,\lambda}^*$ as:
\begin{align*}
  \tau(\Gamma)  = &\{ \bigwedge_{k=0}^{\lambda-1} L_{b_1}^k \land \dots \land L_{b_n}
                    \rightarrow L_{h_1}^k \lor \dots \lor L_{h_m} \mid \tempruleshort \in \Gamma \} \\
                  & \cup \{ \pi^k(\gamma) \mid \gamma \in subf^*(\Gamma), \kinlambda \}
\end{align*}
The transformation from $df^k(\gamma)$ to $\pi^k(\gamma)$ is, for the
most part, simply applying the strongly equivalent, vocabulary
preserving transformations defined in \cite{capeva05a} to unfold the
equivalences in $df^k(\gamma)$ into rules. However, we cannot say that
$\chi(\Gamma) \equivht \tau(\Gamma)$, as they are not over the same
signature due to the auxiliary atoms introduced. Nonetheless, it
straightforward to show that the transformation from $\chi(\Gamma)$ to
$\tau(\Gamma)$ is strongly faithful in the sense of \cite{capeva05a},
so our approach remains valid.

\begin{figure}
\begin{center}
$\begin{array}{|l|l|l|}
\hline 

\gamma & df(\gamma) & \pi^k(\gamma)\\

\hline 

\varphi \wedge \psi 
& \mathbf{L}_{\varphi \wedge \psi}^k \leftrightarrow \mathbf{L}^k_{\varphi} \wedge \mathbf{L}^k_\psi 
& \begin{array}{l}
\mathbf{L}_{\varphi \wedge \psi}^k \rightarrow \mathbf{L}_{\varphi}^k \\
\mathbf{L}_{\varphi \wedge \psi}^k \rightarrow \mathbf{L}_\psi^k \\
\mathbf{L}_{\varphi}^k \wedge \mathbf{L}_\psi^k \rightarrow \mathbf{L}_{\varphi \wedge \psi}^k
\end{array} \\

\hline

\varphi \vee \psi 
& \mathbf{L}_{\varphi \vee \psi}^k \leftrightarrow \mathbf{L}_{\varphi}^k \vee \mathbf{L}_\psi^k 
& \begin{array}{l}
\mathbf{L}_{\varphi}^k \rightarrow \mathbf{L}_{\varphi \vee \psi}^k \\
\mathbf{L}_\psi^k \rightarrow \mathbf{L}_{\varphi \vee \psi}^k \\
\mathbf{L}_{\varphi \vee \psi}^k \rightarrow \mathbf{L}_{\varphi}^k \vee \mathbf{L}_\psi^k
\end{array} \\

\hline 

\neg \varphi 
& \mathbf{L}_{\neg \varphi}^k \leftrightarrow \neg \mathbf{L}_{\varphi}^k 
& \begin{array}{l}
\neg \mathbf{L}_{\varphi}^k \rightarrow \mathbf{L}_{\neg \varphi}^k \\
\mathbf{L}_{\neg \varphi}^k \rightarrow \neg \mathbf{L}_{\varphi}^k
\end{array} \\

\hline

L^k_{\bot}
& \bot
& L^k_{\bot} \rightarrow \bot\\

\hline

L^k_{\top}
& \top
& \top \rightarrow L^k_{\top}\\

\hline

L^k_{\finally}
& L^k_{\finally} \leftrightarrow \begin{cases} \top &\text{ when } k = \lambda - 1\\ \bot &\text{ when } k < \lambda - 1 \end{cases}
&
    \begin{array}{l}
      \top \rightarrow L^{\lambda - 1}_{\finally}\\
      L^k_{\finally} \rightarrow \bot \text{ when } k < \lambda - 1
    \end{array}\\

\hline

L^k_{\initially}
& L^k_{\initially} \leftrightarrow \begin{cases} \top &\text{ when } k = 0\\ \bot &\text{ when } k > 0 \end{cases}
&
    \begin{array}{l}
      \top \rightarrow L^0_{\initially}\\
      L^k_{\initially} \rightarrow \bot \text{ when } k > 0
    \end{array}\\

\hline

\Next \varphi

& L^k_{\Next \varphi} \leftrightarrow \begin{cases} L^{k+1}_{\varphi} &\text{when } k < \lambda - 1\\ \bot &\text{ when } k = \lambda - 1 \end{cases}
& 
  \begin{array}{l}
    L^k_{\Next \varphi} \rightarrow \begin{cases} L^{k+1}_{\varphi} &\text{when } k < \lambda - 1\\ \bot &\text{ when } k = \lambda - 1 \end{cases}\\
    L^{k+1}_{\varphi} \rightarrow L^k_{\Next \varphi} \qquad \text{when } k < \lambda -1
  \end{array} \\

\hline

\previous \varphi
& L^k_{\previous \varphi} \leftrightarrow \begin{cases} L^{k-1}_{\varphi} &\text{when } 0 < k\\ \bot &\text{when } k = 0 \end{cases}
& \begin{array}{l}
L^k_{\previous \varphi} \rightarrow \begin{cases} L^{k-1}_{\varphi} &\text{when } 0 < k\\ \bot &\text{when } k = 0 \end{cases}\\
L_{\varphi}^{k-1} \rightarrow L_{\previous \varphi}^k \qquad \text{when } 0 < k
\end{array} \\


\hline
\varphi \until \psi
& L^k_{\varphi \until \psi} \leftrightarrow \bigvee_{j=k}^{\lambda-1}(L_\psi^j \wedge \bigwedge_{i=k}^{j-1}L_{\varphi}^i)
&
  \begin{array}{l}
    L^k_{\varphi \until \psi} \rightarrow \bigvee_{j=k}^{\lambda-1} L^k_{\varphi \until \psi,j}\\
    L^k_{\varphi \until \psi,j} \rightarrow  L^k_{\varphi \until \psi} \text{ for all } \rangeco{j}{k}{\lambda}\\
    L^k_{\varphi \until \psi,j} \rightarrow L^j_{\psi} \text{ for all } \rangeco{j}{k}{\lambda} \\
    L^k_{\varphi \until \psi,j} \rightarrow L^i_{\varphi} \text{ for all } \rangeco{j}{k}{\lambda}, \rangeco{i}{k}{j}\\
    L_\psi^j \wedge \bigwedge_{i=k}^{j-1}L_{\varphi}^i \rightarrow L^k_{\varphi \until \psi,j} \text{ for all } \rangeco{j}{k}{\lambda}
    
  \end{array} \\

\hline

\varphi \since \psi
& L^k_{\varphi \since \psi} \leftrightarrow \bigvee_{j=0}^{k}(L_\psi^j \wedge \bigwedge_{i=j+1}^{k}L_{\varphi}^i)
&
  \begin{array}{l}
    L^k_{\varphi \since \psi} \rightarrow \bigvee_{j=0}^{k} L^k_{\varphi \since \psi,j}\\
    L^k_{\varphi \since \psi,j} \rightarrow  L^k_{\varphi \since \psi} \text{ for all } \rangecc{j}{0}{k}\\
    L^k_{\varphi \since \psi,j} \rightarrow L^j_{\psi} \text{ for all } \rangecc{j}{0}{k} \\
    L^k_{\varphi \since \psi,j} \rightarrow L^i_{\varphi} \text{ for all } \rangecc{j}{0}{k}, \rangeoc{i}{j}{k}\\
    L_\psi^j \wedge \bigwedge_{i=j+1}^{k}L_{\varphi}^i \rightarrow L^k_{\varphi \since \psi,j} \text{ for all } \rangecc{j}{0}{k}
    
  \end{array} \\

\hline

\end{array}$
\end{center}
\caption{Table defining transformation $\pi^k(\gamma)$ for any temporal formula $\gamma$\label{fig:gamma-df-pi}, omitting derived operators.}
\end{figure}

\textbf{Insert example formula and it's translation here.}
