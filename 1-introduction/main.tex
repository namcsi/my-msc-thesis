\section{Introduction   }
\label{sec:introduction}

Answer Set Programming\cite{martru99a}\cite{niemela99a} has emerged as
a popular approach to declaratively solving practical problems
requiring non-monotonic reasoning. An important development in the
field was the discovery of a purely logical semantics for ASP called
Equilibrium Logic\cite{pearce06a}, allowing further extensions to ASP
to be formalized in a succinct and precise manner. One such extension
of Equilibrium Logic extends ASP with temporal modal operator as in
Linear-time Temporal Logic\cite{pnueli77a}, resulting in Temporal
Equilibrium Logic (TEL)\cite{cabveg07a}. As real-world applications of
ASP often deal with some manner of temporal reasoning, this logic has
a multitude of applications, and have already seen multiple system
implementations based on TEL, such as \emph{STeLP}\cite{cabdie11a} and
\emph{telingo}\cite{cakamosc19a}.

This thesis continues this trend, and is centered around the
implementation of a temporal ASP system, called
\emph{meta-telingo}. As the name suggests, this system is implemented
using meta-programming. We study the theoretical underpinnings of the
system in detail, proving the correctness of the approach on the
ground and non-ground level, tackling issues of safety and grounding
by extending the discussion to Quantified Temporal Equilibrium
Logic\cite{agcapevidi17a}. We also present a novel system called
\emph{renopro} that enables the application of meta-programming to
non-ground ASP programs, and show how it can be applied in the
implementation of meta-telingo. The result is a system which can
handle a very broad class of temporal formulas, allowing for a variety
of programs that the telingo and STeLP system would not accept, but
solving for a finite amount of time steps that must be pre-determined by the
user.

This thesis is organized as follows. In Section \ref{sec:background}
we give an overview of Equilibrium Logic and Temporal Equilibrium
Logic, and present their most important properties. Then, in Section
\ref{sec:ground} we present the core part of meta-telingo, which is
based on a translation from temporal logic programs to regular logic
programs. In Section \ref{sec:non-ground} we extend the discussion of
the system to the non-ground level, give a new definition of safety
and examine it's properties. Finally, in Section \ref{sec:renopro} we
present the renopro system, an how it can be used to finish the
implementation of meta-telingo. We note that Sections
\ref{sec:background} and \ref{sec:ground} of this thesis were
developed as part of an Individual Research Module at the University
of Potsdam.
