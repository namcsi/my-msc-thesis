\documentclass[aspectratio=169,xcolor=svgnames]{beamer} 

\usepackage[utf8]{inputenc}
\usepackage[font=footnotesize]{caption}
\usepackage{hyperref}

% Include image package
\usepackage{graphicx}

\usefonttheme[onlymath]{serif}

\definecolor{universityblue}{RGB}{0, 48, 135}
\definecolor{secondaryblue}{RGB}{65, 182, 230}
\definecolor{universitygray}{RGB}{137, 141, 141}
\definecolor{lightgray}{RGB}{180, 180, 180}
\colorlet{verylightgray}{lightgray!20}

\useinnertheme{circles}
\useoutertheme[right]{sidebar}
\usecolortheme{sidebartab}
\usecolortheme[named=universityblue]{structure}

\setbeamercolor{title}{bg=universityblue, fg=white}
\setbeamercolor{author in head/foot}{bg=universitygray, fg=white}
\setbeamercolor{sidebar right}{bg=universityblue, fg=white}
\setbeamercolor{title in sidebar}{fg=white}
\setbeamercolor{author in sidebar}{fg=white}
\setbeamercolor{section in sidebar shaded}{fg=lightgray}
\setbeamercolor{section in sidebar}{fg=white,bg=secondaryblue}
\setbeamercolor{subsection in sidebar shaded}{fg=lightgray}
\setbeamercolor{subsection in sidebar}{fg=white,bg=secondaryblue}
\setbeamercolor{navigation symbols dimmed}{fg=white}

\setbeamercolor{frametitle}{fg=universityblue}

\setbeamercolor{block title}{use=example text,fg=white,bg=universityblue}
\setbeamercolor{block body}{parent=normal text,use=block title example,bg=verylightgray}
% \setbeamercolor{block title}{fg=white, bg=green!50!black}
% \setbeamercolor{block body}{fg=black, bg=green!10}

\setbeamertemplate{navigation symbols}{}
\makeatletter
\setbeamertemplate{footline}{%
    \leavevmode%
    \hbox{%
        \begin{beamercolorbox}[wd=0.6\paperwidth,ht=2.25ex,dp=1ex,center]{author in head/foot}%
            \usebeamerfont{title in head/foot}%
            \insertshortauthor ~ -- ~\insertshorttitle
        \end{beamercolorbox}%
        \begin{beamercolorbox}[wd=0.3\paperwidth,ht=2.25ex,dp=1ex,center]{author in head/foot}%
            \usebeamerfont{title in head/foot}%
            \vspace{-1.5pt}
            \insertslidenavigationsymbol%
            \insertframenavigationsymbol%
            \insertsubsectionnavigationsymbol%
            \insertsectionnavigationsymbol%
            \insertdocnavigationsymbol%
            \insertbackfindforwardnavigationsymbol%
        \end{beamercolorbox}%
        \begin{beamercolorbox}[wd=0.1\paperwidth,ht=2.25ex,dp=1ex,center]{author in head/foot}%
            \usebeamerfont{title in head/foot}\insertframenumber/\inserttotalframenumber
        \end{beamercolorbox}%
    }
}
\makeatother

% University logo and wordmark
\logo{\includegraphics[scale=0.015]{LogoUP}}
\titlegraphic{\includegraphics[height=2cm]{LogoUP}}



% Section title page
\AtBeginSection[]{
    \begin{frame}
        \centering
        \begin{beamercolorbox}[sep=16pt,center]{title}
            \usebeamerfont{title}\insertsectionhead\par%
        \end{beamercolorbox}
    \end{frame}
}

% \AtBeginEnvironment{theorem}{%
%   \setbeamercolor{block title}{fg=white,bg=universityblue}
%   \setbeamercolor{block body}{parent=normal text,use=block title example,bg=block title example.bg!10!bg}
% }

% Superscript removal for hyperref recognition in pdf string
\newcommand{\supersc}[1]{\texorpdfstring{\textsuperscript{#1}}{}}


\usepackage[normalem]{ulem}
\usepackage{xcolor}
\usepackage[export]{adjustbox}
\usepackage{listings}
\usepackage{tikz}
\usepackage{tikz-qtree}
\usepackage{graphicx}
\usepackage{caption}
\usepackage{subcaption}

\input{include/asp-macros/dtel}
\input{include/asp-macros/logics}
\input{include/asp-macros/systems}
\input{include/asp-macros/ht}
\input{include/asp-macros/automata}
\input{include/asp-macros/listings}



\newcommand{\Next}{\text{\rm \raisebox{-.5pt}{\Large\textopenbullet}}}  % {{\ensuremath{\circ}}}
\renewcommand{\wnext}{\ensuremath{\widehat{\Next}}}

\newcommand{\eqdef}{\ensuremath{=^{def}}}

\newcommand{\A}{\ensuremath{\mathcal{A }}}

\newcommand{\fpbf}{\ensuremath{\bm{\mathit{false}}}}
\newcommand{\tpbf}{\ensuremath{\bm{\mathit{true}}}}


% temporal formalities

\newcommand{\s}[1]{\ensuremath{``#1"\,}}
\newcommand{\interp}[1]{\ensuremath{\lfloor #1 \rceil\,}}

\newcommand{\last}{\ensuremath{\mathit{last}}}
\newcommand{\nnf}{\ensuremath{\mathit{nnf}\,}}
\newcommand{\AFW}{\ensuremath{\mathrm{AFW}}}

\newcommand{\defeq}{\ensuremath{\overset{\mathit{def}}{=}}}

\newcommand{\handt}{\ensuremath{\langle H,T \rangle}}
\newcommand{\tandt}{\ensuremath{\langle T,T \rangle}}
\newcommand{\thandt}{\ensuremath{\langle \bm{H},\bm{T} \rangle}}
\newcommand{\ttandt}{\ensuremath{\langle \bm{T},\bm{T} \rangle}}
\newcommand{\pbf}{\textit{pbf}}

\newcommand{\PV}{\ensuremath{\mathcal{P}}}
\newcommand{\Rel}[2]{\parallel\!#1\!\parallel^{#2}}



\bibliography{include/bibliography/krr.bib}
\bibliography{include/bibliography/procs.bib}
\bibliography{local-bib.bib}

\usepackage[framemethod=tikz]{mdframed}
\usepackage[font=footnotesize]{caption}

% For exported plots and tables
\usepackage{booktabs}
\usepackage[utf8]{inputenc}
\usepackage{pgfplots}
\DeclareUnicodeCharacter{2212}{−}
\usepgfplotslibrary{groupplots,dateplot}
\usetikzlibrary{patterns,shapes.arrows}
\pgfplotsset{compat=newest}

\usepackage{subcaption}
\usepackage[T1]{fontenc}
\usepackage[toc,page]{appendix}
\usepackage{amsmath,amsthm,amssymb}
\usepackage{tabularx}

\lstdefinelanguage{clingo}{
  keywordstyle=[1]\usefont{OT1}{cmtt}{m}{n},%
  keywordstyle=[2]\textbf,%
  keywordstyle=[3]\usefont{OT1}{cmtt}{m}{n},%\textit
  alsoletter={\#,\&},%
  keywords=[1]{not,from,import,def,if,else,return,while,break,and,or,for,in,del,and,class},%
  keywords=[2]{\#const,\#show,\#minimize,\#base,\#theory,\#count,\#external,\#program,\#script,\#end,\#heuristic,\#edge,\#project,\#show},%
  keywords=[3]{&,&dom,&sum,&diff,&show,&minimize},%
  morecomment=[l]{\#\ },%
  morecomment=[l]{\%\ },%
  commentstyle={\color{darkgray}}%
}

\lstset
{ %Formatting for code in appendix
    literate={~}{{\fontfamily{ptm}\selectfont \textasciitilde}}1,
    language=clingo,
    basicstyle=\ttfamily\footnotesize,
    numbers=left,
    stepnumber=1,
    showstringspaces=false,
    tabsize=1,
    breaklines=true,
    breakatwhitespace=false,
    frame=single,
    xleftmargin=2em
}

% \lstset{
%     basicstyle=\ttfamily,
%     numbers=left,
%     firstnumber=1,
%     numberfirstline=true,
%     breaklines=true,                    % sets automatic line breaking
% %   breakatwhitespace=false,            % sets if automatic breaks should only happen at whitespace
%     prebreak = \raisebox{0ex}[0ex][0ex]{\ensuremath{\hookleftarrow}}, % just as an example
% }

%%%%%%%%%%%%%%%%%%%%% Formats %%%%%%%%%%%%%%%%%%%%%%
\definecolor{darkgreen}{rgb}{0,0.5,0}
\definecolor{darkblue}{HTML}{00305E}
\definecolor{darkred}{rgb}{0.8,0,0}
\newtheoremstyle{theoremstyle_space}
{7pt} % Space above
{7pt} % Space below
{} % Body font
{} % Indent amount
{\bfseries} % Theorem head font
{} % Punctuation after theorem head
{\newline} % Space after theorem head: \newline = linebreak
{\thmname{#1}\thmnumber{ #2} \thmnote{(\emph{#3})}}% Theorem head spec (can be left empty, meaning `normal')

\theoremstyle{theoremstyle_space}
\newtheorem{definition}{Definition}
\newtheorem{theorem}{Theorem}
\newtheorem{proposition}{Proposition}
\newtheorem{lemma}{Lemma}
\newtheorem{corollary}{Corollary}

% \newcounter{example}
% \NewCommandCopy{\proofqedsymbol}{\qedsymbol}% save the default
% \newcommand{\exampleqedsymbol}{$\triangle$}% for end of examples

% % ensure that proof has the standard symbol
% \AtBeginEnvironment{proof}{\renewcommand{\qedsymbol}{\proofqedsymbol}}
% \newtheorem{example}{Example}
% \AtBeginEnvironment{example}{%
%   \pushQED{\qed}\renewcommand{\qedsymbol}{$\triangle$}%
% }
% \AtEndEnvironment{example}{\popQED\endexample}

% PRESENTATION PURPOSE
    % * Explaining the problem well
    % * Showing clearly what your encoding does
    % * Justifying your design decisions
    % * Using tests and benchmarks to compare with the other groups

% DOCUMENT INFORMATION
\title[Temporal AST via Metaprogramming]{Implementing a Temporal Answer Set Solver using Metaprogramming Techniques}
\subtitle{MSc. Thesis Defense/IM presentation}
\author{Amadé Nemes}

\institute[]{%
    ~nemes@uni-potsdam.de \\
    ~Cognitive Systems MSc. \\
    ~Universität Potsdam, Institut für Informatik \\
}
\date{\today}

\begin{document}

% TITLE PAGE WITHOUT SIDEBAR
\begingroup
    \makeatletter
    \setlength{\hoffset}{.5\beamer@sidebarwidth}
    \makeatother
    \begin{frame}[plain]
        \maketitle
    \end{frame}
\endgroup

% TABLE OF CONTENTS
\begin{frame}
\frametitle{Table of Contents}
\tableofcontents
\end{frame}

%%% SECTION 1: INTRODUCTION
\section{Introduction}

\begin{frame}[t]{Introduction}
  \begin{itemize}
  \item Temporal ASP is a valuable formalism for modeling dynamic
    systems where negation as failure is needed.
  \item Past work on implementing temporal answer set solvers include
    the telingo system, which solves syntactically restricted set of
    temporal programs incrementally.
  \item In this thesis, we implement a temporal answer set solver using metaprogramming
  \item This system accepts a broader class of temporal formulas than
    telingo, but solves for a fixed amount of time steps
  \end{itemize}
\end{frame}

\section[Temporal Here-and-There]{THT}

%% FRAME: How does MDD-SAT work?
\begin{frame}[t]{Grammar}
A temporal formula over a set of atoms $\A$ is defined by the
following grammar:

\begin{align*}
    \varphi ::=   \; a &\mid \bot \mid \varphi_1 \otimes \varphi_2 \mid\\
  \initially &\mid \previous \varphi \mid \wprevious \varphi \mid \eventuallyP \varphi \mid
  \alwaysP \varphi \mid \varphi_1 \since \varphi_2 \mid \varphi_1 \trigger \varphi_2 \mid\\
  \finally &\mid \Next \varphi \mid \wnext \varphi \mid \eventuallyF \varphi \mid
  \alwaysF \varphi \mid \varphi_1 \until \varphi_2 \mid \varphi_1 \release \varphi_2
\end{align*}
where $\otimes \in \{ \wedge, \vee, \to \}$
\end{frame}

\begin{frame}[t]{Pronunciation Guide}

\[
  \begin{array}{l l|l l}
    \initially & \text{\emph{initial}} & \finally & \text{\emph{final}}\\
    \previous & \text{\emph{previous}} & \Next & \text{\emph{next}}\\
    \wprevious & \text{\emph{weak previous}} & \wnext & \text{\emph{weak next}}\\
    \eventuallyP & \text{\emph{eventually before}} & \eventuallyF & \text{\emph{eventually after}}\\
    \alwaysP & \text{\emph{always before}} & \alwaysF & \text{\emph{always after}}\\
    \since & \text{\emph{since}} & \Next & \text{\emph{until}}\\
    \trigger & \text{\emph{trigger}} & \release & \text{\emph{release}}
\end{array}
\]

\end{frame}

\begin{frame}[t]{HT Traces}

\begin{definition}[Trace]
  A trace $\bm{T}$ of length $\lambda$ over signature $\A$ a (possibly
  infinite) sequence of sets of atoms.
  $\bm{T} \in \A^{\intervco{0}{\lambda}}$.
\end{definition}

\begin{definition}[HT-Trace]
  An HT trace $\thandt$ of length $\lambda$ is an ordered pair of
  traces, where for each $\rangeco{i}{0}{\lambda}$
  $H_i \subseteq T_i$. An $HT$ trace is total iff $\bm{H} = \bm{T}$
\end{definition}
Note: in what follows we will always assume that $\lambda$ is finite.
\end{frame}

\begin{frame}[t]{THT Satisfaction}
\begin{definition}[THT Satisfaction]
An HT-trace $\thandt$ of length $\lambda$ over signature $\A$
  satisfies a temporal formula $\varphi$ at time point
  $\kinlambda$, written as $\thandt,k \models \varphi$ if
  the following condition holds, where we use the shorthand
  $\bf{M} =$ $\thandt$.
\end{definition}
\begin{itemize}
  \item $\bf{M}, k \not \models \bot$
  \item $\bf{M},k \models$ $a$ if $a \in H_{k}$ for any atom $a \in \mathcal{A}$
  \item $\bf{M},k \models \varphi \wedge \psi$ iff $\bf{M}, k \models \varphi$ and $\bf{M}, k \models \psi$
  \item $\bf{M},k \models \varphi \vee \psi$ iff $\bf{M}, k \models \varphi$ or $\bf{M}, k \models \psi$
  \item $\bf{M},k \models \varphi \rightarrow \psi$ iff 
    $\langle \bm{H}^{\prime},\bm{T} \rangle, k \not \models \varphi$ 
    or $\langle \bm{H}^{\prime},\bm{T} \rangle, k \models \psi$, 
    for all $\bm{H}^{\prime} \in\{\bm{H}, \bm{T}\}$
\end{itemize}

\end{frame}

\begin{frame}[t]{THT Satisfaction}
  \begin{itemize}
  \item $\bf{M},k \models \initially$ iff $k=0$
  \item $\bf{M},k \models \previous \varphi$ iff $k>0$ and $\bf{M},k-1 \models \varphi$
  \item $\bf{M},k \models \varphi \since \psi$ iff for some $j \in[0 . . k]$, we have $\bf{M}, j \models \psi$ and $\bf{M},i \models \varphi$ for all $\rangeoc{i}{j}{k}$
  \item $\bf{M},k \models \finally$ iff $k=\lambda - 1$
  \item $\bf{M},k \models \Next \varphi$ iff $k+1<\lambda$ and $\bf{M}, k+1 \models \varphi$
  \item $\bf{M}, k \models \varphi \until \psi$ iff for some $j \in[k . . \lambda)$, we have $\bf{M}, j \models \psi$ and $\bf{M}, i \models \varphi$ for all $\rangeco{i}{k}{j}$
  \end{itemize}

\end{frame}

\begin{frame}[t]{THT models and Temporal Equilibrium Models}

  \begin{itemize}
  \item We say that an HT-trace $\thandt$ is a THT model of a temporal
    theory $\Gamma$ iff $\thandt,0 \models \varphi$ for all
    $\varphi \in \Gamma$.
  \item A formula $\varphi$ is a THT tautology, written as
    $\models \varphi$, if $\handt,k \models \varphi$ for any HT-trace
    $\thandt$ and $\kinlambda$
  \item Two formulas $\varphi$ and $\psi$ are said to be THT
    equivalent, denoted as $\varphi \equivtht \psi$, iff
    $\models \varphi \leftrightarrow \psi$
  \end{itemize}

\begin{definition}[Temporal Equilibrium Model/Temporal Stable Model]
  A total HT-trace $\ttandt$ is said to be a temporal equilibrium
  model of a temporal theory $\Gamma$ iff $\ttandt \models \Gamma$,
  and there is no HT-trace $\thandt$ such that
  $\thandt \models \Gamma$ and $\bm{H} \subset \bm{T}$. A trace
  $\bm{T}$ is a temporal stable model of a temporal theory $\Gamma$
  iff $\ttandt$ is a temporal equilibrium model of $\Gamma$.
\end{definition}

\end{frame}

\begin{frame}[t]{Example temporal theory}
\begin{example}\label{exam:shoot-ground-theory}
  Let us consider the following temporal theory, which models the
  dynamics of shooting a single-shot potato cannon:
\begin{equation*}
\begin{gathered}
\textit{shoot}\\
\Next\textit{not\_load}\\
\Next\Next\textit{shoot}\\
\alwaysF(\textit{shoot} \wedge \previous(\textit{not\_load} \since \textit{shoot})
\rightarrow \eventuallyF \textit{fail} \wedge \textit{forgetful}
\end{gathered}
\end{equation*}

Setting $\lambda=4$, the temporal equilibrium models of this theory
are
\begin{align*}
\bm{T_1}&=(\{\textit{shoot}\}, \{\textit{not\_load}\},
\{\textit{shoot}, \textit{forgetful}\}, \{ \textit{fail} \})\\
\bm{T_2}&=(\{\textit{shoot}\}, \{\textit{not\_load}\},
\{\textit{shoot}, \textit{forgetful}, \textit{fail}\}, \{ \})
\end{align*}

\end{example}
\end{frame}

\begin{frame}{Normal form}
\begin{definition}[Temporal Rule, Temporal Program]
  We call a temporal formula a temporal rule, iff it is of the form:
  $$
  \tempruleshort
  $$
  where $B=b_i \land \dots \land b_n$, $H=h_1 \lor \dots \lor h_m$,
  and $b_i,h_j$ are temporal formulas that contain no implication.  

  We call a set of temporal rules a temporal logic program.
\end{definition}

\begin{equation*}
\begin{gathered}
\alwaysF (\initially \rightarrow \textit{shoot})\\
\alwaysF (\initially \rightarrow \Next\textit{not\_load})\\
\alwaysF (\initially \rightarrow \Next\Next\textit{shoot})\\
\alwaysF(\textit{shoot} \wedge \previous(\textit{not\_load} \since \textit{shoot})
\rightarrow \eventuallyF \textit{fail} \wedge \textit{forgetful})
\end{gathered}
\end{equation*}
\end{frame}

\begin{frame}{Strong Equivalence}

\begin{definition}[Strong Equivalence]
  Two temporal theories $\Gamma_1$ and $\Gamma_2$ are strongly
  equivalent iff for any temporal theory $\Gamma$,
  $\Gamma_1 \cup \Gamma$ and $\Gamma_2 \cup \Gamma$ have the same
  temporal stable models.
\end{definition}

\begin{proposition}[Strong Equivalence and THT Equivalence]
  Two temporal theories $\Gamma_1$ and $\Gamma_2$ are strongly equivalent
  iff $\Gamma_1 \equivtht \Gamma_2$
\end{proposition}

\begin{proposition}[Replacement Property]
  Let $\gamma[\varphi]$ denote a temporal formula with some occurrence of a
  subformula $\varphi$ and let $\gamma[\psi]$ be the temporal formula resulting
  from replacing said occurrence of $\varphi$ with $\psi$. If
  $\varphi \equivtht \psi$, then $\gamma[\varphi] \equivtht \gamma[\psi]$.
\end{proposition}

\end{frame}

\begin{frame}[t]{Derived Operators}
  
\begin{proposition}[Derived Operators]\label{prop:derived-op}
\[
\begin{array}{lll|lll}
\wprevious \varphi & \equivtht & \previous \varphi \vee \initially &
\wnext \varphi & \equivtht & \Next \varphi \vee \finally\\
\eventuallyP \varphi & \equivtht & \top \since \varphi &
\eventuallyF \varphi & \equivtht & \top \until \varphi\\
\alwaysP \varphi & \equivtht & \bot \trigger \varphi &
\alwaysF \varphi & \equivtht & \bot \release \varphi\\
\varphi \trigger \psi &\equivtht & \psi \since (\psi \wedge (\varphi \vee \initially)) &
\varphi \release \psi &\equivtht &\psi \until (\psi \wedge (\varphi \vee \finally))
\end{array}
\]
\end{proposition}
\end{frame}

\begin{frame}[t]{Three-Valued Interpretations}

\begin{definition}
  For any HT trace $\thandt$, we define a corresponding 3-valued
  interpretation $m_{\thandt}$ as 
  \begin{itemize}
  \item $m_{\thandt}(k,a)=2$ iff $a \in H_k$
  \item $m_{\thandt}(k,a)=1$ iff $a \in T_k \setminus H_k$ 
  \item $m_{\thandt}(k,a)=0$ iff $a \not\in T_k$
\end{itemize}
\end{definition}

A 3-valued interpretation can be extended to arbitrary formulas as follows.

\begin{align*}
  m(k, \perp) &= 0 \\
  m(k, \varphi \wedge \psi) &= \min (m(k, \varphi), m(k, \psi)) \\
  m(k, \varphi \vee \psi) &= \max (m(k, \varphi), m(k, \psi)) \\
  m(k, \varphi \rightarrow \psi) &= \operatorname{imp}(m(k, \varphi), m(k, \psi)) \\
\end{align*}
\end{frame}

\begin{frame}{Three-Valued Interpretations}
Let $\operatorname{imp}(x,y)=2$ if $x \leq y$, otherwise
$\operatorname{imp}(x,y)=y$.
\begin{align*}
  m(k, \perp) &= 0 \\
  m(k, \varphi \wedge \psi) &= \min (m(k, \varphi), m(k, \psi)) \\
  m(k, \varphi \vee \psi) &= \max (m(k, \varphi), m(k, \psi)) \\
  m(k, \varphi \rightarrow \psi) &= \operatorname{imp}(m(k, \varphi), m(k, \psi)) \\
  m(k, \previous \varphi) &= \begin{cases}
    0 & \text { if } k=0 \\
    m(k-1, \varphi) & \text { if } k>0
  \end{cases} \\
 m(k, \varphi \since \psi) &= \max \{\min (m(j, \psi), \min \{m(i, \varphi) \mid j<i \leq k\}) \mid 0 \leq j \leq k\} \\
 m(k, \Next \varphi) &= \begin{cases}0 & \text { if } k+1=\lambda(\neq \omega) \\
m(k+1, \varphi) & \text { if } k+1<\lambda\end{cases} \\
 m(k, \varphi \until \psi) &= \max \{\min (m(j, \psi), \min \{m(i, \varphi) \mid k \leq i<j\}) \mid k \leq j<\lambda\} \\
\end{align*}
\end{frame}

\begin{frame}[t]{Three-Valued Interpretations}
\begin{proposition}\label{prop:3-valued-temporal-properties}
  For any 3-valued THT interpretation $m$, $\kinlambda$ and temporal
  formulas $\varphi$, $\psi$:
\begin{description}
\item $m_{\thandt}(k,\varphi) = 2$ iff $\thandt,k \models \varphi$
\item $m_{\thandt}(k,\varphi) = 1$ iff $\ttandt,k \models \varphi$ and $\thandt,k \not\models \varphi$
  \item $m_{\thandt}(k,\varphi) = 0$ iff $\ttandt,k \not\models \varphi$
\item $m_{\thandt}(k,\varphi) = m_{\thandt}(k,\psi)$ iff $\thandt,k \models \varphi \leftrightarrow \psi$
\end{description}
\end{proposition}

\begin{itemize}
\item As a consequence, we can equivalently define THT satisfaction as
  $m \models \varphi$ iff $m(0,\varphi)=2$.
\end{itemize}

\end{frame}

%%% SECTION 3: IMPLEMENTATION
\section{Solving Temporal Logic Programs}

\begin{frame}{Approach}
  \begin{itemize}
  \item We will present the \textit{meta-telingo} system capable of
    solving any temporal logic program given a finite $\lambda$.
  \item The approach translates a temporal logic program into an equivalent
    regular logic program which it then solves.
  \item This approach is implemented via the \textit{meta-programming}
    capabilities of gringo.
  \item We first present the theory behind the translation, and then
    the meta-programming implementation.
  \end{itemize}

\end{frame}

\begin{frame}[t]{Translation - Theory}

  \begin{itemize}
  \item The logic program into which we will be translate our temporal
    logic program $\Gamma$ will use the signature:
  $$
  \A_{\Gamma,\lambda} = \{ L_{\varphi}^k \mid \kinlambda, \varphi \in subf(\Gamma) \}
  $$
\item Goal: $L_{\varphi}^k$ will hold in translated program iff
  $\varphi$ holds in temporal program.
\item To this end, for any $\kinlabda$ and temporal formula, let
  $df^k(\gamma)$ denote the formula over $\A_{\Gamma,\lambda}$ given
  by the following (we show only past operators for brevity):
  \end{itemize}
\end{frame}

\begin{frame}[t]{Translation - Theory}
\begin{align*}
df^k(\gamma) \defeq \begin{cases}
  L^k_{\gamma} \leftrightarrow L_{\varphi}^k \otimes L_{\psi}^k 
  &\text{ if } \gamma = \varphi \otimes \psi \text{ and } \otimes \in \{ \vee, \wedge\}\\[2ex]
  L^k_{\gamma} \leftrightarrow \gamma
  &\text{ if } \gamma \in \{ \bot, \top \}\\[2ex]
  L^k_\gamma \leftrightarrow \begin{cases} 
    \top &\text{ when } k = 0\\
    \bot &\text{ when } k > 0
  \end{cases}
  &\text{ if } \gamma = \initially \\[2ex]
  L^k_{\gamma} \leftrightarrow \begin{cases} 
    L^{k-1}_{\varphi} &\text{ when } 0 < k\\
    \bot &\text{ when } k = 0
    \end{cases}
  &\text{ if } \gamma = \previous \varphi \\[2ex]
  L^k_{\gamma} \leftrightarrow \bigvee_{j=0}^{k}(L_\psi^j \wedge \bigwedge_{i=j+1}^{k}L_{\varphi}^i)
  &\text{ if } \gamma = \varphi \since \psi\\[2ex]
  L^k_{\gamma} \leftrightarrow L^k_{der(\gamma)}
  &\text{ if } \gamma \text{ is derived operator}
\end{cases}
\end{align*}
\end{frame}

\begin{frame}[t]{Translation - Theory}
  \begin{itemize}
  \item We can now define our translation from a temporal program
    $\Gamma$ over $\A$ to the propositional theory $\chi(\Gamma)$ over
    $\A_{\Gamma,\lambda}$ as follows:
\begin{align*}
  \chi(\Gamma)  = &\{ \bigwedge_{k=0}^{\lambda-1} L_{b_1}^k \land \dots \land L_{b_n}
                    \rightarrow L_{h_1}^k \lor \dots \lor L_{h_m} \mid \tempruleshort \in \Gamma \} \\
                  & \cup \{ df^k(\gamma) \mid \gamma \in subf(\Gamma), \kinlambda \}
\end{align*}
\item We would like to show some correspondence between the stable
  models of the two theories.
  \end{itemize}
\end{frame}

\begin{frame}[t]{Translation - Theory}
  \begin{definition}[Mapping of Interpretations]
  Given a temporal logic program $\Gamma$ over $\A$, we define the
  following mappings.  

  For any 3-valued temporal interpretation $m$
  over $\A$ with $m \in \text{THT}(\Gamma,\lambda)$, we define the
  mapping from $m$ to a 3 valued interpretation over
  $\A_{\Gamma,\lambda}$ as $\mathcal{I}(m) = I_m$, where
  $I_m(L^k_\varphi) \defeq m(k,\varphi))$ for any
  $L^k_\varphi \in \A_{\Gamma,\lambda}$.

  Furthermore, for any 3-valued interpretation $I$ over
  $\A_{\Gamma,\lambda}$ with $I \in \text{HT}(\chi(\Gamma))$, we
  define the mapping from $I$ to a 3-valued interpretation $m_I$ over
  $\A$ as $\mathcal{M}(I) = m_I$, where
  $m_I(k,\varphi) \defeq I(L_{\varphi}^k)$ for any $\kinlambda$ and
  $\varphi \in \langat$.
\end{definition}
\begin{theorem}[Bijection of Here-and-There Models]\label{theorem:translation}
  $\mathcal{I}$ is a bijection between $\text{THT}(\Gamma,\lambda)$
  and $\text{HT}(\chi(\Gamma))$, and $\mathcal{M}$ is the inverse
  function of $\mathcal{I}$.
\end{theorem}
\end{frame}

\begin{frame}[t]{Translation - Theory}
\begin{corollary}[Bijection of Equilibrium Models]\label{cor:bijection-of-sm}
  $\mathcal{I}\vert_{\text{TEL}(\Gamma,\lambda)}$ is a bijection
  between $\text{TEL}(\Gamma,\lambda)$ and $\text{EL}(\chi(\Gamma))$
  and $\mathcal{M}\vert_{\text{EL}(\chi(\Gamma))}$ is the inverse
  function of $\mathcal{I}\vert_{\text{TEL}(\Gamma,\lambda)}$.
\end{corollary}
\begin{itemize}
\item Corollary \ref{cor:bijection-of-sm} thus establishes a method of
  finding temporal stable models of a temporal logic program $\Gamma$
  by translating it to the propositional theory $\chi(\Gamma)$, and
  finding it's stable models. 
\item Since we have $m_I(k,a)=I(L_a^k)$ we need simply to extract the
  values of $I(L_a^k), a\in \A$ from a stable model of $\chi(\Gamma)$
  to obtain the corresponding temporal stable model of $\Gamma$
\end{itemize}
\end{frame}

\begin{frame}[t,fragile]{Translation - meta-telingo}
  \begin{itemize}
  \item The ASP grounder gringo offers native support for
    meta-programming by first grounding the input program, and then
    serializing a representation of the ground program as a set of
    facts.
  \item This set of facts can then be fed as input to a meta-encoding,
    which can reinterpret the semantics of the ground language constructs.
  \item The implementation of our translation will be such a
    meta-encoding, referred to as meta-telingo.lp.
  \end{itemize}

\begin{lstlisting}[language=bash,numbers=none]
clingo temporal-program.lp --output=reify | 
  clingo - meta-telingo.lp -c lambda=<n>
\end{lstlisting}

\end{frame}

\begin{frame}[t,fragile]{Translation - meta-telingo Input Language}
The temporal logic program
\begin{equation*}
\begin{gathered}
\alwaysF (\initially \rightarrow \textit{shoot})\\
\alwaysF (\initially \rightarrow \Next\textit{not\_load})\\
\alwaysF (\initially \rightarrow \Next\Next\textit{shoot})\\
\alwaysF(\textit{shoot} \wedge \previous(\textit{not\_load} \since \textit{shoot})
\rightarrow \eventuallyF \textit{fail} \wedge \textit{forgetful})
\end{gathered}
\end{equation*}
corresponds to the following program in the input language of
meta-telingo (e.g. temporal-program.lp).
\begin{center}
    \begin{lstlisting}[numbers=none]
shoot :- initial.
next(not_load) :- initial.
next(next(shoot)) :- initial.
and(eventually_after(fail),forgetful) 
  :- shoot, prev(since(not_load,shoot)).
    \end{lstlisting}
\end{center}
\end{frame}

\begin{frame}[t,fragile]{Translation - meta-telingo Implementation}
    \begin{lstlisting}[]
time(0..lambda-1).

conjunction(B,K) :- literal_tuple(B), time(K),
        hold(L,K) : literal_tuple(B, L), L > 0;
    not hold(L,K) : literal_tuple(B,-L), L > 0.
body(normal(B),K) :- rule(_,normal(B)), conjunction(B,K), 
                     time(K).
hold(A,K) : atom_tuple(H,A) :- rule(disjunction(H),B), 
                               body(B,K), time(K).
    \end{lstlisting}
\begin{align*}
  \chi(\Gamma)  = &\{ \bigwedge_{k=0}^{\lambda-1} L_{b_1}^k \land \dots \land L_{b_n}
                    \rightarrow L_{h_1}^k \lor \dots \lor L_{h_m} \mid \tempruleshort \in \Gamma \} \\
                  & \cup \{ df^k(\gamma) \mid \gamma \in subf^*(\Gamma), \kinlambda \}
\end{align*}
\end{frame}

\begin{frame}[t,fragile]{Translation - meta-telingo Implementation}
      \begin{lstlisting}[]
true(O,K) :- hold(L,K), output(O,B), literal_tuple(B,L).
hold(L,K) :- true(O,K), output(O,B), literal_tuple(B,L).
true(O,K) :- time(K), output(O,B), not literal_tuple(B,_).

subformula(O) :- output(O,_).
subformula((F;G)) :- subformula((and(F,G);or(F,G);until(F,G);since(F,G))).
subformula(F)	 :- subformula((neg(F);next(F);prev(F))).
    \end{lstlisting}
\end{frame}


\begin{frame}[t,fragile]{Translation - meta-telingo Implementation}
    \begin{lstlisting}[]
true(and(F,G),K) :- subformula(and(F,G)), true(F,K), true(G,K).
true((F;G),K) :- subformula(and(F,G)), true(and(F,G),K).

true(F,K-1): time(K-1) :- subformula(prev(F)), true(prev(F),K).
true(prev(F),K) :- true(F,K-1), subformula(prev(F)), time(K).
    \end{lstlisting}
\begin{align*}
  L^k_{\varphi \lor \psi} &\leftrightarrow L_{\varphi}^k \lor L_{\psi}^k\\
  L^k_{\previous \varphi} &\leftrightarrow \begin{cases} 
    L^{k-1}_{\varphi} &\text{ when } 0 < k\\
    \bot &\text{ when } k = 0
    \end{cases}
\end{align*}
\end{frame}

\begin{frame}[t,fragile]{Translation - meta-telingo Implementation}
    \begin{lstlisting}[]
subformula(F2) :- derive(F1,F2), subformula(F1).
true(F2,K) :- derive(F1,F2), true(F1,K).
true(F1,K) :- derive(F1,F2), true(F2,K).

derive(wprev(F),or(prev(F),initial)) :- subformula(wprev(F)).
derive(eventually_before(F),since(true,F)) 
  :- subformula(eventually_before(F)).
derive(always_before(F),trigger(false,F)) 
  :- subformula(always_before(F)).
derive(trigger(L,R),since(R,and(R,or(L,initial)))) 
  :- subformula(trigger(L,R)).
    \end{lstlisting}
\begin{align*}
L^k_{\gamma} \leftrightarrow L^k_{der(\gamma)} \text{ if } \gamma \text{ is a derived operator.}
\end{align*}
\end{frame}

\begin{frame}[t,fragile]{meta-telingo - Issues with Grounding}
For the input program
\begin{lstlisting}[language=clingo,numbers=none]
shoot :- initial.
next(not_load) :- initial.
next(next(shoot)) :- initial.
and(eventually_after(fail),forgetful)
  :- shoot, prev(since(not_load,shoot)).
\end{lstlisting}
We need additional external statements:
\begin{lstlisting}[language=clingo,numbers=none]
#external initial.
#external prev(since(not_load,shoot)).
\end{lstlisting}
\end{frame}

\begin{frame}[t,fragile]{meta-telingo - Issues with Grounding}
  \begin{itemize}
  \item Things become even trickier when we allow the input program to contain variables:
  \end{itemize}
\begin{lstlisting}[language=clingo,numbers=none]
shoot(potato_gun) :- initial.
next(shoot(potato_gun)) :- initial.
and(eventually_after(fail(X)),forgetful(Y)) 
  :- shoot(X), prev(since(not_load(Y,X),shoot(X))).
\end{lstlisting}
\begin{itemize}
\item Ignoring for now the issues with externals, let's think about
  the semantics of this program.
\item Any substitutions of \verb|Y| satisfies the final rule, so we
  derive \verb|forgetful(potato_gun)|.
\item The final rule should be considered unsafe.
\end{itemize}
\end{frame}

%%% SECTION 4: EXPERIMENT RESULTS
\section{Grounding Temporal Logic Programs}

\begin{frame}[t]{Safety}
\begin{definition}[Safety]
  Given an implication-free temporal formula $\varphi$, we say a
  subformula $\gamma \in subf(\varphi)$ is in an unsafe position in
  $\varphi$, if it occurs on the left hand side of a
  $\since/\until/\trigger/\release$ operator, on either side of an
  $\lor$, or as the operand of a $\wprevious/\wnext/\lnot$. Any
  subformula of such a $\gamma$ is also in an unsafe position.\\

  Any subformula of $\varphi$ that is not in an unsafe position is
  said to be in a \textit{safe position}.
  
  We call an atomic subformula of $\varphi$ safe iff it occurs in a
  safe position in $\varphi$.

  We call a variable $x_i$ occurring in a temporal rule
  $\varphi = \alwaysF(B \rightarrow H)$ safe, iff it has an occurrence
  in a safe atomic subformula of $B$. 

  \\We call a temporal rule safe iff all of it's variables are safe
  . We call a universal temporal rule $\forallx \psi$ safe iff $\psi$
  is safe. We call a program safe, if all of it's rules are safe.
\end{definition}
\end{frame}

\begin{frame}[t,fragile]{Safety}
\begin{theorem}[Domain Independence]
  Let $\varphi$ be a safe universal temporal rule. Suppose
  $C \subseteq C'$. A total QTHT model $\qttandt$ is a temporal
  equilibrium model of $\ground{C}{\varphi}$ iff it is a temporal
  equilibrium model of $\ground{C'}{\varphi}$.
\end{theorem}
\begin{itemize}
\item Under this definition, the previous program with variables is
  unsafe. The following in a modification which is safe.
\end{itemize}
  \begin{lstlisting}[language=clingo,numbers=none]
loader(john).
shoot(potato_gun) :- initial.
next(shoot(potato_gun)) :- initial.
and(eventually_after(fail(X)),forgetful(Y)) 
  :- shoot(X), loader(Y), prev(since(not_load(Y,X),shoot(X))).
  \end{lstlisting}
\end{frame}

\begin{frame}[t]{Grounding}
  \begin{itemize}
  \item As seen before, grounding of the input program is not
    straightforward due to the simplifications made by gringo.
  \item Externals can fix the problem on a case-by-case basis.
  \item But, is there a general solution?
  \item Our method will construct a simplified version of the original
    program, which we can ground successfully. We will then use the
    ground atoms of this program to ground the original.
  \end{itemize}
\end{frame}

\begin{frame}[t]{Grounding}
  \begin{itemize}
  \item Given a temporal rule
    $r=\forall x_1, \dots, \forall x_n \alwaysF(B \rightarrow H)$, we define
    it's set of transformed rules $\delta(r)$.
  \item Let $p_1,\dots, p_n$ be the safe atomic subformulas occurring
    in $B$, and $h_1, \dots, h_m$ the atomic subformulas occurring in
    H. Then:
\begin{equation*} \delta(r) \defeq \{ \forall x_1, \dots, \forall x_n
\alwaysF(\bigwedge_{j=1}^{n}p_j \rightarrow h_i) \mid
\rangecc{i}{1}{m})\}
\end{equation*}
  \end{itemize}

\begin{proposition}\label{prop:simplified-facts}
  Let $\qttandt$ be a temporal equilibrium model of a safe universal
  temporal program $\Gamma$, and
  $\langle \dsigma, \bm{J},\bm{J} \rangle$ the unique temporal
  universal equilibrium model of $\delta(\Gamma)$. Then,
  $\bm{T} \subseteq \bm{J}$.
\end{proposition}
\end{frame}

\begin{frame}[t]{Grounding}
  \begin{itemize}
  \item Notice that $\delta(r)$ is a set of positive rules (with each
    enclosed an always operator).
  \item We can therefore obtain the same $J$ by solving the regular
    program $\delta^{\prime}(\Gamma)$, where
\begin{equation*} \delta^{\prime}(r) \defeq \{ \forall x_1, \dots \forall x_n
  \bigwedge_{j=1}^{n}p_j \rightarrow h_i) \mid \rangecc{i}{1}{m})\}
\end{equation*}
  \item This is a positive program, and as such can be fully evaluated during grounding.
  \end{itemize}
\end{frame}

\begin{frame}[t,fragile]{Grounding}
  Let the safe universal temporal program $\Gamma$ be the following:
  \begin{lstlisting}[language=clingo,numbers=none]
loader(john).
shoot(potato_gun) :- initial.
next(shoot(potato_gun)) :- initial.
and(eventually_after(fail(X)),forgetful(Y)) 
 :- shoot(X), loader(Y), prev(since(not_load(Y,X),shoot(X))).
  \end{lstlisting}
  $\delta^{\prime}(\Gamma)$:
  \begin{lstlisting}[language=clingo,numbers=none]
    loader(john).
    shoot(potato_gun).
    fail(X) :- shoot(X), loader(Y), shoot(X).
    forgetful(Y) :- shoot(X), loader(Y), shoot(X).
  \end{lstlisting}
  $$
  J = \{ loader(john), shoot(potato\_gun), fail(potato\_gun), forgetful(john) \}
  $$
\end{frame}

\begin{frame}[t,fragile]{Grounding}
  \begin{itemize}
  \item To ground the program $\Gamma$ using
    $J=SM(\delta^{\prime}(\Gamma))$, we will make clever use of
    external statements.
  \item External statements may also have bodies, in which case the
    external statement is ground like a rule, and the obtained ground
    heads are marked as external, with bodies being discarded.
  \item First, we will generate one external statement for each rule
    in $r' \in \delta^\prime(\Gamma)$, with the head and bodies of the
    external statement being the same as $r'$
    \begin{itemize}
    \item This will cause the grounder to mark all atoms in the unique
      stable model $J$ of $\delta^\prime(\Gamma)$ as external.
    \end{itemize}
  \end{itemize}
\end{frame}

\begin{frame}[t,fragile]{Grounding}
  \begin{itemize}
  \item Second, for each rule $r \in \Gamma$ we will generate one
    external statement for each non-atomic formula $\gamma$ in the
    body of $r$, the head of which will be $\gamma$, and the body will
    be the body of (any) $r' \in \delta^{\prime}(r)$.
    \begin{itemize}
    \item This second set of rules will ensure that formulas in the
      bodies of rules with occurrences of atoms from $J$ will also be
      protected from simplification.
    \end{itemize}
  \end{itemize}
\end{frame}

\begin{frame}[t,fragile]{Grounding}
Given the program:
\begin{lstlisting}[language=clingo,numbers=none]
loader(john).
shoot(potato_gun) :- initial.
next(shoot(potato_gun)) :- initial.
and(eventually_after(fail(X)),forgetful(Y)) 
 :- shoot(X), loader(Y), prev(since(not_load(Y,X),shoot(X))).
  \end{lstlisting}
The generated externals would be:
  \begin{lstlisting}[numbers=none]
#external loader(john).
#external shoot(potato_gun).
#external fail(X) : shoot(X); loader(Y); shoot(X).
#external forgetful(Y) : shoot(X); loader(Y); shoot(X).

#external initial.
#external prev(since(not_load(Y,X),shoot(X))) 
  : shoot(X); loader(Y); shoot(X).
  \end{lstlisting}
\end{frame}

\section{Conclusions}

\begin{frame}
  \begin{itemize}
  \item In this thesis, we saw an implementation of a temporal answer
    set solver from first principles.
  \item We proved it's correctness for ground temporal programs.
  \item We had a careful discussion of the semantic issues around
    grounding temporal programs.
  \item We extended the definition of safety for temporal programs,
    and showed that it has the desired properties.
  \item Finally, we introduced a method that can ground first order temporal programs.
  \end{itemize}
\end{frame}

\end{document}
