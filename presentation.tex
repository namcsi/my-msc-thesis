\documentclass[aspectratio=169,xcolor=svgnames]{beamer} 

\usepackage[utf8]{inputenc}
\usepackage[font=footnotesize]{caption}
\usepackage{hyperref}

% Include image package
\usepackage{graphicx}

\usefonttheme[onlymath]{serif}

\definecolor{universityblue}{RGB}{0, 48, 135}
\definecolor{secondaryblue}{RGB}{65, 182, 230}
\definecolor{universitygray}{RGB}{137, 141, 141}
\definecolor{lightgray}{RGB}{180, 180, 180}
\colorlet{verylightgray}{lightgray!20}

\useinnertheme{circles}
\useoutertheme[right]{sidebar}
\usecolortheme{sidebartab}
\usecolortheme[named=universityblue]{structure}

\setbeamercolor{title}{bg=universityblue, fg=white}
\setbeamercolor{author in head/foot}{bg=universitygray, fg=white}
\setbeamercolor{sidebar right}{bg=universityblue, fg=white}
\setbeamercolor{title in sidebar}{fg=white}
\setbeamercolor{author in sidebar}{fg=white}
\setbeamercolor{section in sidebar shaded}{fg=lightgray}
\setbeamercolor{section in sidebar}{fg=white,bg=secondaryblue}
\setbeamercolor{subsection in sidebar shaded}{fg=lightgray}
\setbeamercolor{subsection in sidebar}{fg=white,bg=secondaryblue}
\setbeamercolor{navigation symbols dimmed}{fg=white}

\setbeamercolor{frametitle}{fg=universityblue}

\setbeamercolor{block title}{use=example text,fg=white,bg=universityblue}
\setbeamercolor{block body}{parent=normal text,use=block title example,bg=verylightgray}
% \setbeamercolor{block title}{fg=white, bg=green!50!black}
% \setbeamercolor{block body}{fg=black, bg=green!10}

\setbeamertemplate{navigation symbols}{}
\makeatletter
\setbeamertemplate{footline}{%
    \leavevmode%
    \hbox{%
        \begin{beamercolorbox}[wd=0.6\paperwidth,ht=2.25ex,dp=1ex,center]{author in head/foot}%
            \usebeamerfont{title in head/foot}%
            \insertshortauthor ~ -- ~\insertshorttitle
        \end{beamercolorbox}%
        \begin{beamercolorbox}[wd=0.3\paperwidth,ht=2.25ex,dp=1ex,center]{author in head/foot}%
            \usebeamerfont{title in head/foot}%
            \vspace{-1.5pt}
            \insertslidenavigationsymbol%
            \insertframenavigationsymbol%
            \insertsubsectionnavigationsymbol%
            \insertsectionnavigationsymbol%
            \insertdocnavigationsymbol%
            \insertbackfindforwardnavigationsymbol%
        \end{beamercolorbox}%
        \begin{beamercolorbox}[wd=0.1\paperwidth,ht=2.25ex,dp=1ex,center]{author in head/foot}%
            \usebeamerfont{title in head/foot}\insertframenumber/\inserttotalframenumber
        \end{beamercolorbox}%
    }
}
\makeatother

% University logo and wordmark
\logo{\includegraphics[scale=0.015]{LogoUP}}
\titlegraphic{\includegraphics[height=2cm]{LogoUP}}



% Section title page
\AtBeginSection[]{
    \begin{frame}
        \centering
        \begin{beamercolorbox}[sep=16pt,center]{title}
            \usebeamerfont{title}\insertsectionhead\par%
        \end{beamercolorbox}
    \end{frame}
}

% \AtBeginEnvironment{theorem}{%
%   \setbeamercolor{block title}{fg=white,bg=universityblue}
%   \setbeamercolor{block body}{parent=normal text,use=block title example,bg=block title example.bg!10!bg}
% }

% Superscript removal for hyperref recognition in pdf string
\newcommand{\supersc}[1]{\texorpdfstring{\textsuperscript{#1}}{}}


\usepackage[normalem]{ulem}
\usepackage{xcolor}
\usepackage[export]{adjustbox}
\usepackage{listings}
\usepackage{tikz}
\usepackage{tikz-qtree}
\usepackage{graphicx}
\usepackage{caption}
\usepackage{subcaption}

\input{include/asp-macros/dtel}
\input{include/asp-macros/logics}
\input{include/asp-macros/systems}
\input{include/asp-macros/ht}
\input{include/asp-macros/automata}
\input{include/asp-macros/listings}



\newcommand{\Next}{\text{\rm \raisebox{-.5pt}{\Large\textopenbullet}}}  % {{\ensuremath{\circ}}}
\renewcommand{\wnext}{\ensuremath{\widehat{\Next}}}

\newcommand{\eqdef}{\ensuremath{=^{def}}}

\newcommand{\A}{\ensuremath{\mathcal{A }}}

\newcommand{\fpbf}{\ensuremath{\bm{\mathit{false}}}}
\newcommand{\tpbf}{\ensuremath{\bm{\mathit{true}}}}


% temporal formalities

\newcommand{\s}[1]{\ensuremath{``#1"\,}}
\newcommand{\interp}[1]{\ensuremath{\lfloor #1 \rceil\,}}

\newcommand{\last}{\ensuremath{\mathit{last}}}
\newcommand{\nnf}{\ensuremath{\mathit{nnf}\,}}
\newcommand{\AFW}{\ensuremath{\mathrm{AFW}}}

\newcommand{\defeq}{\ensuremath{\overset{\mathit{def}}{=}}}

\newcommand{\handt}{\ensuremath{\langle H,T \rangle}}
\newcommand{\tandt}{\ensuremath{\langle T,T \rangle}}
\newcommand{\thandt}{\ensuremath{\langle \bm{H},\bm{T} \rangle}}
\newcommand{\ttandt}{\ensuremath{\langle \bm{T},\bm{T} \rangle}}
\newcommand{\pbf}{\textit{pbf}}

\newcommand{\PV}{\ensuremath{\mathcal{P}}}
\newcommand{\Rel}[2]{\parallel\!#1\!\parallel^{#2}}



\usepackage[framemethod=tikz]{mdframed}
\usepackage[font=footnotesize]{caption}

% For exported plots and tables
\usepackage{booktabs}
\usepackage[utf8]{inputenc}
\usepackage{pgfplots}
\DeclareUnicodeCharacter{2212}{−}
\usepgfplotslibrary{groupplots,dateplot}
\usetikzlibrary{patterns,shapes.arrows}
\pgfplotsset{compat=newest}

\usepackage{subcaption}
\usepackage[T1]{fontenc}
\usepackage[toc,page]{appendix}
\usepackage{amsmath,amsthm,amssymb}
\usepackage[inline]{enumitem}
\usepackage{tabularx}

\lstdefinelanguage{clingo}{
  keywordstyle=[1]\usefont{OT1}{cmtt}{m}{n},%
  keywordstyle=[2]\textbf,%
  keywordstyle=[3]\usefont{OT1}{cmtt}{m}{n},%\textit
  alsoletter={\#,\&},%
  keywords=[1]{not,from,import,def,if,else,return,while,break,and,or,for,in,del,and,class},%
  keywords=[2]{\#const,\#show,\#minimize,\#base,\#theory,\#count,\#external,\#program,\#script,\#end,\#heuristic,\#edge,\#project,\#show},%
  keywords=[3]{&,&dom,&sum,&diff,&show,&minimize},%
  morecomment=[l]{\#\ },%
  morecomment=[l]{\%\ },%
  commentstyle={\color{darkgray}}%
}

\lstset
{ %Formatting for code in appendix
    literate={~}{{\fontfamily{ptm}\selectfont \textasciitilde}}1,
    language=clingo,
    basicstyle=\ttfamily\footnotesize,
    numbers=left,
    stepnumber=1,
    showstringspaces=false,
    tabsize=1,
    breaklines=true,
    breakatwhitespace=false,
    frame=single,
    xleftmargin=2em
}

% \lstset{
%     basicstyle=\ttfamily,
%     numbers=left,
%     firstnumber=1,
%     numberfirstline=true,
%     breaklines=true,                    % sets automatic line breaking
% %   breakatwhitespace=false,            % sets if automatic breaks should only happen at whitespace
%     prebreak = \raisebox{0ex}[0ex][0ex]{\ensuremath{\hookleftarrow}}, % just as an example
% }

%%%%%%%%%%%%%%%%%%%%% Formats %%%%%%%%%%%%%%%%%%%%%%
\definecolor{darkgreen}{rgb}{0,0.5,0}
\definecolor{darkblue}{HTML}{00305E}
\definecolor{darkred}{rgb}{0.8,0,0}
\newtheoremstyle{theoremstyle_space}
{7pt} % Space above
{7pt} % Space below
{} % Body font
{} % Indent amount
{\bfseries} % Theorem head font
{} % Punctuation after theorem head
{\newline} % Space after theorem head: \newline = linebreak
{\thmname{#1}\thmnumber{ #2} \thmnote{(\emph{#3})}}% Theorem head spec (can be left empty, meaning `normal')

\theoremstyle{theoremstyle_space}
\newtheorem{definition}{Definition}
\newtheorem{theorem}{Theorem}
\newtheorem{proposition}{Proposition}
\newtheorem{lemma}{Lemma}
\newtheorem{corollary}{Corollary}

\newcounter{example}
\NewCommandCopy{\proofqedsymbol}{\qedsymbol}% save the default
\newcommand{\exampleqedsymbol}{$\triangle$}% for end of examples

% ensure that proof has the standard symbol
\AtBeginEnvironment{proof}{\renewcommand{\qedsymbol}{\proofqedsymbol}}
\newtheorem{example}{Example}
\AtBeginEnvironment{example}{%
  \pushQED{\qed}\renewcommand{\qedsymbol}{$\triangle$}%
}
\AtEndEnvironment{example}{\popQED\endexample}

% PRESENTATION PURPOSE
    % * Explaining the problem well
    % * Showing clearly what your encoding does
    % * Justifying your design decisions
    % * Using tests and benchmarks to compare with the other groups

% DOCUMENT INFORMATION
\title[Temporal AST via Metaprogramming]{Implementing a Temporal Answer Set Solver using Metaprogramming Techniques}
\subtitle{MSc. Thesis Defense/IM presentation}
\author{Amadé Nemes}

\institute[]{%
    ~nemes@uni-potsdam.de \\
    ~Cognitive Systems MSc. \\
    ~Universität Potsdam, Institut für Informatik \\
}
\date{\today}

\begin{document}

% TITLE PAGE WITHOUT SIDEBAR
\begingroup
    \makeatletter
    \setlength{\hoffset}{.5\beamer@sidebarwidth}
    \makeatother
    \begin{frame}[plain]
        \maketitle
    \end{frame}
\endgroup

% TABLE OF CONTENTS
\begin{frame}
\frametitle{Table of Contents}
\tableofcontents
\end{frame}

%%% SECTION 1: INTRODUCTION
\section{Introduction}


\section[Temporal Here-and-There]{THT}

%% FRAME: How does MDD-SAT work?
\begin{frame}[t]{Grammar}
A temporal formula over a set of atoms $\A$ is defined by the
following grammar:

\begin{align*}
    \varphi ::=   \; a &\mid \bot \mid \varphi_1 \otimes \varphi_2 \mid\\
  \initially &\mid \previous \varphi \mid \wprevious \varphi \mid \eventuallyP \varphi \mid
  \alwaysP \varphi \mid \varphi_1 \since \varphi_2 \mid \varphi_1 \trigger \varphi_2 \mid\\
  \finally &\mid \Next \varphi \mid \wnext \varphi \mid \eventuallyF \varphi \mid
  \alwaysF \varphi \mid \varphi_1 \until \varphi_2 \mid \varphi_1 \release \varphi_2
\end{align*}
where $\otimes \in \{ \wedge, \vee, \to \}$
\end{frame}

\begin{frame}[t]{Pronunciation Guide}

\[
  \begin{array}{l l|l l}
    \initially & \text{\emph{initial}} & \finally & \text{\emph{final}}\\
    \previous & \text{\emph{previous}} & \Next & \text{\emph{next}}\\
    \wprevious & \text{\emph{weak previous}} & \wnext & \text{\emph{weak next}}\\
    \eventuallyP & \text{\emph{eventually before}} & \eventuallyF & \text{\emph{eventually after}}\\
    \alwaysP & \text{\emph{always before}} & \alwaysF & \text{\emph{always after}}\\
    \since & \text{\emph{since}} & \Next & \text{\emph{until}}\\
    \trigger & \text{\emph{trigger}} & \release & \text{\emph{release}}
\end{array}
\]

\end{frame}

\begin{frame}[t]{HT Traces}

\begin{definition}[Trace]
  A trace $\bm{T}$ of length $\lambda$ over signature $\A$ a (possibly
  infinite) sequence of sets of atoms.
  $\bm{T} \in \A^{\intervco{0}{\lambda}}$.
\end{definition}

\begin{definition}[HT-Trace]
  An HT trace $\thandt$ of length $\lambda$ is an ordered pair of
  traces, where for each $\rangeco{i}{0}{\lambda}$
  $H_i \subseteq T_i$. An $HT$ trace is total iff $\bm{H} = \bm{T}$
\end{definition}

Note: in what follows we will always assume that $\lambda$ is
finite.

\end{frame}

\begin{frame}[t]{THT Satisfaction}

\begin{definition}[THT Satisfaction]
An HT-trace $\thandt$ of length $\lambda$ over signature $\A$
  satisfies a temporal formula $\varphi$ at time point
  $\kinlambda$, written as $\thandt,k \models \varphi$ if
  the following condition holds, where we use the shorthand
  $\bf{M} =$ $\thandt$.
\begin{enumerate}
  \item $\bf{M}, k \not \models \bot$
  % \item $\bf{M},k \models$ $a$ if $a \in H_{k}$ for any atom $a \in \mathcal{A}$
  % \item $\bf{M},k \models \varphi \wedge \psi$ iff $\bf{M}, k \models \varphi$ and $\bf{M}, k \models \psi$
  % \item $\bf{M},k \models \varphi \vee \psi$ iff $\bf{M}, k \models \varphi$ or $\bf{M}, k \models \psi$
  % \item $\bf{M},k \models \varphi \rightarrow \psi$ iff 
  %   $\langle \bm{H}^{\prime},\bm{T} \rangle, k \not \models \varphi$ 
  %   or $\langle \bm{H}^{\prime},\bm{T} \rangle, k \models \psi$, 
  %   for all $\bm{H}^{\prime} \in\{\bm{H}, \bm{T}\}$
  % \item $\bf{M},k \models \initially$ iff $k=0$
  % \item $\bf{M},k \models \previous \varphi$ iff $k>0$ and $\bf{M},k-1 \models \varphi$
  % \item $\bf{M},k \models \wprevious \varphi$ iff $k=0$ or $\bf{M},k-1 \models \varphi$
  % \item $\bf{M},k \models \eventuallyP \varphi$ iff $j \models \varphi$ for some $\rangecc{j}{0}{k}$  
  % \item $\bf{M},k \models \alwaysP \varphi$ iff $j \models \varphi$ for all $\rangecc{j}{0}{k}$  
  % \item $\bf{M},k \models \varphi \since \psi$ iff for some $j \in[0 . . k]$, we have $\bf{M}, j \models \psi$ and $\bf{M},i \models \varphi$ for all $\rangeoc{i}{j}{k}$
  % \item $\bf{M},k \models \varphi \trigger \psi$ iff for all $j \in[0 . . k]$, we have $\bf{M}, j \models \psi$ or $\bf{M}, i \models \varphi$ for some $\rangeoc{i}{j}{k}$
  % \item $\bf{M},k \models \finally$ iff $k=\lambda - 1$
  % \item $\bf{M},k \models \Next \varphi$ iff $k+1<\lambda$ and $\bf{M}, k+1 \models \varphi$
  % \item $\bf{M},k \models \wnext \varphi$ iff $k=\lambda-1$ or $\bf{M},k+1 \models \varphi$
  % \item $\bf{M},k \models \eventuallyF \varphi$ iff $j \models \varphi$ for all $\rangeco{j}{k}{\lambda}$  
  % \item $\bf{M},k \models \alwaysF \varphi$ iff $j \models \varphi$ for some $\rangeco{j}{k}{\lambda}$  
  % \item $\bf{M}, k \models \varphi \until \psi$ iff for some $j \in[k . . \lambda)$, we have $\bf{M}, j \models \psi$ and $\bf{M}, i \models \varphi$ for all $\rangeco{i}{k}{j}$
  % \item $\bf{M}, k \models \varphi \release \psi$ iff for all $\rangeco{j}{k}{\lambda}$, we have $\bf{M}, j \models \psi$ or $\bf{M},
  %   i=\varphi$ for some $\rangeco{i}{k}{j}$
\end{enumerate}
\end{definition}

\end{frame}


%%% SECTION 3: IMPLEMENTATION
\section{Implementation}

\begin{frame}[t,fragile]{Dummy Example} 

\begin{lstlisting}[language=clingo,numbers=none]
shoot(potato_gun) :- initial.
next(shoot(potato_gun)) :- initial.
and(eventually_after(fail(X)),forgetful(Y)) 
  :- shoot(X), prev(since(not_load(Y,X),shoot(X))).
\end{lstlisting}

\end{frame}



\begin{frame}[fragile=singleslide]{Incremental Solving - Encoding Structure}
\begin{itemize}
    \item Incremental solving loop of MDD-SAT and SMT-CBS can be implemented via clingo's multi-shot solving capabilities.
    \item Following the incremental solving approach laid out in \cite{kaminski2021build} all our encodings are partitioned into three distinct subprograms:
    \begin{itemize}
        \item \verb|#program base.| 
        static knowledge independent of step parameter \verb|t| (e.g.: layout, agent's initial and goal vertices).
        \item \verb|#program cumulative(t).| cumulative information which remains true for all further steps of \verb|t| (e.g.: positions of agents at time step \verb|t|).
        \item \verb|#program check(t).| volatile information, which is only true for the current maximal time step \verb|t|, and is invalidated once \verb|t| is incremented to \verb|t+1|. (e.g.: costs of agent). Achieved by adding external atom \verb|query(t)| to the body of each rule in the subprogram, and controlling truth value via API.
        \end{itemize}
\end{itemize}
\end{frame}

\begin{frame}[fragile=singleslide]{Incremental Solving - Solving Loop}
\begin{itemize}
    \item We have a base MAPF encoding partitioned into these 3 parts
    \item Each of our solvers extends this base encoding by adding additional rules.
    \item Solving occurs using IncSolver (and inheritors), which controls the solving loop.
    \begin{itemize}
        \item At first time step, grounds \verb|#program base| and     \verb|#program check(0)|, assigns \verb|query(0)| to true and          solves.
        \item As subsequent steps time steps \verb|t|, grounds \verb|#program cumulative(t)| and \verb|#program check(t)|, sets external \verb|{query(t-1)| to           false and \verb|query(t)| to true and solves.
        \item Halts and returns solution if solving returns SAT.
    \end{itemize}
\end{itemize}
\end{frame}

\begin{frame}[fragile=singleslide]{Incremental Solving - Inheritance}
\begin{itemize}
    \item Solving behaviour can be customized/extended via inheritance from base class \verb|IncSolver|.
\end{itemize}
\begin{figure}
\centering
\begin{tikzpicture}
\tikzset{level distance=50pt}
\Tree [.\node[draw]{IncSolver};
    [.\node[draw]{MddSatSolver};
    [.\node[draw]{SmtCbsExtSolver};]
    [.\node[draw]{SmtCbsPropSolver};]]
]
\end{tikzpicture}
\caption{\label{Inheritence Tree}
Inheritence tree of our ASP-based MAPF solvers.
}
\end{figure}
\end{frame}



\begin{frame}[fragile=singleslide]{Base MAPF encoding - base}
\begin{verbatim}
#program base.
grid(X,Y) :- init(object(node,_), value(at,(X,Y))).
move(-1,0). move(1,0). move(0,-1). move(0,1). move(0,0).
robot(R) :- init(object(robot,R),value(at,(X,Y))).
position(R,X,Y,0) :- init(object(robot,R),value(at,(X,Y))).
goal(R,X,Y) :- init(object(destination,R),value(at,(X,Y))).
ngoal(R,0) :- goal(R,X,Y), not position(R,X,Y,0).
\end{verbatim}
\end{frame}

\begin{frame}[fragile=singleslide]{Base MAPF encoding - cumulative}
\begin{verbatim}
#program cumulative(t).
{ position(R,X+U,Y+V,t) : grid(X+U,Y+V), move(U,V) } = 1 :- 
  position(R,X,Y,t-1).
ngoal(R,t) :- goal(R,X,Y), not position(R,X,Y,t).
\end{verbatim}
\begin{itemize}
    \item We do not encode conflict handling - this is added in for each solver individually as they differ.
\end{itemize}
\end{frame}

\begin{frame}[fragile=singleslide]{Base MAPF encoding - check}
\begin{verbatim}
#program check(t).
#external query(t).
:- ngoal(R,t), query(t).
robot_cost(R,N,t) :- N = 
  #max{ T : ngoal(R,T-1), not ngoal(R,T), 0 <= T, T <= t },  
  robot(R), query(t).
soc(M,t) :- M = #sum{ N,R : robot_cost(R,N,t) }, query(t).
#minimize{ N : soc(N,t) }.
\end{verbatim}
\end{frame}

\begin{frame}[fragile=singleslide]{MddSatSolver}
\begin{itemize}
    \item Enhance base encoding by forbidding conflicts.
\begin{verbatim}
#program cumulative(t).
:- position(R1,X,Y,t), position(R2,X,Y,t), R1 > R2.
:- position(R1,X1,Y1,t), position(R2,X2,Y2,t), 
   position(R1,X2,Y2,t-1), position(R2,X1,Y1,t-1), R1 > R2.
\end{verbatim}
    \item Add upper bound on SoC.
\begin{verbatim}
#program check(t).
ub_soc(N,t) :- ub_soc(N-1,t-1).
:- ub_soc(N,t), soc(M,t), M > N, query(t).
\end{verbatim}
    \item Initial upper bound is found in extended setup step to find individual shortest paths, adding initial fact \verb|ub(M_0,t_0).|.
\end{itemize}
\end{frame}

\begin{frame}[fragile=singleslide]{SmtCbsExtSolver - Motivation}
\begin{itemize}
    \item Constraints forbidding conflicts complicate search space - possibly unnecessarily, if agents don't collide in practice.
    \item By relaxing constraints, and adding them lazily we may find a solution faster.
    \item Convincing results for SMT-CBS vs MDD-SAT in literature \cite{ijcai2019p164}.

\end{itemize}
\end{frame}

\begin{frame}[fragile=singleslide]{SmtCbsExtSolver - Implementation}
\begin{itemize}
    \item Idea: implement lazy conflict forbidding constraints by making constraints contingent on external atom.
    \begin{verbatim}
#external constraint(R1,R2,X,Y,t) : 
  robot(R1), robot(R2), R1 < R2, grid(X,Y).
:- position(R1,X,Y,t), position(R2,X,Y,t),
  constraint(R1,R2,X,Y,t).

#external constraint(R1,R2,X1,Y1,X2,Y2,t) : 
  robot(R1), robot(R2), R1 < R2, grid(X1,Y1), grid(X2,Y2).
:- position(R1,X1,Y1,t), position(R2,X2,Y2,t), 
   position(R1,X2,Y2,t-1), position(R2,X1,Y1,t-1),
   constraint(R1,R2,X1,Y1,X2,Y2,t).
    \end{verbatim}

\end{itemize}
\end{frame}

\begin{frame}[fragile=singleslide]{SmtCbsExtSolver - Implementation}
\begin{itemize}
    \item Also include SoC upper bound encoding.
    \item Extend MddSatSolver's solving loop by adding check for conflicts after solving call for fixed time step.
    \begin{itemize}
        \item If no conflicts are found, valid solution found, break from outer solving loop.
        \item If conflicts are found, corresponding external atoms are set to true, and we re-solve for the same time step.
    \end{itemize}

\end{itemize}
\end{frame}

\begin{frame}[fragile=singleslide]{SmtCbsPropSolver - Motivation}
\begin{itemize}
    \item While using external atoms is very straightforward, performance leaves something to be desired.
    \item Ground instantiation of constraints is rather large, and grounding still needs to be done for each possible constraint.
    \item In preliminary experiments, this incurred a noticeable time penalty when grounding iteratively.
    \item Could we add constraints directly to the program, without having to ground every possible one?

\end{itemize}
\end{frame}

\begin{frame}[fragile=singleslide]{SmtCbsPropSolver - Implementation}
\begin{itemize}
    \item clingo 5.0 introduced the Propagator interface to customize the solving process.
    \item Registered Propagator can be used to perform lazy checking after  total assignment of solver literals is found.
    \item This allows us to lazily check for conflicts, and add forbidding nogoods directly to the solver if conflict is found.
    \item Thus, no rules encoding constraints are necessary, base encoding is extended only by SoC upper bound constraints.
\end{itemize}
\end{frame}


%%% SECTION 4: EXPERIMENT RESULTS
\section{Experiment Results}

%% FRAME: What happens when we increase k?
\begin{frame}[t]{How did each algorithm perform?}
    \begin{table}
    \centering
    \caption{Average run time for each algorithm based on the number of agents.}
        \begin{tabular}{cccc}
        Agents & MDD-SAT & PROP   & EXT     \\
        5      & 0.15    & 0.30   & 0.33    \\
        6      & 0.26    & 0.65   & 0.65    \\
        7      & 2.10    & 8.18   & 7.25    \\
        8      & 57.62   & 118.07 & 294.78  \\
        9      & 19.02   & 31.46  & 50.65   \\
        10     & 151.69  & 601.17 & 710.05  
        \end{tabular}
    \label{Table_1}
    \end{table}
\end{frame}

%% FRAME: How does this compare to the performance of CBS?
\begin{frame}[t]{How does this compare to the class?}
    \begin{itemize}
        \item Our instances
        \begin{itemize}
            \item ours finds the optimal solution
            \item occasionally, other groups find optimal solution but not always
        \end{itemize} 
        \item Other groups' instances
        \begin{itemize}
            \item ours finds optimal solution on half of instances
            \item timeout threshold is reached on other half
        \end{itemize}
        \item Our average results
        \begin{itemize}
            \item 1.43 seconds for MDD-SAT
            \item 3.26 secondes for PROP
            \item 6.08 seconds for EXT
        \end{itemize}
        
    \end{itemize}
\end{frame}
    
\section{Conclusion}
\begin{frame}[t]{Conclusions}
    \begin{itemize}
        \item ASP can be used to implement reduction-based solvers in a concise, performant manner.
        \item For SMT-CBS, utilizing the propagator yields significant advantages over grounding external atoms.
        \item This advantage grows as the environment becomes more complicated (more ground rules).
        \item Even in simpler environments, the baseline approach remains the best
    \end{itemize}
\end{frame}


% REFERENCES
\begin{frame}[allowframebreaks]\frametitle{References}
    \bibliographystyle{ieeetr}
    \bibliography{references}
\end{frame}

\end{document}
